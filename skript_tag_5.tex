\section{Dictionaries}

\begin{frame}
\begin{block}{Problemstellung}
\vspace{2pt}
In einem Notenrechner wie eben, sollen nicht nur die Noten gespeichert werden, sondern auch das Fach, in dem die Note erreicht wurde. 

\vspace{8pt}

Wie macht man das? 
\end{block}
\end{frame}

\begin{fragile}{}
\begin{block}{Lösung}
	\begin{minted}{python}
	# ...
	grades = { "Deutsch": 14, "Mathematik": 8, "Biologie": 11, "Sport": 13}
	
	key = input("Welche Note möchtest Du wissen?")
	
	print(f"Deine Note in { key } ist { grades[key] } Punkte")
	\end{minted}
\end{block}
\end{fragile}

\begin{fragile}
	
	\metroset{block=fill}
	\begin{block}{Struktur eines \emph{Dictionaries}}
		\vspace{2pt}
		\large
		\texttt{my\_dict = }\pause {\Large\texttt{\{}}\pause 
		\texttt{key\_1}\pause\texttt{:}\pause\texttt{value\_1}\pause,
		\pause 
		\texttt{key\_2:value\_2}, \pause 
		\dots   
		, \texttt{key\_n:value\_n}\pause \Large{\texttt{\}}}
	\end{block}
	\pause 
	
	Das \emph{Dictionary} \py{my_dict} enthält Schlüssel-Wert-Paare (\emph{key-value-pairs}). Die Schlüssel müssen eindeutig und unveränderlich sein (z.B. vom Typ \pybw{string} oder \pybw{int}). Die Werte dürfen beliebige Datentypen sein. 
	
\end{fragile}

\begin{fragile}

\begin{block}{Good to know}
	\pause
\begin{itemize}[<+->]
\item Zur besseren Übersichtlichkeit werden Dictionaries oftmals wie folgt formatiert: 
\begin{minted}{python}
grades = { 
  "Deutsch": 14, 
  "Mathematik": 8, 
  "Biologie": 11, 
  "Sport": 13
}
\end{minted}
\item Dictionaries sind mutable, können also verändert werden. 
\item Dictionaries besitzen keine vernünftige Anordnung und können nicht geordnet werden. 
\item Ein Dictionary kann leer sein. 
\end{itemize}
\end{block}
\end{fragile}

\begin{fragile}
\begin{itemize}
\item Oftmals bietet es sich an, statt einem Dictionary eine Liste von Dictionaries zu verwenden:
\begin{minted}{python}
grades = [
  { 
    "subject": "Deutsch", 
    "grade": 14,
    "is_major": True
  },
  # ... 
  {
    "subject": "Sport",
    "grade": 11, 
    "is_major": False
  }
]
\end{minted}
\end{itemize}
\end{fragile}



\begin{frame}


\begin{block}{Auf Dictionary-Elemente zugreifen}
	
	\vspace{2pt}
	
	Sei \py{my_dict = {"a": 5, "b": 8}}.
	
	\pause
	
	Mit der Syntax \py{my_dict["a"]} kann man den Wert an der Stelle \py{"a"} auslesen. 
	
	\pause 
	
	Mit der Syntax \py{my_dict["a"] = 12} kann  man einzelne Werte des Dictionaries verändern. 
	
	\pause 
	
	Auf diese Weise können auch ganz neue Paare hinzugefügt werden. Zum Beispiel: \py{my_dict["c"] = -2}. 
\end{block}
\end{frame}


\begin{frame}{Übung}

\begin{block}{Dictionary manipulieren}
	\vspace{2pt}
	Gegeben sei das folgende Dictionary: 
	
	\py{grades = { "Mathe": 8, "Bio": 11, "Sport": 13}} 
	
	Bestimme die Durchschnittsnote dieser drei Fächer. Verbessere danach Deine Mathenote um einen Punkt und füge noch eine weitere Note für Englisch hinzu (Abfrage über Konsole). Gib danach erneut den Durchschnitt an.  
\end{block}

\end{frame}


\begin{frame}<beamer:0>[fragile]{Lösung}

\begin{solutionblock}{Dictionary manipulieren}
\begin{minted}{python}
grades = { "Mathe": 8, "Bio": 11, "Sport": 13}

grades_sum = grades["Mathe"] + grades["Bio"] + grades["Sport"]
average = grades_sum/len(grades)
print(f"Der Durchschnitt ist {average} Punkte")

grades["Mathe"] += 1

eng_grade = input("Welche Note hast Du in Englisch? ")
eng_grade = int(eng_grade)
grades["Englisch"] = eng_grade

grades_sum += grades["Englisch"]
average = grades_sum/len(grades)
print(f"Der Durchschnitt ist {average} Punkte")
\end{minted}
\end{solutionblock}
\end{frame}



\begin{frame}
\begin{block}{Einen Eintrag aus einem Dictionary entfernen}
\vspace{2pt}
Wie bei Listen, kann man mittels \py{del}-Statement einen Eintrag aus einem Dictionary entfernen: 

\py{del my_dict["a"]}	

\end{block}	



\end{frame}	

\begin{fragile}
\begin{block}{Was wird hier passieren?}
\vspace{2pt}
\begin{minted}{python}
grades = { "Mathe": 8, "Bio": 11, "Sport": 13}
new_grades = grades

new_grades["Mathe"] = 15

print(grades)
print(new_grades)
\end{minted}
\end{block}

\pause 
\vspace{12pt}

\begin{block}{Erklärung}
	\vspace{2pt}
Intern befindet sich in der Variable \pybw{grades} nur ein Verweis auf das Dictionary im Speicher. Die Variable \pybw{new_grades} enthält den gleichen Speicherverweis. Das heißt, wenn man das Dictionary \pybw{new_grades} verändert, verändert sich auch das Dictionary \pybw{grades}, weil es sich um ein-und dasselbe Dictionary handelt. 
\end{block}


\end{fragile}

\begin{frame}
	\begin{block}{Eine Kopie von einem Dictionary erstellen}
		\vspace{2pt}
		Mit der Funktion \py{dict()} kann man eine Kopie von einem Dictionary erstellen. 
		
		Beispiel: \py{dict(my_dict)} erstellt eine Kopie von \py{my_dict}.
	\end{block}
\end{frame}

\begin{fragile}
\begin{block}{Schleife über Dictionary I}
\vspace{2pt}
Ähnlich wie bei Listen kann man Schleifen auch über ein Dictionary laufen lassen.  
\end{block}
\vspace{12pt}
\pause 

\begin{exampleblock}{Beispiel}
\vspace{2pt}
\begin{overprint}
\onslide<2|handout:0>
\begin{minted}{python}
grades = { "Mathe": 8, "Bio": 11, "Sport": 13}

for item in grades:
  print(item)
\end{minted}
\onslide<3|handout:1>
\begin{minted}{python}
grades = { "Mathe": 8, "Bio": 11, "Sport": 13}

for item in grades:
  print(item)

# Mathe
# Bio
# Sport
\end{minted}
\end{overprint}
\end{exampleblock}
\end{fragile}

\begin{fragile}
\begin{block}{Schleife über Dictionary II}
\vspace{2pt}
Möchte man in der Schleife nicht nur die Schlüssel, sondern auch die Werte des Dictionaries zur Verfügung haben, so muss man die Methode \py{.items()} auf das Dictionary anwenden.   
\end{block}
\vspace{12pt}
\pause 


\begin{exampleblock}{Beispiel}
\vspace{2pt}
\begin{overprint}
\onslide<2|handout:0>
\begin{minted}{python}
grades = { "Mathe": 8, "Bio": 11, "Sport": 13}

for key, value in grades.items():
  print(f"{key}:{value}")
\end{minted}
\onslide<3|handout:1>
\begin{minted}{python}
grades = { "Mathe": 8, "Bio": 11, "Sport": 13}

for key, value in grades.items():
  print(f"{key}:{value}")

# Mathe:12
# Bio:11
# Sport:13
\end{minted}
\end{overprint}
\end{exampleblock}
\end{fragile}

\begin{frame}{Übungen}

\begin{block}{Zwei Dictionaries kombinieren}
	\vspace{2pt}
Gegeben seien zwei Dictionaries, z.B.  

\py{majors = {"Deutsch": 11, "Mathe": 7}}

und 

\py{minors = {"Sport": 14, "Bio": 8 }}

Füge die Einträge des zweiten Dictionaries zum ersten Dictionary hinzu. 
\end{block}

\pause 

\vspace{12pt}

\begin{block}{Ein Dictionary \enquote{filtern}}
\vspace{2pt}
Sei ein beliebiges Dictionary mit Noten gegeben. Entferne alle Einträge, deren Note schlechter als 5 Punkte ist. 
\end{block}
\end{frame}


\begin{frame}<beamer:0>[fragile]{Lösung}

\begin{solutionblock}{Zwei Dictionaries kombinieren}
\begin{minted}{python}
majors = {"Deutsch": 11, "Mathe": 7}
minors = {"Sport": 14, "Bio": 8}

for key,value in minors.items():
  majors[key] = value
print(majors)
\end{minted}
\end{solutionblock}

\vspace{12pt}

\begin{solutionblock}{Ein Dictionary \enquote{filtern}}
\begin{minted}{python}
grades = {"Deutsch": 11, "Mathe": 3, "Sport": 14, "Geschichte": 1}
#  Man darf die Länge eines Dictionaries in einer Schleife nicht verändern, deshalb machen wir eine Kopie
result = dict(grades)
for key, value in grades.items():
  if value < 5:
    del result[key]
print(result)
\end{minted}
\end{solutionblock}

\end{frame}


\begin{fragile}

\begin{block}{Ein Dictionary zerlegen}
\vspace{2pt}
Mit der Methode \py{.keys()} erhält man eine Liste aller Schlüssel eines Dictionaries. \pause

Mit der Methode \py{.values()} erhält man eine Liste aller Werte eines Dictionaries. \pause 

In beiden Fällen, muss das Ergebnis mittels der Funktion \py{list()} in eine Liste umgewandelt werden. 
\end{block}


\vspace{12pt}
\pause 

\begin{exampleblock}{Beispiel}
\vspace{2pt}
\begin{overprint}
\onslide<4|handout:0>
\begin{minted}{python}
grades = { "Mathe": 8, "Bio": 11, "Sport": 13}

grade_subjects = grades.keys()
grade_subjects = list(grade_subjects)

grade_numbers = grades.values()
grade_numbers = list(grade_numbers)

print(grade_subjects)
print(grade_numbers)
\end{minted}
\onslide<5|handout:1>
\begin{minted}{python}
grades = { "Mathe": 8, "Bio": 11, "Sport": 13}

grade_subjects = grades.keys()
grade_subjects = list(grade_subjects)

grade_numbers = grades.values()
grade_numbers = list(grade_numbers)

print(grade_subjects)  # ["Mathe", "Bio", "Sport"]
print(grade_numbers)   # [8, 11, 13]
\end{minted}
\end{overprint}
\end{exampleblock}
\end{fragile}

\section{Funktionen \\ \footnotesize Wie man Code wiederverwerten kann}

\begin{frame}
\begin{block}{Problemstellung}
	\vspace{2pt}
	Es sei eine Liste mit Noten gegeben: 
	
	\py{grades = [7, 12, 8, 10, 2, 0, 3, 5, 6]}
	
	\pause
	Es sollen zunächst folgende Durchschnittsnoten berechnet werden: 
	\begin{itemize}
		\item Durchschnitt aller Noten
		\item Der Durchschnitt der ersten drei Noten
		\item Der Durchschnitt jeder zweiten Note
	\end{itemize}
\pause 
	Statt durch eine Zahl, soll das Ergebnis jedoch mit den Worten  
	\begin{itemize}
		\item \py{"Passt"} für Durchschnitte $\geq$ 5 Punkte
		\item \py{"Durchgefallen"} für Durschnitte $<$ 5 Punkte 
	\end{itemize}

	abgespeichert werden. 
	
	
	\pause 
	
	\vspace{8pt}
	
	Wie macht man das \emph{elegant}? 
\end{block}
\end{frame}

\begin{fragile}{}
\begin{block}{Lösung \onslide<8->{\footnotesize (Hauptsache es funktioniert)}}
\vspace{2pt}
\begin{overprint}
\onslide<2|handout:0>
\begin{minted}{python}
average = sum(grades) / len(grades)
   
   














#   
\end{minted}
\onslide<3|handout:0>
\begin{minted}{python}
average = sum(grades) / len(grades)
if average >= 5: 
  average = "Passt"
else: 
  average = "Durchgefallen"
  
  
  
  
  
  
  
  
  
  
  
  
#  
\end{minted}
\onslide<4|handout:0>
\begin{minted}{python}
average = sum(grades) / len(grades)
if average >= 5: 
  average = "Passt"
else: 
  average = "Durchgefallen"

average_2 = sum(grades[:3]) / len(grades[:3])










#
\end{minted}
\onslide<5|handout:0>
\begin{minted}{python}
average = sum(grades) / len(grades)
if average >= 5: 
  average = "Passt"
else: 
  average = "Durchgefallen"

average_2 = sum(grades[:3]) / len(grades[:3])
if average_2 >= 5:
  average_2 = "Passt"
else: 
  average_3 = "Durchgefallen"






#
\end{minted}
\onslide<6|handout:0>
\begin{minted}{python}
average = sum(grades) / len(grades)
if average >= 5: 
  average = "Passt"
else: 
  average = "Durchgefallen"

average_2 = sum(grades[:3]) / len(grades[:3])
if average_2 >= 5:
  average_2 = "Passt"
else: 
  average_3 = "Durchgefallen"

average_3 = sum(grades[::2]) / len(grades[::2])




#
\end{minted}
\onslide<7-|handout:1>
\begin{minted}{python}
average = sum(grades) / len(grades)
if average >= 5: 
  average = "Passt"
else: 
  average = "Durchgefallen"

average_2 = sum(grades[:3]) / len(grades[:3])
if average_2 >= 5:
  average_2 = "Passt"
else: 
  average_3 = "Durchgefallen"

average_3 = sum(grades[::2]) / len(grades[::2])
if average_3 < 5:
  average_3 = "Durchgefallen"
else: 
  average_3 = "Passt"
#
\end{minted}
\end{overprint}
\end{block}
\end{fragile}

\begin{frame}
\begin{block}{Nachteile dieser Lösung}
	\pause 
	\begin{itemize}[<+->]
	\item Viel Schreibarbeit, viel Wiederholung
	\item Der Code ist schwierig zu lesen. Man sieht vor lauter Wiederholungen nicht, was passiert. 
	\item Jedes Mal, wenn man diese \enquote{Berechnungslogik} verwendet, könnte man einen (Tipp-)Fehler machen.  
	\item Wenn man das Anforderungsprofil minimal ändert, muss diese \enquote{Logik} bei \emph{jedem} Auftreten im Code geändert werden
		(z.B. statt \py{"Passt"} soll das Ergebnis \py{"Bestanden"} heißen). In echten Projekten, kann das schnell ein paar Hundert Male sein. 	
	\end{itemize}
\end{block}
\end{frame}

\begin{fragile}
\begin{block}{Bessere Lösung}
\begin{minted}{python}
def compute_average(grade_list):
  result = sum(grade_list) / len(grade_list)
  if result >= 5:
    result = "Passt"
  else: 
    result = "Durchgefallen"
  return result

average = compute_average(grades)
average_2 = compute_average(grades[:3])
average_3 = compute_average(grades[::2])
\end{minted}
\end{block}

\end{fragile}

\begin{frame}
\metroset{block=fill}
\begin{block}{Definition: Funktion}
\vspace{2pt}
Eine Funktion ist ein Codeblock, der nur ausgeführt wird, wenn die Funktion \emph{aufgerufen} wird. 
Man kann der Funktion Werte als \emph{Parameter} übergeben. 
Sie kann auch einen Wert als Ergebnis \emph{zurückgeben}. 
\end{block}

\pause 

\vspace{12pt}
Man kann sich eine Funktion wie eine Maschine vorstellen, wo man oben Dinge (=Parameter) hineinfüllt und unten ein Ergebnis (=Rückgabewert) herausbekommt. 
Unabhängig von dem Eingabe-Ausgabe-Prinzip, kann solch eine Maschine auch Nebeneffekte (z.B. Krach) produzieren. 

\pause 

\vspace{12pt}
Man unterscheidet zwischen \emph{Definition} und \emph{Ausführung} einer Funktion. 
\end{frame}



\begin{frame}
\metroset{block=fill}


\renewcommand{\baselinestretch}{1.5}
\begin{block}{Struktur der Funktions-Definition}	
\vspace{2pt}

\pause 

\texttt{def} \pause \textit{Funktionsname}\pause\texttt{(}\pause\textit{Parameter\_0}\pause, \textit{Parameter\_1}, \dots, \textit{Parameter\_n}\pause\texttt{)}\pause\texttt{:}\\
\pause \spacechar \spacechar \textit{Codezeile1} \\
\pause \spacechar \spacechar \textit{Codezeile2} \\
\pause \phantom{Code} \vdots \\
\pause \spacechar \spacechar \texttt{return} \textit{Ergebnis}
\end{block}
\renewcommand{\baselinestretch}{1}
\vspace{12pt}

\pause 

\begin{block}{Struktur eines Funktionsaufrufs}	
\vspace{2pt}
result = \textit{Funktionsname}\texttt{(}\textit{Argument\_0}, \textit{Argument\_1}, \dots, \textit{Argument\_n}\texttt{)}
\end{block}

\end{frame}


\begin{frame}

\begin{block}{Good to know}
\pause 
\begin{itemize}[<+->]
	\item Eine Funktion muss schon \emph{vor} dem ersten Aufruf definiert worden sein (das ist nicht in allen Sprachen so). 
	\item Die Eingabwerte nennt man in der Funktionsdefintion \emph{Parameter}, beim Aufruf der Funktion nennt man sie jedoch \emph{Argumente}.
	\item Nicht jede Funktion braucht Eingangsdaten. Die Liste von Parametern einer Funktion kann daher leer sein.   
	\item Beim Aufruf spielt die Reihenfolge der angegebenen Argumente eine entscheidene Rolle. Sie werden entsprechend der Reihenfolge den Parametern in der Definition zugeordnet. 
%	\item Den Rückgabewert der Funktion erhält man durch den Zuweisungsoperator (\py{=}).
	\item Eine Funktion muss nicht unbedingt etwas zurückgeben, d.h. das \py{return}-Statement ist optional.
	\item Das \py{return}-Statement muss nicht unbedingt am Schluss der Funktion stehen. Jedoch wird Code, der nach dem \py{return}-Statement kommt, nicht mehr ausgeführt. 
\end{itemize}
\end{block}
\end{frame}

\begin{frame}{Übungen}

\begin{block}{Funktion ohne Parameter}
	\vspace{2pt}
Schreibe eine Funktion, die Deinen Namen auf der Konsole ausgibt. 
\end{block}
\vspace{12pt}
\begin{block}{Funktion mit einem Parameter}
\vspace{2pt}
Schreibe eine Funktion, die die übergebene Zahl verdoppelt. 
\end{block}
\vspace{12pt}
\begin{block}{Funktion mit zwei Parametern}
\vspace{2pt}
Schreibe eine Funktion, die die beiden übergebenen Zahlen multipliziert. 
\end{block}
\vspace{12pt}
\begin{block}{Funktion ohne Rückgabewert}
\vspace{2pt}
Was gibt eine Funktion zurück, die kein \py{return}-Statement enthält?
\end{block}
\end{frame}

\begin{frame}<beamer:0>[fragile]{Lösungen}

\begin{solutionblock}{Funktion ohne Parameter}
\begin{minted}{python}
def my_name(): 
  print("Aaron Kunert")
\end{minted}
\end{solutionblock}

\vspace{12pt}

\begin{solutionblock}{Funktion mit einem Parameter}
\begin{minted}{python}
def double(number): 
  return number * 2
\end{minted}
\end{solutionblock}

\vspace{12pt}

\begin{solutionblock}{Funktion mit zwei Parametern}
\begin{minted}{python}
def multiply(number1,number2): 
  return number1 * number2
\end{minted}
\end{solutionblock}


\end{frame}

\begin{frame}{Übung}
\begin{block}{Aggregatzustand von Wasser}
\vspace{2pt}
Schreibe eine Funktion, die entsprechend der übergebenen Temperatur den Aggregatzustand von Wasser (\py{"fest"},\py{"flüssig"},\py{"gasförmig"}) als String zurückgibt. 

Schaffst Du es ohne die Schlüsselwörter \py{elif} und \py{else}? 
\end{block}
\end{frame}

\begin{frame}<beamer:0>[fragile]{Lösung}

\begin{solutionblock}{Aggregatzustand von Wasser}
\begin{minted}{python}
def get_state(temp):
  if temp < 0:
    return "fest"
  if temp > 100:
    return "gasförmig"
  return "flüssig"
\end{minted}
\end{solutionblock}

\end{frame}


\begin{fragile}[Komplexere Übung]

\begin{block}{Gewichtete Durschnittsnote}
	\vspace{2pt}
Schreibe eine Funktion, die eine Liste der folgenden Struktur erwartet: 

\begin{minted}{python}
grades = [ 
  { 
    "subject": "Deutsch", 
    "grade": 14,
    "is_major": True
  },
  # ... 
  {
    "subject": "Sport",
    "grade": 11, 
    "is_major": False
  }
]
\end{minted}

Berechne die Durchschnittsnote, wobei Hauptfächer doppelt gewichtet werden sollen. 
\end{block}

\end{fragile}

\begin{frame}<beamer:0>[fragile]{Lösung}

\begin{solutionblock}{Gewichtete Durchschnittsnote}
\begin{minted}{python}
def weighted_average(grades): 
  weighted_sum = 0
  weighted_length = 0
  for grade in grades: 
    if grade["is_major"]: 
      weighted_sum += 2 * grade["grade"]
      weighted_length += 2
    else:
      weighted_sum += grade["grade"]
      weighted_length += 1
  result = weighted_sum/weighted_length
  return result 
\end{minted}
\end{solutionblock}

\end{frame}


\begin{fragile}

\begin{block}{Optionale Parameter}
	
	\pause 
	
\vspace{2pt}
Manchmal wirst Du bei Funktionen bemerken, dass einige der Parameter fast immer den gleichen Wert haben. In diesem Fall, möchtest Du diese Parameter nicht bei jedem Aufruf immer hinschreiben, sondern nur dort, wo er vom Standardfall abweicht. Dies ist möglich, wenn man den Standardwert (\emph{default value}) bei der Definition mit angibt. 

\pause 

\textbf{Wichtig:} Bei der Definition müssen die optionalen Parameter immer hinter den Pflichtparametern stehen.
\end{block}

\vspace{12pt}
\pause 

\begin{exampleblock}{Beispiel}
\begin{minted}{python}
def double(number, factor=2): 
  return number * factor
\end{minted} 


\pause 

Diese Funktion ist sehr vielseitig: Im einfachen Fall verdoppelt sie die eingegebene Zahl. Optional lässt sich der Faktor aber beliebig verändern. 
\end{exampleblock}

\end{fragile}


\begin{frame}
\begin{block}{Typischer Einsatzbereich}
\vspace{2pt}
Oftmals merkt man im Verlauf eines Projektes, dass eine gegebene Funktion nicht flexibel genug ist, dann kann man sie um einen optionalen Parameter erweitern, ohne den bisherigen Code verändern zu müssen. 
\end{block}

\vspace{12pt}

\pause 

\begin{exampleblock}{Fiktives Beispiel}
\vspace{2pt}
Stell Dir vor, Du baust einen Rechner für Deine Endnote. Hauptfachnoten werden immer doppelt gewichtet, daher verwendest Du die Funktion \py{weighted_average}, wie in der Übung. Plötzlich kommt raus, dass in der Abschlussprüfung, Hauptfächer vierfach gewichtet werden. Also erweiterst Du die Funktion, so dass der Gewichtungsfaktor anpassbar ist.

\pause

Jedoch möchtest Du den bisherigen Code nicht verändern. Daher definierst Du den Gewichtungsfaktor als optionalen Parameter, so dass die Funktion \enquote{abwärtskompatibel} zu ihrer bisherigen Verwendung ist.

\pause 
Die Definition startet dann mit \py{def weighted_average(grades, weight=2):} 
 
\end{exampleblock}

\end{frame}

\begin{frame}{Übung}

\begin{block}{Flexibler Durchschnittsrechner}
\vspace{2pt}
Erweitere die Funktion zur Berechnung von gewichteten Durchschnittsnoten so, dass optional der Gewichtungsfaktor angegeben werden kann. 	
\end{block}

\end{frame}

\begin{frame}<beamer:0>[fragile]{Lösung}

\begin{solutionblock}{Flexibler Durchschnittsrechner}
\begin{minted}{python}
def weighted_average(grades, weight=2): 
  weighted_sum = 0
  weighted_length = 0
  for grade in grades: 
    if grade["is_major"]: 
      weighted_sum += weight * grade["grade"]
      weighted_length += weight
    else:
      weighted_sum += grade["grade"]
      weighted_length += 1
  result = weighted_sum/weighted_length
  return result 
\end{minted}
\end{solutionblock}

\end{frame}

\begin{fragile}
	
\begin{block}{Named Parameters}
\vspace{2pt}
Hat eine Funktion viele Parameter, von denen etliche optional sind, so kann man einen Parameter statt über die Reihenfolge auch über den Namen übergeben. 
\end{block}

\pause 
\vspace{12pt}

\begin{exampleblock}{Beispiel}
\vspace{2pt}

\begin{minted}{python}
def my_function(parameter1, parameter2=0, parameter3="x", parameter4=-17):
  # ... 
\end{minted}
Möchte man jetzt die Funktion mit einem eigenen Wert \pybw{parameter1} und \pybw{parameter4} aufrufen aber alles andere auf Standard lassen, so geht das wie folgt: 

\py{my_function(15, parameter4=-20)}
\end{exampleblock}

	
\end{fragile}

\section{Scope \\ \footnotesize Wo Variablen gültig sind}


\begin{frame}
\begin{block}{Problemstellung}
\vspace{2pt}
Sei \py{my_variable} eine Variable mit Wert 1. 
Schreibe eine Funktion, die bei Aufruf die Variable \py{my_variable} um 1 erhöht. 

\vspace{8pt}

Wie macht man das? 
\end{block}
\end{frame}

\begin{fragile}
	
\begin{block}{Das Problem}
\vspace{2pt}

\begin{minted}{python}
my_variable = 1

def increment(): 
  my_variable = my_variable + 1

increment()
print(my_variable)
\end{minted}

\pause 
Die offensichtliche Lösung 
funktioniert nicht. Warum nicht? 
\end{block}
\end{fragile}


\begin{fragile}
	
\begin{block}{Experiment I}
\vspace{2pt}

\begin{minted}{python}
outer_variable = 1

def my_function(): 
  inner_variable = 5

my_function()
print(outer_variable)
print(inner_variable)
\end{minted}


\vspace{12pt}

\end{block}

\begin{exampleblock}{Beobachtung}

\pause 

Eine Variable, die innerhalb einer Funktion definiert wurde, ist auch nur innerhalb der Funktion sichtbar. 
\end{exampleblock}


\end{fragile}

\begin{fragile}
	
\begin{block}{Experiment II}
\vspace{2pt}

\begin{minted}{python}
outer_variable = 1

def my_function(): 
  print(outer_variable)
 

my_function()
print(outer_variable)
\end{minted}


\vspace{12pt}

\end{block}

\begin{exampleblock}{Beobachtung}

\pause 

Eine \emph{globale} Variable ist auch innerhalb einer Funktion definiert.  
\end{exampleblock}

	
\end{fragile}

\begin{fragile}

\begin{block}{Experiment III}
\vspace{2pt}

\begin{minted}{python}
outer_variable = 1

def my_function(): 
 outer_variable = 5
 print(outer_variable)


my_function()
print(outer_variable)
\end{minted}
\vspace{12pt}

\end{block}

\begin{exampleblock}{Beobachtung}

\pause 

Eine Variable innerhalb einer Funktion kann den gleichen Namen wie eine Variable außerhalb haben, allerdings ist die innere Variable nur innerhalb der Funktion sichtbar. 
\end{exampleblock}


\end{fragile}


\begin{fragile}
	
\begin{block}{Experiment IV}
\vspace{2pt}

\begin{minted}{python}
outer_variable = 1

def my_function(): 
  print(outer_variable)
  outer_variable = 5


my_function()
print(outer_variable)
\end{minted}
\vspace{12pt}

\end{block}

\begin{exampleblock}{Beobachtung/Erklärung}

\pause 

Python entscheidet anhand des Kontexts ob \py{outer_variable} eine globale Variable ist, oder eine lokale Variable, die zufällig den gleichen Namen wie eine globale Variable trägt. 

\pause 

Falls Python denkt, dass es sich um eine globale Variable handelt, so kann diese nur gelesen, nicht aber geschrieben (d.h. neu definiert) werden. 

\end{exampleblock}

	
\end{fragile}


\begin{fragile}
	
\begin{block}{Das Eingangsbeispiel}
\vspace{2pt}

\begin{minted}{python}
my_variable = 1

def increment(): 
  my_variable = my_variable + 1

increment()
print(my_variable)
\end{minted}

\vspace{12pt}

\end{block}

\begin{exampleblock}{Erklärung}

\pause 

Da \py{my_variable} rechts vom Gleichheitszeichen steht, denkt Python, dass es sich um die globale Variable \py{my_variable} handelt. Da \py{my_variable} aber auch links vom Gleichheitszeichen steht, wird auch schreibend auf die Variable zugegriffen. Das ist nicht erlaubt. 

\end{exampleblock}
	
\end{fragile}


\begin{fragile}

\begin{block}{Mögliche Lösung}
	\vspace{2pt}
\begin{minted}{python}
my_variable = 1

def increment(var): 
  return var + 1

my_variable = increment(my_variable)
print(my_variable)
\end{minted}

\vspace{12pt}

\end{block}
	
\end{fragile}

\begin{frame}
\metroset{block=fill}
	
\begin{block}{Definition}
\vspace{2pt}
Der Gültigkeitsbereich einer Variable wird \emph{Scope} genannt. 
\end{block}
\vspace{12pt}
\pause 

\metroset{block=transparent}
\begin{block}{Scope in Python}
\vspace{2pt}
In Python unterscheidet man zwischen \emph{global Scope} und \emph{local Scope}. Im local Scope hat man nur Lesezugriff auf den global Scope. 	
\end{block}
	
\end{frame}

\begin{frame}
	
\begin{block}{Warum ist das so eingeschränkt?}
	\pause 
	\begin{itemize}[<+->]
		\item Funktionen sollen möglichst wenige Nebeneffekte haben. Wenn eine Funktion den global Scope verändern kann, ist dies ein großer Nebeneffekt. 
		\item Wenn man eine Funktion schreibt, muss man sich keine Gedanken machen, ob ein Variablenname schon vergeben ist. 
		\item Wenn man sich innerhalb einer Funktion den Kontakt zum global Scope reduziert, so ist die Funktion besser zu verstehen, zu warten und zu testen. 
		\item \dots
	\end{itemize}
\end{block}
	
\end{frame}



%\section{Input/Output II \\ \footnotesize Dateien lesen/schreiben}



