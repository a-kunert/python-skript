\section{\texttt{break}, \texttt{continue} und \texttt{else} \\ \footnotesize Den Fluss einer Schleife kontrollieren}


\begin{fragile}
	
\metroset{block=fill}

\begin{block}{Das \texttt{break}-Statement}
Taucht innerhalb einer Schleife das Schlüsselwort \py{break} auf, so wird die weitere Abarbeitung der Schleife abgebrochen. Die Ausführung wird mit dem Code \emph{nach} dem Schleifenblock ausgeführt. 		
\end{block}

\metroset{block=transparent}

\vspace{12pt} \pause 


\begin{exampleblock}{Beispiel}
\vspace{2pt}

\begin{overprint}
	\onslide<2|handout:0>
\begin{minted}{python}
for k in range(1,100):
  print(k)
  if k > 3:
    break
\end{minted}
\onslide<3|handout:1>
\begin{minted}{python}
for k in range(1,100):
  print(k)
  if k > 3:
    break
# prints 1 2 3 4 
\end{minted}
\end{overprint}

\end{exampleblock}

	
\end{fragile}



\begin{fragile}
	
\metroset{block=fill}

\begin{block}{Das \texttt{continue}-Statement}
Taucht innerhalb einer Schleife das Schlüsselwort \py{continue} auf, so wird der aktuelle Schleifendurchgang abgebrochen. Die Ausführung wird mit der nächsten Schleifeniteration fortgesetzt. 
\end{block}

\metroset{block=transparent}

\vspace{12pt} \pause 


\begin{exampleblock}{Beispiel}
\vspace{2pt}

\begin{overprint}
\onslide<2|handout:0>
\begin{minted}{python}
for k in range(1,11):
  if k % 2 == 0:
    continue
  print(k)
\end{minted}
\onslide<3|handout:1>
\begin{minted}{python}
for k in range(1,11):
  if k % 2 == 0:
    continue
  print(k)
# prints 1 3 5 7 9 
\end{minted}
\end{overprint}
\end{exampleblock}
	
	
\end{fragile}


\begin{fragile}
	
\metroset{block=fill}

\begin{block}{Der \texttt{else}-Block einer Schleife}
Analog zum \py{if}-Statement, kann auch eine Schleife einen \py{else}-Block haben. Dieser wird ausgeführt, wenn die Schleife \emph{regulär} (also nicht durch die Verwendung von \py{break}) beendet wird.  
\end{block}

\metroset{block=transparent}

\vspace{12pt} \pause 


\begin{exampleblock}{Beispiel}
\begin{minted}{python}
name = input("Your name: ")

for letter in name: 
  if letter == "a":
    print("Your name contains an a")
    break
else: 
  print("Your name contains no a")
\end{minted}
\end{exampleblock}
	
	
\end{fragile}



\begin{fragile}[Übungen]

\begin{block}{Zählen bis zur nächsten 10er-Zahl}
	\vspace{2pt}
Lies eine Zahl \pybw{x} ein und gib auf der Konsole die Zahlen von \pybw{x} bis zur nächsten 10er-Zahl aus. 
\\
Ist die Eingabe \pybw{x = 17}, so soll die Ausgabe wie folgt aussehen: 
\begin{verbatim}
17
18
19
20
\end{verbatim}
\end{block}
	
\vspace{12pt}
\pause 

\begin{block}{Zählen mit Lücken}
	\vspace{2pt}
	Schreibe ein Skript, dass die Zahlen von 1 bis 99 aufzählt, dabei allerdings die 10er-Zahlen weglässt. Versuche dabei, ein \pybw{continue}-Statement zu verwenden.
\end{block}
\end{fragile}

\begin{frame}<beamer:0>[fragile]{Lösungen}

\begin{solutionblock}{Zählen bis zur nächsten 10er-Zahl}
\begin{minted}{python}
x = input("Gib eine Zahl an: ")
x = int(x)

for k in range(x, x + 11):
  print(k)
  if k % 10 == 0:
    break
\end{minted}
\end{solutionblock}

\vspace{12pt}

\begin{solutionblock}{Zählen mit Lücken}
\begin{minted}{python}
for k in range(1, 100):
  if k % 10 == 0:
    continue
  print(k)
\end{minted}
\end{solutionblock}

\end{frame}



\begin{fragile}[Harte Übung]
\begin{block}{Primzahltest}
\vspace{2pt}
Lies eine ganze Zahl \py{x} ein und überprüfe, ob diese Zahl eine Primzahl ist. Die Ausgabe des Programms soll etwa wie folgt aussehen:  

\texttt{Die Zahl 28061983 ist eine Primzahl.}
\end{block}

\vspace{12pt}
\begin{solutionblock}{Lösung}
\begin{minted}{python}
x = input("Gib eine Zahl ein: ")
x = int(x)

for k in range(2, x):
  if x % k == 0:
    print(f"{x} ist keine Primzahl.")
    break
else:
  print(f"{x} ist eine Primzahl.")
\end{minted}
\end{solutionblock}

\end{fragile}





%\begin{frame}{Harte Übung}
%\begin{block}{Finde die nächste Primzahl}
%\vspace{2pt}
%Lies eine ganze Zahl \py{x} ein und finde die nächste Zahl größer \py{x}, die gleichzeitig eine Primzahl ist. 
%\end{block}
%\end{frame}


\section{Listen \\ \footnotesize Viele Variablen gleichzeitig speichern}


\begin{frame}
\begin{block}{Problemstellung}
\vspace{2pt}
Lies mit Hilfe einer Schleife nach und nach Schulnoten von Dir ein. 
Alle Noten sollen gespeichert werden. Danach sollst Du die Wahl haben, die soundsovielte Note anzeigen lassen zu können.   

\vspace{8pt}

Wie macht man das? 
\end{block}
\end{frame}

\begin{fragile}{}
\begin{block}{Lösung \footnotesize(fast)}
\begin{minted}{python}
# ...
# Um das Eingeben der Noten kümmern wir uns noch
grades = [12, 10, 7, 14, 13, 13, 6, 4, 15, 14] # Noten in Notenpunkten

index = input("Die wievielte Note möchtest Du nochmal anschauen?")
index = int(index)

print(f"Deine { index }. Note ist { grades[index] } Punkte")
\end{minted}
\end{block}
\end{fragile}


\begin{fragile}

\metroset{block=fill}
\begin{block}{Struktur einer \emph{Liste}}
\vspace{2pt}
\large
\texttt{my\_list = }\pause {\Large\texttt{[}}\pause 
\texttt{element\_0}\pause,
\pause 
\texttt{element\_1}, \pause 
 \dots   
, \texttt{element\_n}\pause \Large{\texttt{]}}
\end{block}

\pause 

Die Variable \py{my_list} trägt nicht nur einen Wert, sondern $n+1$ Werte. Ansonsten verhält sich \py{my_list} wie eine ganz \enquote{normale} Variable. 
Als Einträge einer Liste sind beliebige Werte mit beliebigen Datentypen zugelassen. 


\vspace{12pt}

\pause

\textbf{Frage:} Welchen Datentyp hat die Liste \py{[2, 2.3, "Hello"]} ? 
	
\end{fragile}

\begin{frame}
	
\begin{block}{Auf Listenelemente zugreifen}
	
\vspace{2pt}

Auf das \pybw{n}-te Element der Liste \py{my_list} kann man mittels \py{my_list[n]} zugreifen. 

\pause 

Mit \py{my_list[-1]}, \py{my_list[-2]}, etc. kann man auf das letzte, vorletzte, etc. Element 
der Liste zugreifen. 

\end{block}

\pause 
\vspace{12pt}

\begin{alertblock}{Achtung}
\vspace{2pt}
Python fängt bei 0 an zu zählen. D.h. das erste Element in der Liste hat den Index 0. \\
Beispiel: \py{my_list[1]} liefert das \textbf{2. Element} der Liste. 
\end{alertblock}

	
\end{frame}	


\begin{frame}
\begin{block}{Schreibzugriff auf Listenelemente}
\vspace{2pt}
Nach dem gleichen Prinzip lassen sich einzelne Listeneinträge verändern. \\
Beispiel: \py{my_list[3] = -23}. 
\end{block}

\pause 
\vspace{12pt}



\begin{alertblock}{Achtung}
\vspace{2pt}
Man kann nur schon existierende Listeneinträge verändern. 
\end{alertblock}

\pause 
\vspace{12pt}



\begin{exampleblock}{Neues Konzept}
\vspace{2pt}
Listen sind der erste Datentyp, den wir kennenlernen, der \emph{mutable} (veränderbar) ist. Die bisherigen Datentypen waren \emph{immutable}, d.h. man konnte sie zwar überschreiben, aber nicht verändern. 
\end{exampleblock}

\end{frame}

\begin{frame}
	
\begin{block}{Listeneinträge hinzufügen}
	\vspace{2pt}
	Mit der \emph{Methode} \pybw{.append()} kann ein Eintrag zur Liste hinzugefügt werden. \\ 
	Bsp: \py{my_list.append(12)} fügt einen weiteren Eintrag mit Wert \pybw{12} hinzu. 
\end{block}	

\pause 
\vspace{12pt}


\begin{block}{Listeneinträge entfernen}
	\pause 
\vspace{2pt}
Mit dem Keyword \pybw{del} kann man Einträge an einer bestimmten Position löschen. Dabei verschieben sich die darauffolgenden Einträge um \pybw{1} nach vorne. \\
Beispiel: \py{del my_list[2]} löscht das dritte Element.  

\pause 

Mit der Methode \pybw{.remove()} kann man Einträge mit einem bestimmten Wert löschen. \\
Beispiel: \py{my_list.remove(-23)} entfernt den ersten Eintrag mit dem Wert \pybw{-23}. Ist der Wert nicht vorhanden gibt es eine Fehlermeldung. 
\end{block}
\end{frame}

\begin{fragile}[Übung]
\begin{block}{Eine Liste erstellen}
\vspace{2pt}
Schreibe ein kleines Programm, dass Dich ca. 4x nach dem Namen einer Freund*in fragt und Dir am Schluss die Liste der eingegebenen Freund*innen ausgibt. 	
\end{block}
\vspace{12pt}
\begin{solutionblock}{Lösung}
\begin{minted}{python}
friends = []
for k in range(1, 5):
  friend = input("Wie heißt ein*e Freund*in von Dir? ")
  friends.append(friend)
 print(friends)
\end{minted}
\end{solutionblock}
\end{fragile}





\begin{fragile}[Übung]
\begin{block}{Das Eingangsproblem}
\vspace{2pt}
Schreibe ein kleines Programm, dass solange Deine Noten einliest, bis Du \textbf{q} drückst. Danach sollst Du die Möglichkeit haben, eine Zahl \pybw{k} einzugeben, so dass Dir die \pybw{k}-te Note angezeigt wird. 
\end{block}	
\end{fragile}

\begin{frame}<beamer:0>[fragile]{Lösung}
\begin{solutionblock}{Das Eingangsproblem}
\begin{minted}{python}
grades = []
while True:
  grade = input("Gib eine Note an: ")
  if grade == "q":
    break
  grades.append(int(grade))

index = input("Die wievielte Note möchtest Du nochmal anschauen?")
index = int(index)
print(f"Deine { index }. Note ist { grades[index-1] } Punkte")
\end{minted}
\end{solutionblock}
\end{frame}

\begin{fragile}
\begin{block}{Schleife über Liste}
\vspace{2pt}
Analog wie über Strings und Ranges kann man Schleifen auch über eine Liste laufen lassen.  
\end{block}
\vspace{12pt}
\pause 

\begin{exampleblock}{Beispiel}
\vspace{2pt}
\begin{overprint}
\onslide<2|handout:0>
\begin{minted}{python}
my_hobbies = ["Segeln", "Tennis", "Schwimmen", "Lesen"]

for hobby in my_hobbies:
  print(hobby)
\end{minted}
\onslide<3|handout:1>
\begin{minted}{python}
my_hobbies = ["Segeln", "Tennis", "Schwimmen", "Lesen"]

for hobby in my_hobbies:
  print(hobby)
  
# prints:
# Segeln
# Tennis
# Schwimmen
# Lesen
\end{minted}
\end{overprint}
\end{exampleblock}
\end{fragile}


\begin{fragile}
\begin{block}{Schleife über Liste mit Indizes}
\vspace{2pt}
Möchte man in einer Schleife nicht nur die Listeneinträge, sondern auch die Indizes verwenden, so muss man die Funktion \py{enumerate()} auf die Liste anwenden. 
\end{block}
\vspace{12pt}
\pause 

\begin{exampleblock}{Beispiel}
\vspace{2pt}
\begin{overprint}
\onslide<2|handout:0>
\begin{minted}{python}
my_hobbies = ["Segeln", "Tennis", "Schwimmen", "Lesen"]

for index, hobby in enumerate(my_hobbies):
  print(f"Mein {index + 1}.Hobby ist {hobby}")
\end{minted}
\onslide<3|handout:1>
\begin{minted}{python}
my_hobbies = ["Segeln", "Tennis", "Schwimmen", "Lesen"]

for index, hobby in enumerate(my_hobbies):
  print(f"Mein {index + 1}.Hobby ist {hobby}")
  
# prints:
# Mein 1. Hobby ist Segeln
# Mein 2. Hobby ist Tennis
# Mein 3. Hobby ist Schwimmen
# Mein 4. Hobby ist Lesen
\end{minted}
\end{overprint}
\end{exampleblock}
\end{fragile}


\begin{frame}{Übung}

\begin{block}{Liste durchsuchen}
	\vspace{2pt}
Lies wieder eine Liste Deiner Noten ein. Prüfe, ob Du mindestens einmal unterpunktet hast (d.h. 0 Punkte hattest). 
Auf der Konsole soll dann entweder 

\pybw{Du hast irgendwo unterpunktet} 

oder 

\pybw{Du hast nirgendwo unterpunktet} 

ausgegeben werden. 
\end{block}
\end{frame}


\begin{frame}<beamer:0>[fragile]{Lösung}
\begin{solutionblock}{Liste durchsuchen}
\begin{minted}{python}
grades = []
while True:
  grade = input("Gib eine Note an: ")
  if grade == "q":
    break
  grades.append(int(grade))

for grade in grades:
  if grade == 0:
    print("Du hast irgendwo unterpunktet.")
    break
else: 
  print("Du hast nirgendwo unterpunktet.")
\end{minted}
\end{solutionblock}
\end{frame}

\begin{fragile}

\begin{block}{Ist ein Element in einer Liste enthalten?}
\vspace{2pt}	
Möchte man prüfen, ob ein Element in einer Liste enthalten ist, so kann man auch das Schlüsselwort \py{in} verwenden. 
\end{block}

\pause 
\vspace{12pt}

\begin{exampleblock}{Beispiel}
\vspace{2pt}
\begin{overprint}
\onslide<2|handout:0>
\begin{minted}{python}
my_hobbies = ["Segeln", "Tennis", "Schwimmen", "Lesen"]

sailing_in_list = "Segeln" in my_hobbies
climbing_in_list = "Klettern" in my_hobbies

print(sailing_in_list)  
print(climbing_in_list)
\end{minted}
\onslide<3|handout:1>
\begin{minted}{python}
my_hobbies = ["Segeln", "Tennis", "Schwimmen", "Lesen"]

sailing_in_list = "Segeln" in my_hobbies
climbing_in_list = "Klettern" in my_hobbies

print(sailing_in_list)   #  True
print(climbing_in_list)  #  False
\end{minted}
\end{overprint}

\end{exampleblock}
\end{fragile}


\begin{fragile}
	
\begin{block}{Eine Liste sortieren}
\vspace{2pt}
Um eine Liste zu sortieren, verwende die Methode \pybw{.sort()}. Dies verändert die Liste dauerhaft. \\
\pause 
Um eine sortierte Kopie einer Liste zu erstellen, verwende die Funktion \pybw{sorted()}.  \\
\pause 
Mit Hilfe des Parameters \pybw{reverse=True} lässt sich eine Liste absteigend ordnen. 
\end{block}	

\pause \vspace{12pt}

\begin{exampleblock}{Beispiel für \texttt{sort}}
\vspace{2pt}
\begin{overprint}
\onslide<4|handout:0>
\begin{minted}{python}
my_list = [1, 5, 2, 7]
my_list.sort()
print(my_list)  
\end{minted}
\onslide<5-|handout:1>
\begin{minted}{python}
my_list = [1, 5, 2, 7]
my_list.sort()
print(my_list)  # [1, 2, 5, 7]
\end{minted}
\end{overprint}

\end{exampleblock}

\vspace{12pt}

\pause \pause 

\begin{exampleblock}{Beispiel für \texttt{sorted}}
\vspace{2pt}
\begin{overprint}
\onslide<6|handout:0>
\begin{minted}{python}
my_list = [1, 5, 2, 7]
sorted_list = sorted(my_list)
print(my_list)  
print(sorted_list)  
\end{minted}
\onslide<7|handout:1>
\begin{minted}{python}
my_list = [1, 5, 2, 7]
sorted_list = sorted(my_list)
print(my_list)  # [1, 5, 2, 7]
print(sorted_list)  # [1, 2, 5, 7]
\end{minted}
\end{overprint}
\end{exampleblock}
\end{fragile}

\begin{fragile}
\begin{exampleblock}{Beispiel für absteigende Sortierung}
\vspace{2pt}
\begin{overprint}
\onslide<1|handout:0>
\begin{minted}{python}
my_list = [1, 5, 2, 7]
my_list.sort(reverse=True)
print(my_list)  

my_list = [7, 12, 5, 18]
sorted_list = sorted(my_list, reverse=True)
print(sorted_list) 
\end{minted}
\onslide<2|handout:1>
\begin{minted}{python}
my_list = [1, 5, 2, 7]
my_list.sort(reverse=True)
print(my_list)  # [7, 5, 2, 1]

my_list = [7, 12, 5, 18]
sorted_list = sorted(my_list, reverse=True)
print(sorted_list) # [18, 12, 7, 5] 
\end{minted}
\end{overprint}
\end{exampleblock}
\end{fragile}


\begin{fragile}[Übung]
	
\begin{block}{Beste/Schlechteste Note}
\vspace{2pt}
Lies wieder ein paar Noten ein. Gib dann auf der Konsole einmal die beste und einmal die schlechteste Note aus. 
\end{block}	

\vspace{12pt}

\begin{solutionblock}{Lösung}
\begin{minted}{python}
#  ...
#  einlesen wie immer
grades.sort()
min_grade = grades[0]
max_grade = grades[-1]
print(f"Schlechteste Note: {min_grade}")
print(f"Beste Note: {max_grade}")
\end{minted}
\end{solutionblock}
\end{fragile}

\begin{fragile}
\begin{block}{Nützliche Funktionen/Methoden}
\vspace{2pt}	
Für Listen stellt Python viele nützliche Methoden bzw. Funktionen bereit. Wenn Du googlest, findest Du für viele \enquote{Alltagsfragen} eine Lösung. 

Zum Beispiel hier: \texttt{https://docs.python.org/3/tutorial/datastructures.html}
\end{block}

\pause
\vspace{12pt}


\begin{exampleblock}{Beispiele}
\begin{minted}{python}
my_list = [2, 4, 8, 1]

len(my_list)   # = 4  (Gibt die Anzahl der Elemente an)
sum(my_list)   # = 15 (Berechnet die Summe der Elemente)
my_list.reverse() # [1, 8, 4, 2] (Dreht die Reihenfolge um)
my_list.insert(2,-1) # [2, 4, -1, 8, 1] (fügt den Wert -1 an Position 2 ein)
my_list.pop() # 1 (Gibt den letzten Eintrag der Liste zurück und entfernt ihn aus der Liste)
\end{minted}
\end{exampleblock}
\end{fragile}


\begin{fragile}[Übung]

\begin{block}{Durchschnittsnote}
\vspace{2pt}
Lies wieder ein paar Noten ein. Gib auf der Konsole die Durchschnittsnote aus. 
\end{block}	

\vspace{12pt}

\begin{solutionblock}{Lösung}
	\begin{minted}{python}
	#  ...
	#  einlesen wie immer
	
	total_sum = sum(grades)
	count = len(grades)
	average = total_sum/count
	print(f"Die Durchschnittsnote ist {average}")
	\end{minted}
\end{solutionblock}
	
\end{fragile}

\begin{frame}
\begin{block}{Slicing}
\vspace{2pt}
Wenn man eine Liste hat, ist es oft nötig, einen Teil der Liste \enquote{auszuschneiden}.\\
\pause
Dafür hat Python die \emph{Slice-Notation} eingeführt. \\
\pause 
Diese funktioniert nach folgendem Schema: 

\pause  \py{my_list[start:stop:step]}. 

\pause 
Die Einträge (start, stop, step) sind dabei jeweils optional. Wie immer wird der obere Wert (\pybw{stop}) gerade nicht erreicht.  
\pause 


Slicing lässt sich übrigens auch nach dem gleichen Schema auch auf Strings anwenden. 
\end{block}	

\pause
\vspace{12pt}

\begin{alertblock}{Wichtig}
\vspace{2pt}
Wenn man Slicing anwendet, erhält man eine Kopie der ausgewählten Elemente zurück. Die ursprüngliche Liste wird \emph{nicht} verändert. 
\end{alertblock}
\end{frame}


\begin{fragile}
\begin{exampleblock}{Beispiele}
	\vspace{2pt}
\begin{overprint}
\onslide<1|handout:0>
\begin{minted}{python}
my_list = [2, 4, 6, 8, 10]

my_list[1:3]    
my_list[0:4]     
my_list[1:1]     
my_list[0:4:2]   
my_list[:3]      
my_list[2:]      
my_list[:]      
my_list[1:-2]    
my_list[-3:-1]  
my_list[::-1]    
\end{minted}

\onslide<2|handout:0>
\begin{minted}{python}
my_list = [2, 4, 6, 8, 10]

my_list[1:3]     # [4, 6]
my_list[0:4]     
my_list[1:1]     
my_list[0:4:2]   
my_list[:3]      
my_list[2:]     
my_list[:]       
my_list[1:-2]   
my_list[-3:-1]  
my_list[::-1]    
\end{minted}

\onslide<3|handout:0>
\begin{minted}{python}
my_list = [2, 4, 6, 8, 10]

my_list[1:3]     # [4, 6]
my_list[0:4]     # [2, 4, 6, 8]
my_list[1:1]     
my_list[0:4:2]   
my_list[:3]      
my_list[2:]      
my_list[:]       
my_list[1:-2]    
my_list[-3:-1]  
my_list[::-1]    
\end{minted}

\onslide<4|handout:0>
\begin{minted}{python}
my_list = [2, 4, 6, 8, 10]

my_list[1:3]     # [4, 6]
my_list[0:4]     # [2, 4, 6, 8]
my_list[1:1]     # []
my_list[0:4:2]   
my_list[:3]      
my_list[2:]     
my_list[:]       
my_list[1:-2]    
my_list[-3:-1]   
my_list[::-1]    
\end{minted}


\onslide<5|handout:0>
\begin{minted}{python}
my_list = [2, 4, 6, 8, 10]

my_list[1:3]     # [4, 6]
my_list[0:4]     # [2, 4, 6, 8]
my_list[1:1]     # []
my_list[0:4:2]   
my_list[:3]      
my_list[2:]      
my_list[:]       
my_list[1:-2]    
my_list[-3:-1]   
my_list[::-1]    
\end{minted}

\onslide<6|handout:0>
\begin{minted}{python}
my_list = [2, 4, 6, 8, 10]

my_list[1:3]     # [4, 6]
my_list[0:4]     # [2, 4, 6, 8]
my_list[1:1]     # []
my_list[0:4:2]   # [2, 6]
my_list[:3]      
my_list[2:]      
my_list[:]       
my_list[1:-2]    
my_list[-3:-1]   
my_list[::-1]    
\end{minted}

\onslide<7|handout:0>
\begin{minted}{python}
my_list = [2, 4, 6, 8, 10]

my_list[1:3]     # [4, 6]
my_list[0:4]     # [2, 4, 6, 8]
my_list[1:1]     # []
my_list[0:4:2]   # [2, 6]
my_list[:3]      # [2, 4, 6]
my_list[2:]      
my_list[:]       
my_list[1:-2]   
my_list[-3:-1]   
my_list[::-1]    
\end{minted}

\onslide<8|handout:0>
\begin{minted}{python}
my_list = [2, 4, 6, 8, 10]

my_list[1:3]     # [4, 6]
my_list[0:4]     # [2, 4, 6, 8]
my_list[1:1]     # []
my_list[0:4:2]   # [2, 6]
my_list[:3]      # [2, 4, 6]
my_list[2:]      # [6, 8, 10]
my_list[:]       
my_list[1:-2]    
my_list[-3:-1]  
my_list[::-1]    
\end{minted}

\onslide<9|handout:0>
\begin{minted}{python}
my_list = [2, 4, 6, 8, 10]

my_list[1:3]     # [4, 6]
my_list[0:4]     # [2, 4, 6, 8]
my_list[1:1]     # []
my_list[0:4:2]   # [2, 6]
my_list[:3]      # [2, 4, 6]
my_list[2:]      # [6, 8, 10]
my_list[:]       # [2, 4, 6, 8, 10]
my_list[1:-2]    
my_list[-3:-1]   
my_list[::-1]    
\end{minted}

\onslide<10|handout:0>
\begin{minted}{python}
my_list = [2, 4, 6, 8, 10]

my_list[1:3]     # [4, 6]
my_list[0:4]     # [2, 4, 6, 8]
my_list[1:1]     # []
my_list[0:4:2]   # [2, 6]
my_list[:3]      # [2, 4, 6]
my_list[2:]      # [6, 8, 10]
my_list[:]       # [2, 4, 6, 8, 10]
my_list[1:-2]    # [6]
my_list[-3:-1]    
my_list[::-1]     
\end{minted}

\onslide<11|handout:0>
\begin{minted}{python}
my_list = [2, 4, 6, 8, 10]

my_list[1:3]     # [4, 6]
my_list[0:4]     # [2, 4, 6, 8]
my_list[1:1]     # []
my_list[0:4:2]   # [2, 6]
my_list[:3]      # [2, 4, 6]
my_list[2:]      # [6, 8, 10]
my_list[:]       # [2, 4, 6, 8, 10]
my_list[1:-2]    # [6]
my_list[-3:-1]   # [6, 8] 
my_list[::-1]     
\end{minted}

\onslide<12|handout:1>
\begin{minted}{python}
my_list = [2, 4, 6, 8, 10]

my_list[1:3]     # [4, 6]
my_list[0:4]     # [2, 4, 6, 8]
my_list[1:1]     # []
my_list[0:4:2]   # [2, 6]
my_list[:3]      # [2, 4, 6]
my_list[2:]      # [6, 8, 10]
my_list[:]       # [2, 4, 6, 8, 10]
my_list[1:-2]    # [6]
my_list[-3:-1]   # [6, 8] 
my_list[::-1]    # [10, 8, 6, 4, 2]  
\end{minted}

\end{overprint}
\end{exampleblock}
\end{fragile}







