\documentclass[algorithm,pgfplots,colortheme=dark]{cuzbeamer}
\usepackage[ngerman]{babel}
\usepackage[scale=2]{ccicons}
\usepackage{listings}

\newcommand{\py}[1]{\mintinline{python}{#1}}
\newcommand{\pybw}[1]{\mintinline[style=bw]{python}{#1}}
\newcommand{\bash}[1]{\mintinline{bash}{#1}}


\begin{document}
\title{Programmieren mit Python}
\subtitle{Eine Einführung}
\date{\today}
\author{Dr. Aaron Kunert}
\email{aaron.kunert@abiturma.de}
\maketitle

\section{Grundlegendes}

\begin{frame}
	\begin{block}{Ablauf des Kurses}
		\begin{itemize}
			\item Kommunikation über Slack 
			\item Jede Woche ein Aufgabenblatt $\rightarrow$ Besprechung in der nächsten Woche
			\item Mischung aus Vortrag, Live-Coding und Präsenzübungen
			\item Download und Kollaboration über Github
			\item Im Idealfall: Mehr Praxis statt Erklärungen
		\end{itemize}
	\end{block}
\end{frame}

\begin{frame}
	\begin{block}{Warum Python?}
		\begin{itemize}
			\item Einfaches Setup
			\item Einstiegsfreundliche Syntax
			\item Python ist eine Hochsprache
			\item Python muss nicht kompiliert, sondern nur interpretiert werden
			\item Großes \textit{ecosystem}
			\item Python ist extrem vielseitig
		\end{itemize}
	\end{block}
\end{frame}

\begin{frame}
	\begin{block}{Typische Einsatzbereiche}
		\begin{itemize}
		\item Webentwicklung
		\item Webscraping
		\item Datenanalyse
		\item Automatisierung
		\end{itemize}
	\end{block}
\end{frame}



\begin{frame}{Was wird benötigt}
\uncover<+->{
\metroset{block=fill}
\begin{block}{Am Anfang}
\begin{itemize}
	\item Compiler/Interpreter
	\item Texteditor
\end{itemize}
\end{block}
}
\uncover<+->{
\metroset{block=fill}	
\begin{block}{Später}
	\begin{itemize}
	\item Google
	\item Integrierte Entwicklungsumgebung (IDE)
	\item Versionskontrolle (VCS)
	\item Virtueller Maschinen
	\item Datenbanken
	\item Grafikbearbeitung
	\end{itemize}
\end{block}
}
\end{frame}


\begin{frame}{Installation}

\begin{block}{Ist Python schon installiert ?}
	\begin{itemize}
		\item Öffne ein Terminal/die Eingabeaufforderung
		\item Gib ein \bash{python --version}
		\item oder alternativ \bash{python3 --version}
		\item Erhältst Du die Antwort \bash{Python} und eine Zahl $\geq 3.6$, dann ist alles fein
		\item Falls nicht: Installiere Python!
	\end{itemize}
\end{block}
\end{frame}


\begin{frame}{Installation}
\begin{block}{Installation}
\begin{enumerate}
	\item Gehe auf https://www.python.org/downloads/
	\item Klicke den Button "Download Python 3.9.3."
	\item Führe die Installationsdatei aus
	\item Falls Du gefragt wirst, bestätige, dass Python zum PATH hinzugefügt wird
	\item Eventuell muss der Rechner neu gestartet werden
\end{enumerate}	
\end{block}

\begin{alertblock}{Achtung bei Windows}
\vspace{2pt}
Python muss zum PATH hinzugefügt werden. 

\includegraphics[width=0.65\textwidth]{python_path.jpg}
\end{alertblock}

\end{frame}

\begin{frame}
\begin{block}{Cross-Check}
\vspace{2pt}
Gib \bash{python} (Win) oder \bash{python3} (Mac) im \textit{Terminal} ein.
Du solltest etwa folgendes sehen:  
\vspace{12pt}

\texttt{Python 3.9.2 (tags/v3.9.2:1a79785, Feb 19 2021, 13:44:55) [MSC v.1928 64 bit (AMD64)] on win32
	Type "help", "copyright", "credits"{} or "license"{} for more information.}

\texttt{>{}>{}>}
\end{block}
\begin{block}{}
Jetzt bist Du im \textit{interactive mode} (REPL) von Python. Hier kannst Du einzelne Codezeilen eingeben und mittels \bash{Enter} ausführen. 
Um den interactive mode zu verlassen, gib \py{exit()} ein und bestätige mit der \bash{Enter}-Taste. 	
\end{block}
\end{frame}


%\begin{standout}
%	Erste Schritte im Interactive Mode\\
%\end{standout}
\section{Erste Schritte im REPL \\ \footnotesize REPL (Read-Evaluate-Print-Loop)}


\begin{frame}
\begin{block}{Probier mal folgende Kommandos aus}	
	\begin{itemize}
		\item \py{3 + 4}
		\item \py{2 - 7}
		\item \py{"Hello" + "Python"}
	\end{itemize}
\end{block}	
\end{frame}



\begin{frame}{Übung}
\uncover<+->{\begin{block}{Was machen die folgenden \textit{Operatoren}?}
	\begin{itemize}
		\item \pybw{+}
		\item \pybw{-}
		\item \pybw{*}
		\item \pybw{/}
		\item \pybw{**}
	\end{itemize}
\end{block}}
\uncover<+->{\begin{block}{Und diese?}
\begin{itemize}
		\item \%
		\item \pybw{//}
		\item \pybw{==}
		\item \pybw{<=}
		\item \pybw{<}
\end{itemize}
\end{block}}

\end{frame}

\begin{frame}{Übung}
	\begin{block}{Wie rechnet Python?}
		\begin{itemize}
			\item Wird Punkt-vor-Strich berücksichtigt?
			\item Kann man mit Klammern die Reihenfolge beeinflussen?
			\item Was ist der Unterschied zwischen \py{10/5} und \py{10//5} ?
			\item Was bedeutet das Kommando \py{_}? 
			\item Wie kann man Zwischenergebnisse in Variablen speichern?
		\end{itemize}
	\end{block}
\end{frame}

\section{Variablen}

\begin{frame}
\uncover<+->{\begin{block}{}
		Jeder Wert in Python kann in einer Variable gespeichert werden: 
		
		\py{my_variable = 3}
\end{block}}

\uncover<+->{\begin{block}{}
		Die Zuweisung darf auch das Ergebnis einer Berechnung sein: 
		
		\py{my_new_variable = 3 + 5}
\end{block}}
\uncover<+->{\begin{block}{}
		Die Zuweisung darf auch weitere Variablen enthalten: 
		
		\py{my_brand_new_variable = my_variable + my_new_variable }
\end{block}}

\uncover<+->{\begin{block}{}
	Man darf auch Kettenzuweisungen machen: 
	
	\py{a = b = c = 100 }
\end{block}}
\end{frame}


\begin{frame}
\uncover<+->{\begin{block}{Gültige Variablennamen}
\begin{itemize}[<+->]
	\item Erlaubt sind Buchstaben (nur ASCII), Ziffern und Unterstriche
	\item Der Name darf nicht mit einer Ziffer starten
	\item Beliebige Länge 
	\item Als \textit{regulärer Ausdruck}:  \mintinline{php}{[_a-zA-Z][_0-9a-zA-Z]*}
	\item Schlüsselwörter sind nicht erlaubt
\end{itemize} 
\end{block}}
\vspace{12pt}
\uncover<+->{\begin{block}{Liste der Schlüsselwörter}
	\texttt{
	\begin{columns}[T,onlytextwidth]
		\column{0.2\textwidth}
		False\\ 	await\\ 	else\\ 	import\\ 	pass\\ assert \\	del\\ 	
		\column{0.2\textwidth}
		None \\	break \\	except \\ 	in \\	raise \\ global \\	not \\	 
		\column{0.2\textwidth}
		True \\	class \\ 	finally \\ 	is \\	return \\ with \\ async 
		\column{0.2\textwidth}
		and \\	continue \\ 	for \\	lambda \\	try \\ 	elif  \\	if  \\
		\column{0.2\textwidth}
		as \\ 	def \\ 	from  \\	nonlocal \\	while \\ 	or \\ 	yield
	\end{columns}
}
\end{block}}
\end{frame}

\begin{frame}
\uncover<+->{\begin{exampleblock}{Style-Guide Variablennamen}
	\begin{itemize}
		\item Immer englisch
		\item Nur Kleinbuchstaben
		\item Möglichst aussdrucksstarke Namen verwenden
		\item Keine Angst vor langen Namen 
		\item Namen, die aus mehreren Worten bestehen, mit Unterstrich trennen (\textit{snake-case})
	\end{itemize}
	
	\uncover<+->{z.B. \py{students_in_this_room}, \py{number_of_unpaid_bills}}
\end{exampleblock}}

\end{frame}


\section{Datentypen}

\begin{frame}
	\begin{block}{}
		Jeder Wert in Python hat einen \textit{Datentyp}. Unter anderem gibt es folgende \textit{primitive} Typen in Python.
		\begin{itemize}
			\item \py{int}  Integer (ganze Zahlen)
			\item \py{float} Float (Dezimalzahlen)
			\item \py{bool} Boolean (Wahrheitswerte)
			\item \py{str}  String (Zeichenketten)
			\item \pybw{NoneType} (Typ des leeren Werts \py{None})
		\end{itemize}
	\end{block}
\end{frame}


\begin{frame}
	\metroset{block=fill}
	
	\uncover<+->{\begin{block}{Integer}
		Ganze Zahlen wie z.B. \py{1}, \py{-1}, \py{0}. Nicht aber 
		\py{2.0} oder \py{0.0}. 	
	\end{block}}
	\vspace{12pt}
	\uncover<+->{\begin{block}{Float}
		Fließkommazahlen, z.B. \py{3.1415925}. Achtung: Bei Float-Berechnungen können schnell Rundungsfehler auftreten: Was ergibt z.B. \mintinline{python}{1.2 - 1.0} ? 
	\end{block}}
	\vspace{12pt}
	\uncover<+->{\begin{block}{Boolean}
		Booleans sind eine Sonderform von \py{int} und können nur die Werte \py{True} (entspricht 1) und \py{False} (entspricht 0) annehmen. Sie entstehen in der Regel, wenn man Fragen im Programm stellt (z.B. \py{3 < 4} oder \py{1 == 2}).   	
	\end{block}}
\end{frame}



\begin{frame}
	\metroset{block=fill}
	\uncover<+->{\begin{block}{String}
		Strings müssen in (ein-, zwei- und dreifache) Anführungszeichen eingeschlossen werden. Die Ausdrücke \py{'hello'}, \py{"Hello"} und \py{"""Hello"""} sind (fast) äquivalent. 
		
		Strings können auch Steuerzeichen, wie Zeilenumbrüche enthalten: z.B. \py{"Hello\n"} 
	\end{block}}
	\vspace{12pt}

	\uncover<+->{\begin{block}{Mehrzeilige Strings}
		Ein \textit{Stringliteral} kann nur innerhalb einer Zeile definiert werden. Soll ein String mehrere Zeilen umfassen, müssen dreifache Anführungszeichen verwendet werden.  
	\end{block}}

	\end{frame}

	\begin{frame}
			\metroset{block=fill}
		\uncover<+->{\begin{block}{Steuerzeichen}
			Gewisse Kombinationen mit Backslash sind reservierte Steuerzeichen. So bezeichnet beispielsweise \py{\n} einen Zeilenumbruch und \py{\t} ein Tabulatorzeichen. \\
			Beispiel: \py{"This text\noccupies two lines"}
		\end{block}}
			\vspace{12pt}
		\uncover<+->{\begin{block}{Escaping}
			Möchte man ein Steuerzeichen nicht ausführen, sondern buchstäblich nehmen. Muss man sie mit einem Backslash \textit{escapen} bzw. maskieren. \\
			Beispiel: \py{"This text fits in\\n one line"}
		\end{block}}
		\vspace{12pt}
		\uncover<+->{\begin{block}{Raw-Strings}
				Möchte man alle Steuerzeichen eines Strings ignorieren, kann man ihn als \textit{Raw-String} definieren. \\
				Beispiel: \py{r"This \n String \t has no control characters"}
		\end{block}}
		
	\end{frame}

	
	\begin{frame}
		\metroset{block=fill}
		\uncover<+->{\begin{block}{Woher weiß ich, welchen Typ eine Variable hat?}
			\vspace{2pt}
			Mit der Funktion \py{type()} lässt sich der Typ bestimmen, z.B. \py{type(3.2)}.  	
		\end{block}}
		\vspace{12pt}
		\uncover<+->{\begin{block}{Wie lassen sich Typen umwandeln?}
			\vspace{2pt}
			\uncover<+->{\textbf{Implizit}\\
			Bei manchen Operationen nimmt Python automatisch eine Typumwandlung vor. z.B. \py{1+2.0} ergibt \py{3.0}	\\ \\
		}
		\uncover<+->{\textbf{Explizit}\\
			Die Funktionen \py{int()}, \py{float()}, \py{str()} und \py{bool()} führen jeweils eine Typumwandlung durch (sofern möglich). Beispiele: 
			\begin{itemize}
				\item \py{int(2.0)} ergibt \py{2} 
				\item \py{float(2)} ergibt \py{2.0} 
				\item \py{int("3")} ergibt \py{3}
			\end{itemize} 
		}
		\end{block}}
		
		
	\end{frame}
	
	
	\begin{frame}{Übung}
		
		\begin{block}{Versuche die Fragen erst ohne Python zu beantworten, überprüfe Deine Vermutung}
			\begin{itemize}
				\item Welchen Datentyp hat das Ergebnis von \py{3 - 1.0} ?
				\item Was ist das Ergebnis von \py{"2" + 1} ? 
				\item Was ist das Ergebnis von \py{"2"} + \py{"2"}? 
				\item Sind die beiden Werte \py{0} und \py{"0"} gleich? 
				\item Sind die beiden Werte \py{2} und \py{True} gleich? 
				\item Sind die beiden Werte \py{bool(2)} und \py{True} gleich? 
				\item Sind die beiden Werte \py{1} und \py{True} gleich? 
			\end{itemize}
		\end{block}
		
		
	\end{frame}

	\begin{frame}{Übung}
	
	\begin{block}{Erkläre mit Deinen eigenen Worten}
		\begin{itemize}
			\item Nach welcher Regel wandelt \py{int()} ein \py{float} in eine ganze Zahl um? 
			\item Nach welchen Regeln wandelt \py{bool()} Zahlen und Strings in einen Wahrheitswert um? 
		\end{itemize}
	\end{block}
	
	
\end{frame}




\section{Operatoren}

\begin{frame}
\begin{block}{Die wichtigsten Operatoren}
	\begin{itemize}
		\item \pybw{+} (Addition oder Zusammenkleben von Strings)
		\item \pybw{-} (Subtraktion)
		\item \pybw{*} (Multiplikation)
		\item \pybw{/} (Division, ergibt immer ein Wert vom Typ \pybw{float})
		\item \pybw{**} (Potenzierung)
			\item \% (\textit{modulo-Operator}: Rest bei ganzzahliger Division)
		\item \pybw{//} (Division und Abrunden, ergibt immer ein Wert vom Typ \pybw{int})
		\item \pybw{==} (Vergleichsoperator, ergibt immer ein Wert vom Typ \pybw{bool})
	\end{itemize}
\end{block}
\end{frame}

\begin{frame}
\begin{block}{Präzedenz (oben höchste, unten niedrigste)}
	\begin{enumerate}
		\item Klammern
		\item \pybw{**}
		\item \pybw{*}, \pybw{/}, \pybw{//}, \%
		\item \pybw{+},\pybw{-}
	\end{enumerate}	
Operatoren gleichen Rangs werden innerhalb eines Ausdrucks von links nach rechts abgearbeitet. 

\vspace{10pt}
\textbf{Ausnahmen:}\\
Potenzierung (\py{**}) und Zuweisung (\py{=}). 
\end{block}
\end{frame}

\begin{fragile}[]
	\begin{block}{Kombinierte Zuweisung}
		\vspace{2pt}
		Oft möchte man eine gegebene Variable neu zuweisen: 
		\begin{minted}{python}
		counter = 1
		counter = counter + 1 	# counter = 2
		\end{minted}
	Dies lässt sich auch kurz schreiben als 
		\begin{minted}{python}
	counter = 1
	counter += 1 	# counter = 2
	\end{minted}
	Analog sind die Operatoren \py{-=}, \py{*=}, \py{/=}, etc. definiert. 
	\end{block}
	
	
	\end{fragile}



\section{Input/Output \\ \footnotesize Teil 1}

\begin{fragile}[]

\begin{block}{Output}
	\vspace{2pt}
	Um einen String auf der Konsole auszugeben, verwende die Funktion \py{print()}. 
	Zum Beispiel \py{print("Hello there")}. 
	
	Es können auch Variablen eingesetzt werden: 
	\begin{minted}{python}
	message = "Hello there"
	print(message) # Hello there
	\end{minted}
	
	\end{block}
	
\end{fragile}

\begin{fragile}[]
	
\begin{block}{String Interpolation}
\vspace{2pt}
Um Variablenwerte innerhalb eines Strings auszugeben, verwenden wir die String-Interpolation-Syntax
\begin{minted}{python}
my_value = 5
print(f"The variable my_value has the value {my_value}")
# The variable my_value has the value 5
\end{minted}

Das geht auch als \textit{inline expression}: 
\begin{minted}{python}
print(f"The sum of 1 and 2 is {1+2}")
# The sum of 1 and 2 is 3
\end{minted}

\end{block}
	
\end{fragile}

\begin{fragile}
\begin{block}{Output}
	\vspace{2pt}
Um einen String vom User einzulesen, verwende die Funktion \py{input()}:
\begin{minted}{python}
age = input("How old are you?")
print(f"I am {age} years old")
\end{minted}
\end{block}

\begin{alertblock}{Achtung}
	\vspace{2pt}
Das Ergebnis von \py{input} hat stets den Datentyp \py{string} auch wenn Zahlen eingelesen werden. Gegebenenfalls muss das Ergebnis mittels \py{int()} oder \py{float()} in den gewünschten Typ umgewandelt werden. 	
\end{alertblock}

\end{fragile}

\section{Script Mode}
\begin{frame}
\begin{block}{Script Mode}
	\vspace{2pt}
	Sobald man mehrere zusammenhängende Zeilen hat, wird Pythons \textit{interactive mode} sehr unübersichtlich. Daher gibt es auch die Möglichkeit, alle Programmzeilen in eine Text-Datei zu schreiben und diese gebündelt auszuführen.   
\end{block}

\end{frame}
\begin{fragile}[]
	\begin{exampleblock}{Ein erstes Beispiel}
		\begin{minted}{python}
		name = input("What is your name?")
		age = input("What is your age?")
		print(f"Hello {name}, you are {age} years old") 
		\end{minted}
		
		Speichere diesen Code in der Datei \py{my_script.py} ab. 
		
		Führe danach in diesem Ordner das Kommando 
		\pybw{python my_script.py} aus. 
	\end{exampleblock}
\end{fragile}

\begin{frame}{Übung}
	Schreibe ein kurzes Skript, dass Dich nach Deinem Namen, Alter und Adresse fragt. Wenn es alles eingelesen hat, soll es diese Infos in folgender Form auf der Konsole ausgeben: 	
	
	\texttt{Hallo Max, schön dass Du da bist. Du bist 21 Jahre alt und wohnst in der Bismarckstraße 12 in Glücksstadt.}
	
\end{frame}


\begin{fragile}
	
	\begin{block}{Kommentare}
		\vspace{2pt}
		Alle Zeichen einer Zeile, die hinter einem \texttt{\#} (Hashtag) kommen, werden von Python ignoriert.
		So lassen sich Kommentare im Quellcode platzieren. 
	\end{block}
	\vspace{12pt}
	\begin{exampleblock}{Beispiel}
	\begin{minted}{python}
	print("This line will be printed")
	# print("This line won't") 
	\end{minted}
	\end{exampleblock}
	
\end{fragile}



\end{document}






