\documentclass[algorithm,pgfplots,colortheme=dark]{cuzbeamer}
\usepackage[ngerman]{babel}
\usepackage[scale=2]{ccicons}
\usepackage{listings}

\newcommand{\py}[1]{\mintinline{python}{#1}}
\newcommand{\pybw}[1]{\mintinline[style=bw]{python}{#1}}

\begin{document}
\title{Programmieren mit Python}
\subtitle{Eine Einführung}
\date{\today}
\author{Dr. Aaron Kunert}
\email{aaron.kunert@abiturma.de}
\maketitle

\section{Grundlegendes}

\begin{frame}
\begin{block}{Warum Python?}
	\begin{itemize}
	\item Einfaches Setup
	\item Einstiegsfreundliche Syntax
	\item Python ist eine Hochsprache
	\item Python muss nicht kompiliert, sondern nur interpretiert werden
	\item Großes \textit{ecosystem}
	\item Python ist extrem vielseitig
	\end{itemize}
\end{block}
\end{frame}

\begin{frame}

\begin{block}{Ablauf des Kurses}
	\begin{itemize}
	\item Kommunikation über Slack 
	\item Jede Woche ein Aufgabenblatt $\rightarrow$ Besprechung in der nächsten Woche
	\item Download und Kollaboration über Github
	\item Mischung aus Vortrag, Live-Coding und Präsenzübungen
	\end{itemize}
\end{block}
\end{frame}

\begin{frame}{Was wird benötigt}
\uncover<+->{
\metroset{block=fill}
\begin{block}{Am Anfang}
\begin{itemize}
	\item Texteditor
	\item Compiler/Interpreter
	\item Google
\end{itemize}
\end{block}
}
\uncover<+->{
\metroset{block=fill}	
\begin{block}{Später}
	\begin{itemize}
	\item Integrierte Entwicklungsumgebung (IDE)
	\item Versionskontrolle (VCS)
	\item Virtueller Maschinen
	\item Debugger
	\item Linter
	\end{itemize}
\end{block}
}
\end{frame}


\begin{frame}{Installation}

\begin{block}{Ist Python schon installiert ?}
	\begin{itemize}
		\item Öffne ein Terminal/die Eingabeaufforderung
		\item Gib ein \mintinline{bash}{python --version}
		\item oder alternativ \mintinline{bash}{python3 --version}
		\item Erhältst Du die Antwort \mintinline{bash}{Python} und eine Zahl $\geq 3.6$, dann ist alles fein
		\item Falls nicht: Installiere Python!
	\end{itemize}
\end{block}
\end{frame}


\begin{frame}{Installation}
\begin{block}{Installation}
\begin{enumerate}
	\item Gehe auf https://www.python.org/downloads/
	\item Klicke den Button "Download Python 3.9.3."
	\item Führe die Installationsdatei aus
	\item Falls Du gefragt wirst, bestätige, dass Python zum PATH hinzugefügt wird
	\item Eventuell muss der Rechner neu gestartet werden
\end{enumerate}	
\end{block}
\end{frame}

\begin{frame}
\begin{block}{Cross-Check}
\vspace{2pt}
Gib \mintinline{bash}{python} (Win) oder \mintinline{bash}{python3} (Mac) im \textit{Terminal} ein.
Du solltest etwa folgendes sehen:  
\vspace{12pt}

\texttt{Python 3.9.2 (tags/v3.9.2:1a79785, Feb 19 2021, 13:44:55) [MSC v.1928 64 bit (AMD64)] on win32
	Type "help", "copyright", "credits"{} or "license"{} for more information.}

\texttt{>{}>{}>}
\end{block}
\begin{block}{}
Jetzt bist Du im \textit{interactive mode} von Python. Hier kannst Du einzelne Codezeilen eingeben und mittels \mintinline{bash}{Enter} ausführen. 
Um den interactive mode zu verlassen, gib \mintinline{python}{exit()} ein und bestätige mit der \mintinline{bash}{Enter}-Taste. 	
\end{block}
\end{frame}


%\begin{standout}
%	Erste Schritte im Interactive Mode\\
%\end{standout}
\section{Erste Schritte im Interactive Mode}


\begin{frame}
\begin{block}{Versuche folgende Kommandos}	
	\begin{itemize}
		\item \mintinline{python}{3 + 4}
		\item \mintinline{python}{2 - 7}
		\item \mintinline{python}{"Hello" + "Python"}
	\end{itemize}
\end{block}	
\end{frame}

\begin{frame}{Übung}
\uncover<+->{\begin{block}{Was machen die folgenden \textit{Operatoren}?}
	\begin{itemize}
		\item \mintinline[style=bw]{python}{+}
		\item \mintinline[style=bw]{python}{-}
		\item \mintinline[style=bw]{python}{*}
		\item \mintinline[style=bw]{python}{/}
		\item \mintinline[style=bw]{python}{**}
	\end{itemize}
\end{block}}
\uncover<+->{\begin{block}{Und diese?}
\begin{itemize}
		\item \%
		\item \mintinline[style=bw]{python}{//}
		\item \mintinline[style=bw]{python}{==}
		\item \mintinline[style=bw]{python}{<=}
		\item \mintinline[style=bw]{python}{<}
\end{itemize}
\end{block}}

\end{frame}

\begin{frame}{Übung}
	\begin{block}{Wie rechnet Python?}
		\begin{itemize}
			\item Wird Punkt-vor-Strich berücksichtigt?
			\item Kann man mit Klammern die Reihenfolge beeinflussen?
			\item Was ist der Unterschied zwischen \mintinline{python}{10/5} und \mintinline{python}{10//5} ?
			\item Was bedeutet das Kommando \mintinline{python}{_}? 
			\item Wie kann man Zwischenergebnisse in Variablen speichern?
		\end{itemize}
	\end{block}
\end{frame}

\section{Variablen}

\begin{frame}
\uncover<+->{\begin{block}{}
		Jeder Wert in Python kann in einer Variable gespeichert werden: 
		
		\mintinline{python}{my_variable = 3}
\end{block}}

\uncover<+->{\begin{block}{}
		Die Zuweisung darf auch das Ergebnis einer Berechnung sein: 
		
		\mintinline{python}{my_new_variable = 3 + 5}
\end{block}}
\uncover<+->{\begin{block}{}
		Die Zuweisung darf auch weitere Variablen enthalten: 
		
		\mintinline{python}{my_brand_new_variable = my_variable + my_new_variable }
\end{block}}

\uncover<+->{\begin{block}{}
	Man darf auch Kettenzuweisungen machen: 
	
	\mintinline{python}{a = b = c = 100 }
\end{block}}
\end{frame}


\begin{frame}
\begin{block}{Erlaubte Variablenname}
\begin{itemize}
	\item Beliebige Länge
	\item Bestehend aus Buchstaben (ASCII, klein/groß), Ziffern und Unterstrichen
	\item Der Name darf nicht mit einer Ziffer starten
	\item Schlüsselwörter sind nicht erlaubt
\end{itemize} 
\end{block}
\vspace{12pt}
\begin{block}{Liste der Schlüsselwörter}
	\texttt{
	\begin{columns}[T,onlytextwidth]
		\column{0.2\textwidth}
		False\\ 	await\\ 	else\\ 	import\\ 	pass\\ assert \\	del\\ 	
		\column{0.2\textwidth}
		None \\	break \\	except \\ 	in \\	raise \\ global \\	not \\	 
		\column{0.2\textwidth}
		True \\	class \\ 	finally \\ 	is \\	return \\ with \\ async 
		\column{0.2\textwidth}
		and \\	continue \\ 	for \\	lambda \\	try \\ 	elif  \\	if  \\
		\column{0.2\textwidth}
		as \\ 	def \\ 	from  \\	nonlocal \\	while \\ 	or \\ 	yield
	\end{columns}
}
\end{block}
\end{frame}

\begin{frame}
\begin{exampleblock}{Style-Guide Variablennamen}
	\begin{itemize}
		\item Immer englisch
		\item Nur Kleinbuchstaben
		\item Möglichst aussdrucksstarke Namen verwenden
		\item Keine Angst vor langen Namen 
		\item Namen, die aus mehreren Worten bestehen, mit Unterstrich trennen (\textit{snake-case})
	\end{itemize}
	
	z.B. \mintinline{python}{students_in_this_room}, \mintinline{python}{number_of_unpaid_bills}
\end{exampleblock}

\end{frame}


\section{Datentypen}

\begin{frame}
\begin{block}{}
	Jeder Wert in Python hat einen \textit{Datentyp}. Es gibt folgende \textit{primitive} Typen in Python.
	\begin{itemize}
		\item \mintinline{python}{int}  Integer (ganze Zahlen)
		\item \mintinline{python}{float} Float (Dezimalzahlen)
		\item \mintinline{python}{str}  String (Zeichenketten)
		\item \mintinline{python}{bool} Boolean (Wahrheitswerte)
		\item \mintinline[style=bw]{python}{None} (Typ des leeren Werts)
	\end{itemize}
\end{block}
\end{frame}


\begin{frame}
\metroset{block=fill}

\begin{block}{Integer}
Ganze Zahlen wie z.B. \mintinline{python}{1}, \mintinline{python}{-1}, \mintinline{python}{0}. Nicht aber 
\mintinline{python}{2.0} oder \mintinline{python}{0.0}. 	
\end{block}
\vspace{12pt}
\begin{block}{Float}
Fließkommazahlen, z.B. \mintinline{python}{3.1415925}. Achtung: Bei Float-Berechnungen können schnell Rundungsfehler auftreten: Was ergibt z.B. \mintinline{python}{1.2 - 1.0} ? 
\end{block}
\end{frame}

\begin{frame}
\metroset{block=fill}
\begin{block}{String}
Strings müssen in Anführungszeichen eingeschlossen werden. Dabei ist es fast egal, ob einfache, zweifache oder dreifache. Die Ausdrücke \mintinline{python}{'hello'}, \mintinline{python}{"Hello"} und \mintinline{python}{"""Hello"""} sind (fast) äquivalent. Strings können auch Steuerzeichen, wie Zeilenumbrüche enthalten: z.B. \mintinline{python}{"Hello\n"} 
\end{block} 

\vspace{12pt}
\begin{block}{Boolean}
Booleans können nur die Werte \mintinline{python}{True} und \mintinline{python}{False} haben. Sie entstehen in der Regel, wenn man Fragen im Programm stellt (z.B. \py{3 < 4} oder \py{1 == 2}).   	
\end{block}
\end{frame}

\begin{frame}
\metroset{block=fill}
\begin{block}{Woher weiß ich, welchen Typ eine Variable hat?}
	\vspace{2pt}
Mit der Funktion \py{type()} lässt sich der Typ bestimmen, z.B. \py{type(3.2)}.  	
\end{block}
\vspace{12pt}
\begin{block}{Wie lassen sich Typen umwandeln?}
	\vspace{2pt}
	\textbf{Implizit}\\
	Bei manchen Operationen nimmt Python automatisch eine Typumwandlung vor. z.B. \py{1+2.0} ergibt \py{3.0}	\\ \\
	\textbf{Explizit}\\
	Die Funktionen \py{int()}, \py{float()}, \py{str()} und \py{bool()} führen jeweils eine Typumwandlung durch (sofern möglich). Beispiele: 
	\begin{itemize}
		\item \py{int(2.0)} ergibt \py{2} 
		\item \py{float(2)} ergibt \py{2.0} 
		\item \py{int("3")} ergibt \py{3}
	\end{itemize} 
\end{block}


\end{frame}


\begin{frame}{Übung}

\begin{block}{Versuche die Fragen erst ohne Python zu beantworten, überprüfe Deine Vermutung}
	\begin{itemize}
		\item Welchen Datentyp hat das Ergebnis von \py{3 - 1.2} ?
		\item Was ist das Ergebnis von \py{"2" + 1} ? 
		\item Sind die beiden Werte \py{0} und \py{"0"} gleich? 
		\item Sind die beiden Werte \py{2} und \py{True} gleich? 
		\item Sind die beiden Werte \py{bool(2)} und \py{True} gleich? 
		\item Sind die beiden Werte \py{1} und \py{True} gleich? 
	\end{itemize}
\end{block}


\end{frame}


\section{Operatoren}

\begin{frame}
\begin{block}{Die wichtigsten Operatoren}
	\begin{itemize}
		\item \pybw{+} (Addition oder Zusammenkleben von Strings)
		\item \pybw{-} (Subtraktion)
		\item \pybw{*} (Multiplikation)
		\item \pybw{/} (Division, ergibt immer ein Wert vom Typ \pybw{float})
		\item \pybw{**} (Potenzierung)
			\item \% (\textit{modulo-Operator}: Rest bei ganzzahliger Division)
		\item \pybw{//} (Division und Abrunden, ergibt immer ein Wert vom Typ \pybw{int})
		\item \pybw{==} (Vergleichsoperator, ergibt immer ein Wert vom Typ \pybw{bool})
	\end{itemize}
\end{block}
\end{frame}

\begin{frame}
\begin{block}{Präzedenz (oben höchste, unten niedrigste)}
	\begin{enumerate}
		\item Klammern
		\item \pybw{**}
		\item \pybw{*}, \pybw{/}, \pybw{//}, \%
		\item \pybw{+},\pybw{-}
	\end{enumerate}	
Operatoren gleichen Rangs werden von links nach rechts abgearbeitet (außer Potenzierung). 
\end{block}
\end{frame}

\begin{fragile}[]
	\begin{block}{Kombinierte Zuweisung}
		\vspace{2pt}
		Oft möchte man eine gegebene Variable neu zuweisen: 
		\begin{minted}{python}
		counter = 1
		counter = counter + 1 	# counter = 2
		\end{minted}
	Dies lässt sich auch kurz schreiben als 
		\begin{minted}{python}
	counter = 1
	counter += 1 	# counter = 2
	\end{minted}
	Analog sind die Operatoren \py{-=}, \py{*=}, \py{/=}, etc. definiert. 
	\end{block}
	
	
	\end{fragile}



\section{Input/Output -- Basics}

\begin{fragile}[]

\begin{block}{Output}
	\vspace{2pt}
	Um einen String auf der Konsole auszugeben, verwende die Funktion \py{print()}. 
	Zum Beispiel \py{print("Hello there")}. 
	
	Es können auch Variablen eingesetzt werden: 
	\begin{minted}{python}
	message = "Hello there"
	print(message) # Hello there
	\end{minted}
	
	\end{block}
	
\end{fragile}

\begin{fragile}[]
	
\begin{block}{String Interpolation}
\vspace{2pt}
Um Variablenwerte innerhalb eines Strings auszugeben, verwenden wir die String-Interpolation-Syntax
\begin{minted}{python}
my_value = 5
print(f"The variable my_value has the value {my_value}")
# The variable my_value has the value 5
\end{minted}

Das geht auch als \textit{inline expression}: 
\begin{minted}{python}
print(f"The sum of 1 and 2 is {1+2}")
# The sum of 1 and 2 is 3
\end{minted}

\end{block}
	
\end{fragile}

\begin{fragile}
\begin{block}{Output}
	\vspace{2pt}
Um einen String vom User einzulesen, verwende die Funktion \py{input()}:
\begin{minted}{python}
age = input("How old are you?")
print(f"I am {age} years old")
\end{minted}
\end{block}

\begin{alertblock}{Achtung}
	\vspace{2pt}
Das Ergebnis von \py{input} hat stets den Datentyp \py{string} auch wenn Zahlen eingelesen werden. Gegebenenfalls muss das Ergebnis mittels \py{int()} oder \py{float()} in den gewünschten Typ umgewandelt werden. 	
\end{alertblock}

\end{fragile}

\section{Erste Schritte im Script Mode}
\begin{frame}
\begin{block}{Script Mode}
	\vspace{2pt}
	Sobald man mehrere zusammenhängende Zeilen hat, wird Pythons \textit{interactive mode} sehr unübersichtlich. Daher gibt es auch die Möglichkeit, alle Programmzeilen in eine Text-Datei zu schreiben und diese gebündelt auszuführen.   
\end{block}

\end{frame}
\begin{fragile}[]
	\begin{exampleblock}{Ein erstes Beispiel}
		\begin{minted}{python}
		name = input("What is your name?")
		age = input("What is your age?")
		print(f"Hello {name}, you are {age} years old") 
		\end{minted}
		
		Speichere diesen Code in der Datei \py{my_script.py} ab. 
		
		Führe danach in diesem Ordner das Kommando 
		\pybw{python my_script.py} aus. 
	\end{exampleblock}
\end{fragile}



\end{document}






