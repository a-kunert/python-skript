\documentclass[algorithm,
pgfplots,
colortheme=dark,
handout
]{cuzbeamer}
\usepackage[ngerman]{babel}
\usepackage[scale=2]{ccicons}
\usepackage{listings}
\usepackage{csquotes}

\newcommand{\py}[1]{\mintinline{python}{#1}}
\newcommand{\pybw}[1]{\mintinline[style=bw]{python}{#1}}
\newcommand{\bash}[1]{\mintinline{bash}{#1}}
\newcommand{\spacechar}{\texttt{\char32\hspace{2pt}}}

%\newenvironment{mysolution}{\begin{frame}<beamer:0>[fragile]\begin{exampleblock}{Lösung}\begin{minted}{python}}%
%		{\end{minted}\end{exampleblock}\end{frame}}


%\newtheorem{solutionblock}{Example}
\newenvironment<>{solutionblock}[1]{%
\setbeamercolor{block title example}{fg=cyan!60!white}%
\begin{exampleblock}<beamer:0>{#1}
\vspace{2pt}
	
}%
{\end{exampleblock}}




\begin{document}
\title{Programmieren mit Python}
\subtitle{Eine Einführung}
\date{\today}
\author{Dr. Aaron Kunert}
\email{aaron.kunert@abiturma.de}
\subtitle{Teil 5: Dictionaries und Funktionen}
\date{20. Mai 2021}

%\subtitle{Eine Einführung}
%\date{\today}
\maketitle






%\section{Zu Beginn \dots}

\begin{frame}
\begin{block}{Kurze Vorstellungsrunde}
	\vspace{2pt}
	Schaffst Du es \emph{in 60 Sekunden} folgende Fragen möglichst knackig und aussagekräftig zu beantworten?
	\begin{itemize}
		\item Wer bist Du? 
		\item Windows, Mac oder Linux?
		\item Welche Vorkenntnisse hast Du beim Programmieren?  
		\item Warum hast Du Dich zum Python-Kurs angemeldet? 
		\item Wann wäre der Kurs für Dich perfekt gelaufen? (Best Case Szenario)
		\item Wann würdest Du den Kurs nicht weiter besuchen? (Worst Case Szenario)
	\end{itemize}
\end{block}
\end{frame}


\begin{frame}
	\begin{block}{Ablauf des Kurses}
		\begin{itemize}
			\item Mischung aus Vortrag, Live-Coding und Präsenzübungen
			\item Im Idealfall: Mehr Praxis statt Erklärungen
			\item Jede Woche gibt's ein Aufgabenblatt $\rightarrow$ Besprechung in der nächsten Woche
			\item Kommunikation über Slack: \texttt{https://bit.ly/3a5W9fE} 
			%\texttt{python-kurs.slack.com} 
			(freiwillig)
		\end{itemize}
	\end{block}
\end{frame}

\begin{frame}
	\begin{block}{Warum Python?}
		\begin{itemize}
			\item Einfaches Setup
			\item Einstiegsfreundliche Syntax
			\item Python ist eine Hochsprache
			\item Python muss nicht kompiliert, sondern nur interpretiert werden
			\item Große Community $\rightarrow$ großes \emph{Ecosystem}
			\item Python ist extrem vielseitig
			\item Python ist plattformunabhängig
		\end{itemize}
	\end{block}
\end{frame}

\begin{frame}
	\begin{block}{Typische Einsatzbereiche}
		\begin{itemize}
			\item Automatisierung
			\item Webscraping
			\item Datenanalyse
			\item Webentwicklung
		\end{itemize}
	\end{block}
\end{frame}

\begin{frame}
\begin{block}{Phasen des Lernens einer Programmiersprache}
	\vspace{2pt}
	\uncover<+->{
\begin{enumerate}[<+->]
	\item Annäherung:  Fokus auf dem Begreifen der Grundkonzepte
	\item Syntax: Fokus auf der korrekten Anwendung der Syntax
	\item Funktionalität: Fokus liegt darauf, Problemstellungen \emph{pragmatisch} zu lösen
	\item Design: Fokus auf les-und wartbaren Code
	\item Architektur: Fokus auf Strategie, Projekte nachhaltig und erweiterbar umzusetzen 
\end{enumerate}
	}
\end{block}
\end{frame}


\begin{frame}
	\metroset{block=fill}
	\uncover<+->{\begin{block}{Was wird benötigt?}
			\vspace{2pt}
			\uncover<+->{
				\textbf{Am Anfang}
				\begin{itemize}
					\item Compiler/Interpreter
					\item Texteditor (z.B. Mac: Xcode, Windows: Edit)
				\end{itemize}
			}
			\uncover<+->{
				\textbf{Später}
				\begin{itemize}
					\item Google
					\item Integrierte Entwicklungsumgebung (IDE)
					\item Versionskontrolle (VCS)
					\item Virtueller Maschinen
					\item Datenbanken
					\item Grafikbearbeitung
				\end{itemize}
			}
	\end{block}}
\end{frame}

\begin{frame}
\begin{block}{Wo findet man Hilfe/Infos?}
	\vspace{2pt}
	\begin{itemize}
		\item Google
		\item \texttt{stackoverflow.com}
		\item Youtube (z.B. Tutorials)
		\item Austausch über Slack 
		\item \texttt{docs.python.org/3}
		\item Bücher (z.B. \textit{Python Crashkurs} v. \textsc{Eric Matthes})
		\item \texttt{mailto: aaron.kunert@salemkolleg.de}
	\end{itemize}
\end{block}
\end{frame}


\section{Installation von Python}

\begin{frame}
\begin{block}{Ist Python schon installiert ?}
	\begin{itemize}
		\item Öffne ein Terminal/die Eingabeaufforderung
		\item Gib ein \bash{python --version}
		\item oder alternativ \bash{python3 --version}
		\item Erhältst Du die Antwort \bash{Python} und eine Zahl $\geq 3.6$, dann ist alles fein
		\item Falls nicht: Installiere Python!
	\end{itemize}
\end{block}
\end{frame}


\begin{frame}
\begin{block}{Installation}
\begin{enumerate}
	\item Gehe auf https://www.python.org/downloads/
	\item Klicke den Button "Download Python 3.9.3."
	\item Führe die Installationsdatei aus
	\item Falls Du gefragt wirst, bestätige, dass Python zum PATH hinzugefügt wird
	\item Eventuell muss der Rechner neu gestartet werden
\end{enumerate}	
\end{block}

\begin{alertblock}{Achtung bei Windows}
\vspace{2pt}
Python muss zum PATH hinzugefügt werden. 

\includegraphics[width=0.65\textwidth]{python_path.jpg}
\end{alertblock}

\end{frame}

\begin{frame}
\begin{block}{Cross-Check}
\vspace{2pt}
Gib \bash{python} (Win) oder \bash{python3} (Mac) im \textit{Terminal} ein.
Du solltest etwa folgendes sehen:  
\vspace{12pt}

\texttt{Python 3.9.2 (tags/v3.9.2:1a79785, Feb 19 2021, 13:44:55) [MSC v.1928 64 bit (AMD64)] on win32
	Type "help", "copyright", "credits"{} or "license"{} for more information.}

\texttt{>{}>{}>}
\end{block}
\begin{block}{}
Jetzt bist Du im \textit{interactive mode} (REPL) von Python. Hier kannst Du einzelne Codezeilen eingeben und mittels \bash{Enter} ausführen. 
Um den interactive mode zu verlassen, gib \py{exit()} ein und bestätige mit der \bash{Enter}-Taste. 	
\end{block}
\end{frame}


%\begin{standout}
%	Erste Schritte im Interactive Mode\\
%\end{standout}
\section{Erste Schritte im REPL \\ \footnotesize (Read-Evaluate-Print-Loop)}


\begin{frame}
\begin{block}{Probier mal folgende Kommandos aus}	
	\begin{itemize}
		\item \py{3 + 4}
		\item \py{2 - 7}
		\item \py{"Hello" + "Python"}
	\end{itemize}
\end{block}	
\end{frame}



\begin{frame}{Übung}
\uncover<+->{\begin{block}{Was machen die folgenden \textit{Operatoren}?}
	\begin{itemize}
		\item \pybw{+}
		\item \pybw{-}
		\item \pybw{*}
		\item \pybw{/}
		\item \pybw{**}
	\end{itemize}
\end{block}}
\uncover<+->{\begin{block}{Und diese?}
\begin{itemize}
		\item \%
		\item \pybw{//}
		\item \pybw{==}
		\item \pybw{<=}
		\item \pybw{<}
\end{itemize}
\end{block}}

\end{frame}

\begin{frame}{Übung}
	\begin{block}{Wie rechnet Python?}
		\begin{itemize}
			\item Wird Punkt-vor-Strich berücksichtigt?
			\item Kann man mit Klammern die Reihenfolge beeinflussen?
			\item Was ist der Unterschied zwischen \py{10/5} und \py{10//5} ?
			\item Was bedeutet das Kommando \py{_}? 
			\item Wie kann man Zwischenergebnisse in Variablen speichern?
		\end{itemize}
	\end{block}
\end{frame}

\section{Variablen}

\begin{frame}
\uncover<+->{\begin{block}{}
		Jeder Wert in Python kann in einer Variable gespeichert werden: 
		
		\py{my_variable = 3}
\end{block}}

\uncover<+->{\begin{block}{}
		Die Zuweisung darf auch das Ergebnis einer Berechnung sein: 
		
		\py{my_new_variable = 3 + 5}
\end{block}}
\uncover<+->{\begin{block}{}
		Die Zuweisung darf auch weitere Variablen enthalten: 
		
		\py{my_brand_new_variable = my_variable + my_new_variable }
\end{block}}

\uncover<+->{\begin{block}{}
	Man darf auch Kettenzuweisungen machen: 
	
	\py{a = b = c = 100 }
\end{block}}
\end{frame}


\begin{frame}
\uncover<+->{\begin{block}{Gültige Variablennamen}
\begin{itemize}[<+->]
	\item Erlaubt sind Buchstaben (nur ASCII), Ziffern und Unterstriche
	\item Der Name darf nicht mit einer Ziffer starten
	\item Beliebige Länge 
	\item Wer's schon kennt als \emph{regulärer Ausdruck}:  \mintinline{php}{[_a-zA-Z][_0-9a-zA-Z]*}
	\item Schlüsselwörter sind nicht erlaubt
\end{itemize} 
\end{block}}
\vspace{12pt}
\uncover<+->{\begin{block}{Liste der Schlüsselwörter}
	\texttt{
	\begin{columns}[T,onlytextwidth]
		\column{0.2\textwidth}
		False\\ 	await\\ 	else\\ 	import\\ 	pass\\ assert \\	del\\ 	
		\column{0.2\textwidth}
		None \\	break \\	except \\ 	in \\	raise \\ global \\	not \\	 
		\column{0.2\textwidth}
		True \\	class \\ 	finally \\ 	is \\	return \\ with \\ async 
		\column{0.2\textwidth}
		and \\	continue \\ 	for \\	lambda \\	try \\ 	elif  \\	if  \\
		\column{0.2\textwidth}
		as \\ 	def \\ 	from  \\	nonlocal \\	while \\ 	or \\ 	yield
	\end{columns}
}
\end{block}}
\end{frame}


\begin{frame}
\uncover<+->{\begin{exampleblock}{Style-Guide Variablennamen}
	\begin{itemize}
		\item Englische Wörter
		\item Nur Kleinbuchstaben
		\item Möglichst aussdrucksstarke Namen verwenden
		\item Keine Angst vor langen Namen 
		\item Namen, die aus mehreren Worten bestehen, mit Unterstrich trennen (\textit{snake-case})
	\end{itemize}
	
	\uncover<+->{z.B. \py{students_in_this_room}, \py{number_of_unpaid_bills}}
\end{exampleblock}}

\end{frame}

\begin{frame}{Übung}
	
	\begin{block}{Probier's aus!}
		\begin{itemize}
			\item Welchen Wert hat eine Variable, wenn man sie nicht vorher definiert hat? 
			\item Was passiert, wenn man eine Variable definiert, die schonmal verwendet wurde?
			\item Wie kann man eine Variable mit Wert \py{3} um \py{1} vergrößern?
		\end{itemize}	
	\end{block}
	
\end{frame}


\section{Datentypen}

\begin{frame}
	\begin{block}{}
		Jeder Wert in Python hat einen \textit{Datentyp}. Unter anderem gibt es folgende \textit{primitive} Typen in Python.
		\begin{itemize}
			\item \py{int}  Integer (ganze Zahlen)
			\item \py{float} Float (Dezimalzahlen)
			\item \py{bool} Boolean (Wahrheitswerte)
			\item \py{str}  String (Zeichenketten)
			\item \pybw{NoneType} (Typ des leeren Werts \py{None})
		\end{itemize}
	\end{block}
\end{frame}


\begin{frame}
	\metroset{block=fill}
	
	\uncover<+->{\begin{block}{Integer}
		Ganze Zahlen wie z.B. \py{1}, \py{-1}, \py{0}. Nicht aber 
		\py{2.0} oder \py{0.0}. 	
	\end{block}}
	\vspace{12pt}
	\uncover<+->{\begin{block}{Float}
		Fließkommazahlen, z.B. \py{3.1415925}. Achtung: Bei Float-Berechnungen können schnell \enquote{Überraschungen} auftreten: Was ergibt z.B. \mintinline{python}{1.2 - 1.0} ? 
	\end{block}}
	\vspace{12pt}
	\uncover<+->{\begin{block}{Boolean}
		Booleans sind eine Sonderform von \py{int} und können nur die Werte \py{True} (entspricht 1) und \py{False} (entspricht 0) annehmen. Sie entstehen in der Regel, wenn man Fragen im Programm stellt (z.B. \py{3 < 4} oder \py{1 == 2}).   	
	\end{block}}
\end{frame}



\begin{frame}
	\metroset{block=fill}
	\uncover<+->{\begin{block}{String}
		Strings sind beliebige Zeichenketten und müssen in (ein-, zwei- oder dreifache) Anführungszeichen eingeschlossen werden. Die Ausdrücke \py{'hello'}, \py{"Hello"} und \py{"""Hello"""} sind (fast) äquivalent. 
	\end{block}}
	\vspace{12pt}

	\uncover<+->{\begin{block}{Mehrzeilige Strings}
		Ein \textit{Stringliteral} kann nur innerhalb einer Zeile definiert werden. Soll ein String mehrere Zeilen umfassen, müssen dreifache Anführungszeichen verwendet werden.  
	\end{block}}

	\end{frame}

	\begin{frame}
			\metroset{block=fill}
		\uncover<+->{\begin{block}{Steuerzeichen}
			Gewisse Kombinationen mit Backslash sind reservierte Steuerzeichen. So bezeichnet beispielsweise \py{\n} einen Zeilenumbruch und \py{\t} ein Tabulatorzeichen. \\
			Beispiel: \py{"This text\nfills two lines"}
		\end{block}}
			\vspace{12pt}
		\uncover<+->{\begin{block}{Escaping}
			Möchte man ein Steuerzeichen nicht ausführen, sondern buchstäblich nehmen. Muss man sie mit einem Backslash \textit{escapen} bzw. maskieren. \\
			Beispiel: \py{"This text fits in\\n one line"}
		\end{block}}
		\vspace{12pt}
		\uncover<+->{\begin{block}{Raw-Strings}
				Möchte man alle Steuerzeichen eines Strings ignorieren, kann man ihn als \textit{Raw-String} definieren. \\
				Beispiel: \py{r"This \n String \t has no control characters"}
		\end{block}}
		
	\end{frame}

	
	\begin{frame}
		\metroset{block=fill}
		\uncover<+->{\begin{block}{Typ einer Variablen ermitteln}
			\vspace{2pt}
			Mit der Funktion \py{type()} lässt sich der Typ bestimmen, z.B. \py{type(3.2)}.  	
		\end{block}}
		\vspace{12pt}
		\uncover<+->{\begin{block}{Typumwandlung (\emph{Typecasting})}
			\vspace{2pt}
			\uncover<+->{\textbf{Implizit}\\
			Bei manchen Operationen nimmt Python automatisch eine Typumwandlung vor. \\ Beispiel: \py{1 + 2.0} ergibt \py{3.0}	
		} \\ \\
		\uncover<+->{\textbf{Explizit \\}
			Die Funktionen \py{int()}, \py{float()}, \py{str()} und \py{bool()} führen jeweils eine Typumwandlung durch (sofern möglich). Beispiele: 
			\begin{itemize}
				\item \py{int(2.0)} ergibt \py{2} 
				\item \py{float(2)} ergibt \py{2.0} 
				\item \py{int("3")} ergibt \py{3}
			\end{itemize} 
		}
		\end{block}}
		
		
	\end{frame}
	
	
	\begin{frame}{Übung}
		
		\begin{block}{Versuche die Fragen erst ohne Python zu beantworten, überprüfe Deine Vermutung}
			\begin{itemize}
				\item Welchen Datentyp hat das Ergebnis von \py{3 - 1.0} ?
				\item Was ist das Ergebnis von \py{"2" + 1} ? 
				\item Was ist das Ergebnis von \py{"2"} + \py{"2"}? 
				\item Sind die beiden Werte \py{0} und \py{"0"} gleich? 
				\item Sind die beiden Werte \py{2} und \py{True} gleich? 
				\item Sind die beiden Werte \py{bool(2)} und \py{True} gleich? 
				\item Sind die beiden Werte \py{1} und \py{True} gleich? 
			\end{itemize}
		\end{block}
		
		
	\end{frame}

	\begin{frame}{Übung}
	
	\begin{block}{Erkläre mit Deinen eigenen Worten}
		\begin{itemize}
			\item Nach welcher Regel wandelt \py{int()} eine Fließkommazahl in eine ganze Zahl um? 
			\item Nach welchen Regeln wandelt \py{bool()} Zahlen und Strings in einen Wahrheitswert um? 
		\end{itemize}
	\end{block}
	
	
\end{frame}




\section{Operatoren}

\begin{frame}
\begin{block}{Die wichtigsten Operatoren}
	\begin{itemize}
		\item \pybw{+} (Addition oder Zusammenkleben von Strings)
		\item \pybw{-} (Subtraktion)
		\item \pybw{*} (Multiplikation)
		\item \pybw{/} (Division, ergibt immer ein Wert vom Typ \pybw{float})
		\item \pybw{**} (Potenzierung)
			\item \% (\textit{modulo-Operator}: Rest bei ganzzahliger Division)
		\item \pybw{//} (Division und Abrunden, ergibt immer ein Wert vom Typ \pybw{int})
		\item \pybw{==} (Vergleichsoperator, ergibt immer ein Wert vom Typ \pybw{bool})
		\item \pybw{!=} (Ungleichheitsoperator, ergibt das Gegenteil von \pybw{==})
	\end{itemize}
\end{block}
\end{frame}

\begin{frame}
\begin{block}{Operator-Präzedenz}
	\uncover<+->{
	\begin{enumerate}[<+->]
		\item Klammern
		\item \pybw{**}
		\item \pybw{*}, \pybw{/}, \pybw{//}, \%
		\item \pybw{+},\pybw{-}
	\end{enumerate}	
	}
\uncover<+->{
Operatoren gleichen Rangs werden innerhalb eines Ausdrucks von links nach rechts abgearbeitet. 
}

\uncover<+->{
\vspace{10pt}
\textbf{Ausnahmen:}\\
Potenzierung (\py{**}) und Zuweisung (\py{=}) werden von rechts nach links verarbeitet. 
}
\end{block}
\end{frame}

\begin{fragile}[]
\begin{block}{Kombinierte Zuweisung}
		\vspace{2pt}
		Oft möchte man eine gegebene Variable neu zuweisen: 
		\begin{minted}{python}
		counter = 1
		counter = counter + 1 	# counter = 2
		\end{minted}
		\pause
		Dies lässt sich auch kurz schreiben als 
		\begin{minted}{python}
		counter = 1
		counter += 1 	# counter = 2
		\end{minted}
		\pause
		Analog sind die Operatoren \py{-=}, \py{*=}, \py{/=}, etc. definiert. 
	\end{block}
\end{fragile}



\section{Input/Output \\ \footnotesize Teil 1}

\begin{fragile}[]

\begin{block}{Output}
	\vspace{2pt}
	Um einen String auf der \emph{Konsole} auszugeben, verwende die Funktion \py{print()}. 
	
	
	Zum Beispiel \py{print("Hello there")}. 
	\pause
	
	Es können auch Variablen eingesetzt werden: 
	\begin{minted}{python}
	message = "Hello there"
	print(message) # Hello there
	\end{minted}
	
	\end{block}
	
\end{fragile}

\begin{fragile}[]
	
\begin{block}{String Interpolation}
\vspace{2pt}
Um Variablenwerte innerhalb eines Strings auszugeben, verwenden wir die String-Interpolation-Syntax:
\begin{minted}{python}
my_value = 5
print(f"The variable my_value has the value {my_value}")
# The variable my_value has the value 5
\end{minted}
\pause
Das geht auch als \textit{inline expression}: 
\begin{minted}{python}
print(f"The sum of 1 and 2 is {1+2}")
# The sum of 1 and 2 is 3
\end{minted}

\end{block}
	
\end{fragile}

\begin{fragile}
\begin{block}{Output}
	\vspace{2pt}
Um einen String vom User einzulesen, verwende die Funktion \py{input()}:

\begin{minted}{python}
age = input("How old are you?")
print(f"I am {age} years old")
\end{minted}
\end{block}
\pause 
\begin{alertblock}{Achtung}
	\vspace{2pt}
Das Ergebnis von \py{input} hat stets den Datentyp \py{string} auch wenn Zahlen eingelesen werden. Gegebenenfalls muss das Ergebnis mittels \py{int()} oder \py{float()} in den gewünschten Typ umgewandelt werden. 	
\end{alertblock}

\end{fragile}

\section{Script Mode}
\begin{frame}
\begin{block}{Script Mode}
	\vspace{2pt}
	Sobald man mehrere zusammenhängende Zeilen hat, wird Pythons \textit{interactive mode} sehr unübersichtlich. Daher gibt es auch die Möglichkeit, alle Programmzeilen in eine Text-Datei zu schreiben und diese gebündelt auszuführen.   
\end{block}

\end{frame}
\begin{fragile}[]
	\begin{exampleblock}{Ein erstes Beispiel}
		\begin{minted}{python}
		name = input("What is your name?")
		age = input("What is your age?")
		print(f"Hello {name}, you are {age} years old") 
		\end{minted}
		
		Speichere diesen Code in der Datei \pybw{my_script.py} ab. 
		
		Führe danach in diesem Ordner das Kommando 
		\pybw{python my_script.py} aus. 
	\end{exampleblock}
\end{fragile}

\begin{frame}{Übung}
	Schreibe ein kurzes Skript, dass Dich nach Deinem Namen, Alter und Adresse fragt. Wenn es alles eingelesen hat, soll es diese Infos in folgender Form auf der Konsole ausgeben: 	
	
	\texttt{Hallo Max, schön dass Du da bist. Du bist 21 Jahre alt und wohnst in der Bismarckstraße 12 in Glücksstadt.}
	
\end{frame}


\begin{fragile}
	
	\begin{block}{Kommentare}
		\vspace{2pt}
		Alle Zeichen einer Zeile, die hinter einem \texttt{\#} (Hashtag) kommen, werden von Python ignoriert.
		So lassen sich Kommentare im Quellcode platzieren. 
	\end{block}
	\vspace{12pt}
	\begin{exampleblock}{Beispiel}
	\begin{minted}{python}
	print("This line will be printed")
	# print("This line won't") 
	\end{minted}
	\end{exampleblock}
	
\end{fragile}


\section{Arbeit mit einer IDE}

\begin{frame}
	\begin{block}{Integrierte Entwicklungsumgebungen (IDE)}
		\vspace{2pt}
		\uncover<+->{
	Die Arbeit mit gewöhnlichen Texteditoren ist auf Dauer sehr mühsam. Daher empfiehlt es sich eine IDE zu verwenden. 
	Das bringt zum u.a. folgende Vorteile: 
	\begin{itemize}[<+->]
		\item Syntax-Highlighting
		\item Code-Inspection
		\item Autocomplete
		\item Geile Shortcuts
		\item Code direkt ausführen
		\item Hilfe bei der Fehlersuche (\emph{Debugging})
	\end{itemize}	
		}
	\end{block}
\end{frame}

\begin{frame}
	\uncover<+->{
	\begin{block}{Installation von PyCharm}
		\vspace{2pt}
		\begin{enumerate}
			\item Gehe auf 
			\texttt{https://www.jetbrains.com/pycharm/download} \\
			\item Lade die kostenlose \textit{Community Edition} herunter
			\item Führe den Installer aus
			\item Öffne PyCharm
		\end{enumerate}
	\end{block}
	}
	
\uncover<+->{
	\begin{block}{Wenn alles passt, sollte es etwa so aussehen:}
		\vspace{2pt}
		\begin{center}
		\includegraphics[width=0.65\textwidth]{pycharme.jpg}
		\end{center}
	\end{block}
}	
	
\end{frame}



\begin{frame}
	\begin{block}{Installation PyCharm}
		\vspace{2pt}
		\begin{enumerate}
			\item Gehe auf \texttt{Customize > All Settings...}
			\item Einstellungen synchronisieren
			\begin{enumerate}
				\item \texttt{Tools > Settings Repository}
				\item Unter \textit{Read-only Sources} auf \texttt{+}
				\item \texttt{https://github.com/a-kunert/ide-settings.git}	eingeben
				%\item  File > Manage IDE Settings > Sync with Settings Repository > Merge ausführen 
			\end{enumerate}
			\item Verknüpfe den Interpreter
			\begin{enumerate}
				\item In den Settings auf \texttt{Python Interpreter}
				\item Falls möglich unter \texttt{Python Interpreter} einen Interpreter wählen. Ansonsten wie folgt: 
				\item \texttt{Zahnrad > Add}
				\item \texttt{System Interpreter}
				\item Dort den Pfad zu Python angeben
			\end{enumerate}
			\item Mit dem Button \textit{Apply} alles bestätigen
		\end{enumerate}
	\end{block}
\end{frame}

\begin{frame}
	\begin{block}{Konfiguration abschließen}
		\begin{enumerate}
		\item Lege eine Ordner für den Kurs an
		\item Öffne diesen Ordner mit \texttt{Projects > Open}
		\item \texttt{File > Manage IDE Settings > Sync with Settings Repository > Merge} ausführen
		\item Bei \texttt{File > Settings} unter \texttt{Keymap} die Keymap \textit{Salem-Win/Mac} auswählen. 
		\item Code in die Datei \texttt{main.py} schreiben
		\item Mittels grünem Pfeil (oben rechts) Code ausführen
	\end{enumerate}	
	\end{block}
\end{frame}
%\section{Script Mode}
\begin{frame}
\begin{block}{Script Mode}
	\vspace{2pt}
	Sobald man mehrere zusammenhängende Zeilen hat, wird die REPL sehr unübersichtlich. Daher gibt es auch die Möglichkeit, alle Programmzeilen zunächst aufzuschreiben und diese dann gebündelt von Python ausführen zu lassen. Im Gegensatz zum REPL bzw. interactive Mode von Python wird dies \emph{Script Mode} genannt.    
\end{block}
\end{frame}

\begin{fragile}[]
\begin{exampleblock}{Beispiel}
\begin{minted}{python}
name = "Max"
age = 20
print(f"Hello, I'm {name} and I'm {age} years old") 
\end{minted}
\end{exampleblock}
\pause
\begin{alertblock}{Problem:}
\vspace{2pt}
Wie kann man Python erklären, diese 3 Zeilen auf einmal auszuführen? 
\end{alertblock}
\end{fragile}

\begin{frame}
\begin{block}{Old-School-Lösung}
\vspace{2pt}
\begin{itemize}
	\item Erstelle eine neue Datei (z.B. \pybw{my_script.py})
	\item Öffne die Datei mit einem Texteditor und speichere den Beispiel-Code darin ab.
	\item Öffne den Ordner mit der Datei \pybw{my_script.by} mit dem Terminal bzw. der Eingabeaufforderung
	\item Führe das Kommando \pybw{python my_script.py} aus. 
\end{itemize}
\end{block}
\end{frame}

\begin{frame}
\begin{block}{Optimallösung: Verwende eine IDE}
	\vspace{2pt}	
	Eine IDE (integrierte Entwicklungsumgebung) hilft Dir beim Programmieren und unterstüzt Dich wo immer möglich. Dadurch lassen sich auch große Projekte schnell umsetzen. 
\end{block}
\pause 

\vspace{12pt}

\begin{alertblock}{Nachteile}
	\vspace{2pt}
	Die anfängliche Einrichtung kann schnell kompliziert werden. Aufgrund der vielen Features fühlt man sich schnell mal überfordert. 
	\pause 
	
	$\rightarrow$ Das machen wir etwas später.	
\end{alertblock}
\end{frame}


\begin{frame}
\begin{block}{Kompromiss für den Anfang: Browserbasierte Editoren}
\vspace{2pt}
Um schnell einzusteigen, kann zu Beginn auch ein browsergestützter Editor/Interpreter verwendet werden. Zum Beispiel: 
\pause
\begin{itemize}
	\item Programiz (\texttt{https://www.programiz.com/python-programming/online-compiler})
	\begin{itemize}
		\item einfacher Einstieg
		\item schnell und unkompliziert
		\item geringer Funktionsumfang
	\end{itemize}
	\pause
	\item Repl.it (\texttt{https://replit.com/languages/python3})
	\begin{itemize}
		\item Auch für viele andere Sprachen geeignet
		\item Manchmal etwas langsam
		\item Man kann mehrere Dateien und Projekte verwalten (braucht Account) 
		\item Hat fast alle IDE-Features (braucht Account)
	\end{itemize}
%	\item Onecompiler (\texttt{https://onecompiler.com/python})
%	\begin{itemize}
%		\item Benötigt immer einen Account
%		\item Viele Funktionen
%	\end{itemize}
\end{itemize}
\end{block}

\end{frame}



\section{Input/Output \\ \footnotesize Kommunikation über die Konsole}


\begin{frame}

\begin{block}{Die Konsole}
\vspace{2pt}
Da wir zu Beginn noch über keine grafische Benutzeroberfläche verfügen, verwenden wir für die Kommunikation mit unserem Programm die \emph{Konsole}. 
Dabei handelt es sich um ein einfaches Textfenster, auf dem Dein Programm Informationen ausgeben kann (\emph{Output}) und Text einlesen kann (\emph{Input}). 
\end{block}

\end{frame}

\begin{fragile}[]
	
	\begin{block}{Output}
		\vspace{2pt}
		Um einen String auf der \emph{Konsole} auszugeben, verwende die Funktion \py{print()}. 
		
		
		Zum Beispiel: \py{print("Hello there")}. 
		\pause
		
		\vspace{12pt}
		
		Es können auch Variablen eingesetzt werden: 
		\begin{minted}{python}
		message = "Hello there"
		print(message) # Hello there
		\end{minted}
		
	\end{block}
	
\end{fragile}

\begin{fragile}[]
	
	\begin{block}{String Interpolation}
		\vspace{2pt}
		Um Variablenwerte innerhalb eines Strings auszugeben, verwenden wir die String-Interpolation-Syntax:
		\begin{minted}{python}
		my_value = 5
		print(f"The variable my_value has the value {my_value}")
		# The variable my_value has the value 5
		\end{minted}
		
		\pause
		
		\vspace{12pt}
		
		Das geht auch als \textit{inline expression}: 
		\begin{minted}{python}
		print(f"The sum of 1 and 2 is {1+2}")
		# The sum of 1 and 2 is 3
		\end{minted}
		
	\end{block}
	
\end{fragile}

\begin{fragile}
	\begin{block}{Input}
		\vspace{2pt}
		Um einen String vom User einzulesen, verwende die Funktion \py{input()}:
		
		\begin{minted}{python}
		age = input("How old are you?")
		print(f"I am {age} years old")
		\end{minted}
	\end{block}
	\pause 
	\begin{alertblock}{Achtung}
		\vspace{2pt}
		Das Ergebnis von \py{input} hat stets den Datentyp \py{string} auch wenn Zahlen eingelesen werden. Gegebenenfalls muss das Ergebnis mittels \py{int()} oder \py{float()} in den gewünschten Typ umgewandelt werden. 	
	\end{alertblock}
	
\end{fragile}


\begin{fragile}[]
	\begin{exampleblock}{Beispiel: Input und Output kombiniert}
		\begin{minted}{python}
		name = input("What is your name?")
		age = input("What is your age?")
		print(f"Hello {name}, you are {age} years old") 
		\end{minted}
	\end{exampleblock}
\end{fragile}

\begin{frame}{Übung}
\begin{block}{Adressabfrage}
\vspace{2pt}
Schreibe ein kurzes Skript, dass Dich nach Deinem Namen, Alter und Adresse fragt. Wenn es alles eingelesen hat, soll es diese Infos in folgender Form auf der Konsole ausgeben: 	

\texttt{Hallo Max, schön dass Du da bist. Du bist 21 Jahre alt und wohnst in der Bismarckstraße 12 in Glücksstadt.}
\end{block}
\end{frame}


\begin{frame}{Übung}
\begin{block}{Blick in die Zukunft}
	\vspace{2pt}
Schreibe ein kurzes Skript, dass Dich nach Deinem Alter fragt. Daraufhin soll es auf der Konsole ausgeben, wie alt Du in 15 Jahren sein wirst. 
\end{block}
\end{frame}

\section{Kommentare}

\begin{fragile}


\begin{block}{Kommentare}
\vspace{2pt}
Alle Zeichen einer Zeile, die hinter einem \texttt{\#} (Hashtag) kommen, werden von Python ignoriert.
So lassen sich Kommentare im Quellcode platzieren. 
\end{block}

\vspace{12pt}

\pause
\begin{exampleblock}{Beispiel}
\begin{minted}{python}
print("This line will be printed")
# print("This line won't") 
\end{minted}
\end{exampleblock}

\end{fragile}



\section{Conditionals \\ \footnotesize Ein Programm verzweigen}

\begin{frame}
	\begin{block}{Problemstellung}
		\vspace{2pt}
		Lies eine Zahl \py{x} ein. In Abhängigkeit von \py{x} soll Folgendes ausgegeben werden: 
		
		\texttt{Die Zahl x ist größer als 0} 
		
		bzw. 
		
		\texttt{Die Zahl x ist kleiner 0}  
		\vspace{8pt}
		
		
		Wie macht man das?
		\end{block}
\end{frame}

\begin{fragile}
	
\begin{block}{Lösung \footnotesize(fast)}
\begin{minted}{python}
x = input("Gib eine Zahl x an")
x = int(x)

if x < 0:
  print("x ist größer 0")
else:
  print("x ist kleiner 0")
\end{minted}
\end{block}
	
\end{fragile}


\begin{frame}
\metroset{block=fill}


	\renewcommand{\baselinestretch}{1.5}
\begin{block}{Struktur \texttt{if-else} Statement}	
	\vspace{2pt}
	\uncover<+->{
	\uncover<+->{\texttt{if}} \uncover<+->{\textit{Bedingung}}\uncover<+->{\texttt{:}}\\
	\uncover<+->{\spacechar\spacechar }\uncover<+->{\textit{Codezeile A1}}	\\
	\uncover<+->{\spacechar\spacechar \textit{Codezeile A2}	\\
	\spacechar\spacechar \phantom{Code}\vdots}\\
	\uncover<+->{\texttt{else:}}\\
	\uncover<+->{\spacechar\spacechar \textit{Codezeile B1}	\\
				 \spacechar\spacechar \textit{Codezeile B2}	\\
				 \spacechar\spacechar \phantom{Code}\vdots}\\
	\uncover<+->{\textit{Codezeile C1}\\
	\phantom{Code}\vdots
}
}
\end{block}
\renewcommand{\baselinestretch}{1}
%\vspace{10pt}

\end{frame}
\begin{frame}
\begin{block}{Wie funktioniert's?}
	\vspace{2pt}
Ist die \texttt{if}-Bedingung \py{True}, so wird der \texttt{if}-\textit{Block} ausgeführt. Ist sie \py{False} wird der \texttt{else}-\textit{Block} ausgeführt. 
\end{block}
\pause

\vspace{10pt}
\metroset{block=fill}
	\begin{block}{Definition: Block}
		\vspace{2pt}
		Aufeinanderfolgende Codezeilen, die alle die gleiche Einrückung besitzen, nennt man \emph{Block}. 
		D.h. Leerzeichen am Zeilenanfang haben in Python eine syntaktische Bedeutung.  
	\end{block}


\pause 
\vspace{10pt}

\metroset{block=transparent}
\begin{block}{Good to know}
	\begin{itemize}
		\item Der \pybw{else}-Block ist optional.
		\item Falls die Bedingung nicht vom Typ \py{bool} ist, so wird sie implizit umgewandelt.  
	\end{itemize}
\end{block}
\end{frame}

\begin{frame}{Übungen}

	\begin{block}{Volljährigkeit prüfen/Zutrittskontrolle}
		\vspace{2pt}
		Schreibe ein Skript, dass nach dem Alter eines Users fragt und überprüft, ob der User schon volljährig ist. Dementsprechend soll auf der Konsole entweder 
		
		\texttt{Willkommen}
		
		 oder
		 
		  \texttt{Du darfst hier nicht rein} 
		  
		  erscheinen.  
	\end{block}
\pause 
\vspace{12pt}
	\begin{block}{Teilbarkeit bestimmen}
		\vspace{2pt}
		Schreibe ein Skript, dass eine ganze Zahl einliest. Daraufhin soll auf der Konsole ausgegeben werden, ob die Zahl durch \pybw{7} teilbar ist. Beispiel: Ist die Eingabe 12, so ist die Ausgabe:   

		\texttt{Die Zahl 12 ist nicht durch 7 teilbar.}
	\end{block}

\end{frame}



\begin{frame}

\metroset{block=fill}
\uncover<+->{\begin{block}{Logische Operatoren}
	\vspace{2pt}
Booleans können mittels folgender Operatoren miteinander verknüpft werden: 
\uncover<+->{
\begin{description}
	\item[\pybw{and}] Ist genau dann \py{True}, wenn beide Operanden \py{True} sind.
	\item[\pybw{or}] Ist genau dann \py{True}, wenn mindestens ein Operand \py{True} ist.
	\item[\pybw{not}] Kehrt den nachfolgenden Wahrheitswert um.  
\end{description}
} 
\end{block}}
\vspace{10pt}
\metroset{block=transparent}
\uncover<+->{\begin{exampleblock}{Beispiel}
\begin{itemize}
	\item \pybw{2 > 0 and 3 > 4} ist \py{False}
	\item \pybw{1 > 0 or 6 > 1} ist \py{True}
	\item \pybw{not 2 < 1} ist \py{True}
\end{itemize}
\end{exampleblock}
}
\end{frame}


\begin{frame}{Übung}
\uncover<+->{
\begin{block}{Was ergeben die folgenden Ausdrücke?}
	\begin{itemize}
		\item \py{not 2 < 3 and 4 < 7}
		\item \py{4 not == 8}
		\item \py{3 != 4 and not 4 == 8}
		\item \py{7 <= 7.0 and not 7 != 7.0}
		\item \py{7 > 5 or 4 < 5 and not 9 > 6}
		\item \py{not 3 < 6 > 8}
		\item \py{not 3}
	\end{itemize}
\end{block}
}
\uncover<+->{
\begin{alertblock}{Präzedenz beachten!}
	\begin{enumerate}
		\item \pybw{==}, \pybw{!=}, \pybw{<=}, \pybw{<}, \pybw{>}, \pybw{>=}
		\item \pybw{not}
		\item \pybw{and}
		\item \pybw{or}
	\end{enumerate}
\end{alertblock}
}

\end{frame}

\begin{fragile}

\begin{block}{Das \pybw{elif}-Statement}
	\vspace{2pt}
Mit der reinen \pybw{if-else}-Syntax können nur \emph{binäre} Verzweigungen dargestellt werden. Um mehrer, gleichrangige Verzweigungsäste zu realisieren kann man das \pybw{elif}-Conditional verwenden. 
\end{block}
\pause
\begin{exampleblock}{Beispiel}
\begin{minted}{python}
if x < 0: 
    print("x is < 0")
elif x == 0: 
    print("x is 0")
elif x == 1: 
    print("x is 1")
else: 
    print("x is not negative but neither 0 nor 1")         
\end{minted}
\end{exampleblock}
\pause
Die Anzahl der \pybw{elif}-Blöcke ist beliebig. Der \pybw{else}-Block ist wie immer optional. 

\end{fragile}

\begin{fragile}{Übung}
	\begin{block}{Worin unterscheiden sich die beiden Abschnitte?}
		\vspace{5pt}
		Abschnitt 1: 
		\begin{minted}{python}
		if x % 2 == 0: 
		   # some Code here
		if x % 3 == 0: 
		   # some Code here
		else: 
		   # some Code here  
		\end{minted}
		Abschnitt 2: 
		\begin{minted}{python}
		if x % 2 == 0: 
		# some Code here
		elif x % 3 == 0: 
		# some Code here
		else: 
		# some Code here  
		\end{minted}
	\end{block}
\end{fragile}

\begin{frame}{Übung}
\begin{block}{Baue einen Bestätigungsdialog}
\vspace{2pt}
Schreibe ein Skript was einen typischen Bestätigungsdialog simuliert. 
Zum Beispiel: 

\texttt{Are you sure to continue? (y)es/(n)o}. 

Mögliche Antworten sind \texttt{yes}, \texttt{no} bzw. \texttt{y}, \texttt{n}. 
Daraufhin soll auf der Konsole \texttt{confirmed} oder \texttt{aborted} erscheinen. 
\end{block}
\end{frame}

\begin{frame}{Komplexere Übung}
%\begin{block}{Berechne Deinen Urlaubsort}
%\vspace{2pt}
%\end{block}
\begin{center}
\includegraphics[width=0.5\textwidth]{urlaubsort.png}
\end{center}
Lies eine Zahl zwischen 1 und 9 ein und gib auf der Konsole \emph{deinen nächsten Urlaubsort} aus. 
\end{frame}


\begin{fragile}{}
\begin{block}{Der \emph{Ternary Operator}}
\vspace{2pt}
Oftmals möchte man eine Variable in Abhängigkeit eines Wahrheitswertes definieren. Für diesen einfachen Fall, ist das \pybw{if-else}-Konstrukt sehr umständlich. Stattdessen kann man für die Kürze den \emph{ternary operator} verwenden. 
\end{block}
\vspace{12pt}
\pause
\begin{exampleblock}{Beispiel}
	\vspace{2pt}
	\begin{minted}{python}
	if x < 0: 
	  sign = "sign"	
	else: 
	  sign = "positive"
	\end{minted}
\end{exampleblock}
\pause 
\begin{block}{Stattdessen mit Ternary Operator}
	\vspace{2pt}
	\py{sign = "negative" if x < 0 else "positive"}
\end{block}
\end{fragile}

\begin{frame}{Übung}

\begin{block}{}
	\vspace{2pt}
Lies eine ganze Zahl ein und gib ihren Betrag auf der Konsole aus. Schaffst Du es, das Ganze mit weniger als 5 Zeilen Code zu programmieren? 
\end{block}

\end{frame}

\section{Die For-Schleife \\ \footnotesize Einen Programmabschnitt x-mal ausführen}

\begin{frame}

\begin{block}{Problemstellung}
\vspace{2pt}
Lies eine ganze Zahl \py{x} ein. Gib dann folgende Zeilen auf der Konsole aus 

\texttt{1}\\
\texttt{2}\\
\texttt{3}\\
\texttt{4}\\
\vdots \\
\texttt{x}

\vspace{12pt}
Wie macht man das? 

\end{block}
\end{frame}

\begin{fragile}{}
	\begin{block}{Lösung}
		\begin{minted}{python}
			x = input("Enter a number")
			
			for k in range(1, x + 1):	
			  print(k)
		\end{minted}
	\end{block}
\end{fragile}

\begin{frame}

	\renewcommand{\baselinestretch}{1.5}
	\metroset{block=fill}
	\begin{block}{Struktur der \texttt{for...in} Schleife}
		\vspace{2pt}
		\pause \py{for} \pause \textit{Variable} \pause \py{in} \pause \py{range}(\textit{min}, \textit{max})\pause\texttt{:} \pause \\
		\spacechar\spacechar Codezeile 1 \pause \\ 
\spacechar\spacechar Codezeile 2 \pause \\
\spacechar\spacechar \phantom{Code} \vdots \pause  \\
\textit{Code, der nicht mehr Teil der Schleife ist}
	\end{block}

\vspace{12pt}
\pause 

\metroset{block=transparent}
	\renewcommand{\baselinestretch}{1}
	\begin{block}{Wie funktioniert's?}
		\vspace{2pt}
	Die Schleifenvariable wird zunächst gleich dem unteren Wert in \py{range} gesetzt. Dann wird der \pybw{for}-Block wiederholt ausgeführt. Bei jedem Durchgang wird die Schleifenvariable um \pybw{1} vergrößert und zwar so lange, wie der Wert der Schleifenvariable kleiner als der obere Wert in \py{range} ist. 	
	\end{block}
\end{frame}

\begin{frame}
\begin{block}{Good to know}
	\pause
	\begin{itemize}[<+->]
		\item Achtung: Die Schleifenvariable erreicht nie das obere Ende der \py{range}-Funktion, sondern bleibt immer \pybw{1} drunter. 
		\item Die \py{range}-Funktion ist nicht auf 1er-Schrittweite beschränkt. Mit folgendem Ausdruck werden die Zahlen von \py{0} bis \py{9} z.B. in 3er-Schritten durchlaufen: \py{range(0, 10, 3)}. 
		\item \texttt{For}-Schleifen sind flexibel und können alles mögliche durchlaufen, z.B. auch die einzelnen Buchstaben eines Strings (dazu später mehr).
	\end{itemize}
\end{block}
\end{frame}

\begin{frame}{Übung}

\begin{block}{Einmaleins: Die 7er-Reihe}
	\vspace{2pt}
Schreibe ein kleines Skript, was die 7er-Reihe (bis 70) auf der Konsole ausgibt.	
\end{block}

\vspace{12pt}

\pause 

\begin{block}{7er-Reihe mit beliebigem oberen Ende}
	\vspace{2pt}
	Lies eine positive ganze Zahl \py{x} ein, gib die 7er-Reihe von \py{7} bis \py{x} auf der Konsole aus.  
\end{block}

\vspace{12pt}

\pause 

\begin{block}{7er-Reihe mit ganzen Sätzen}
	\vspace{2pt}
Lies wie eben eine obere Grenze für die 7-er Reihe ein. Gib dann die 7-er Reihe wie folgt auf der Konsole aus:

\texttt{1 mal 7 ist 7}\\	
\texttt{2 mal 7 ist 14}\\
\phantom{4 mal} \vdots

\end{block}
\end{frame}


\begin{frame}{Schwierigere Übungen}

\begin{block}{Das Gauss-Problem}
\vspace{2pt}	
Berechne die Summe der Zahlen 1 bis 100. 
\end{block}
\vspace{12pt}
\pause
\begin{block}{Zahlenmuster}
\vspace{2pt}	
Gib folgendes Muster auf der Konsole aus: 

\texttt{1} \\
\texttt{1 2} \\
\texttt{1 2 3} \\
\texttt{1 2 3 4} \\
\phantom{1 2 } \vdots \\
\texttt{1 2 $\cdots$ 20}
\end{block}

\end{frame}

\begin{frame}{Harte Übungen}

\begin{block}{Quersumme}
	\vspace{2pt}
	Lies eine ganze Zahl \py{x} ein und bestimme ihre Quersumme. 
	
	\textbf{Tipp 1:} Die Anzahl der Stellen einer Zahl bekommt man mittels \py{len(str(x))} heraus. \\
	\textbf{Tipp 2:} Man benötigt Tipp 1 gar nicht.  
	
\end{block}

\vspace{12pt}
\pause

\begin{block}{Primzahltest}
	\vspace{2pt}
	Lies eine ganze Zahl \py{x} ein und überprüfe, ob diese Zahl eine Primzahl ist. Das Programm soll etwa folgende Ausgabe liefern 
	
	\texttt{Die Zahl 28061983 ist eine Primzahl.}
\end{block}

\end{frame}


\section{Die While-Schleife \\ \footnotesize Wie die For-Schleife nur abstrakter und open-end}

\begin{frame}
\begin{block}{Problemstellung}
	\vspace{2pt}
	Lies immer wieder eine Zahl von der Konsole ein. Höre auf, wenn diese Zahl 7 ist. 
	
	Wie macht man das? 
\end{block}
\end{frame}

\begin{fragile}
	
\begin{block}{Lösung}
		\vspace{2pt}
		
	\begin{minted}{python}
		x = 0
		
		while x != 7: 
		  x = input("Enter a number")
		  x = int(x)
		  
		print("Yeah, you picked the right number.")
	\end{minted}
	
\end{block}
\end{fragile}


\begin{frame}

\renewcommand{\baselinestretch}{1.5}
\metroset{block=fill}
\begin{block}{Struktur der \texttt{While}-Schleife}
	\vspace{2pt}
	\pause \py{while} \pause \textit{Bedingung}\pause\texttt{:} \pause \\
	\spacechar\spacechar Codezeile 1 \pause \\ 
	\spacechar\spacechar Codezeile 2 \pause \\
	\spacechar\spacechar \phantom{Code} \vdots \pause  \\
	\textit{Code, der nicht mehr Teil der Schleife ist}
\end{block}
\vspace{12pt}
\pause 
\metroset{block=transparent}
\renewcommand{\baselinestretch}{1}
\begin{block}{Wie funktioniert's?}
	\vspace{2pt}
	Die Schleife wird solange ausgeführt, wie die \emph{Bedingung} \py{True} ergibt. Nach jedem Durchgang wird der Ausdruck der \emph{Bedingung} neu ausgewertet. 
	Ist die Bedingung \py{False} wird der Code unterhalb des Schleifenblocks ausgeführt. 
\end{block}

\end{frame}

\begin{frame}
\begin{alertblock}{Achtung Endlosschleife}
	\vspace{2pt}
	Man sollte immer darauf achten, dass die Bedingung in der \pybw{while}-Schleife auch wirklich irgendwannmal \py{False} wird. Ansonsten bleibt das Programm in einer \emph{Endlosschleife} gefangen. 
\end{alertblock}
\end{frame}

\begin{frame}{Übung}

\begin{block}{Ersetze eine \pybw{for}-Schleife durch eine \pybw{while}-Schleife}
\vspace{2pt}
Schreib ein Programm, dass alle 7er-Zahlen von 7 bis 700 auf der Konsole ausgibt.  
\end{block}

\pause 
\vspace{12pt}

\begin{block}{Bestätigungsdialog \emph{deluxe}}
\vspace{2pt}
Verbessere den Bestätigungsdialog:  

\texttt{Are you sure to continue? (y)es/(n)o}. 

Mögliche Antworten sind \texttt{yes}, \texttt{no} bzw. \texttt{y}, \texttt{n}. 

Daraufhin soll auf der Konsole \texttt{continued} oder \texttt{aborted} erscheinen. 
Falls die Eingabe nicht klar erkannt wird, soll die Frage nochmal auf der Konsole gestellt werden. 

Zum Beispiel: 
\texttt{Please enter either "yes"{} or "no"}  
\end{block}

\end{frame}

\begin{frame}{Übung}

\begin{block}{Notenrechner}
\vspace{2pt}
Schreib ein Programm, dass wiederholt nach einer Note von Dir fragt und Dir dann jeweils die aktuelle Durchschnittsnote auf der Konsole ausgibt. 
Das Programm soll durch die Eingabe vom Buchstaben \textbf{q} beendet werden können. 

Beispielausgabe: 

\texttt{Bitte gib eine Note oder q zum Beenden ein:} \py{1} \\
\texttt{Deine Durchschnittsnote ist 1.0} \\
\texttt{Bitte gib eine Note oder q zum Beenden ein:} \py{2} \\
\texttt{Deine Durschnittsnote ist 1.5} \\
\phantom{Code} \vdots 
	
\end{block}

\end{frame}

\begin{frame}{Übung}

\begin{block}{Ratespiel}
\vspace{2pt}
Definiere eine positive ganze Zahl \pybw{number_to_guess}. Der User kann nun wiederholt eine Zahl eingeben. Das Spiel endet, wenn die eingegebene Zahl mit \pybw{number_to_guess} übereinstimmt. 
Andernfalls wird auf der Konsole beispielsweise ausgegeben: 

\texttt{Sorry, Deine eingegebene Zahl war zu klein, versuche es nochmal: }

\pause
\textbf{Zusatz 1:} \\
Am Ende soll die Anzahl der Versuche angegeben werden.

\pause
\textbf{Zusatz 2:} \\
Das Spiel soll mit der Eingabe von q abgebrochen werden können. 

\pause
\textbf{Zusatz 3:} \\
Google, wie Python die Zahl \pybw{number_to_guess} zufällig erzeugen kann (das verbessert das Gameplay).  


\end{block}

\end{frame}
%\section{Die For-Schleife \\ \footnotesize Einen Programmabschnitt x-mal ausführen}

\begin{frame}

\begin{block}{Problemstellung}
\vspace{2pt}
Lies eine ganze Zahl \py{x} ein. Gib dann folgende Zeilen auf der Konsole aus 

\texttt{1}\\
\texttt{2}\\
\texttt{3}\\
\texttt{4}\\
\vdots \\
\texttt{x}

\vspace{12pt}
Wie macht man das? 

\end{block}
\end{frame}

\begin{fragile}{}
	\begin{block}{Lösung \footnotesize (fast)}
		\begin{minted}{python}
			x = input("Enter a number")
			x = int(x)
			
			for k in range(1, x):	
			  print(k)
		\end{minted}
	\end{block}
\end{fragile}

\begin{frame}

	\renewcommand{\baselinestretch}{1.5}
	\metroset{block=fill}
	\begin{block}{Struktur der \texttt{for...in} Schleife}
		\vspace{2pt}
		\pause \py{for} \pause \textit{Variable} \pause \py{in} \pause \py{range}(\textit{min}, \textit{max})\pause\texttt{:} \pause \\
		\spacechar\spacechar Codezeile 1 \pause \\ 
\spacechar\spacechar Codezeile 2 \pause \\
\spacechar\spacechar \phantom{Code} \vdots \pause  \\
\textit{Code, der nicht mehr Teil der Schleife ist}
	\end{block}

\vspace{12pt}
\pause 

\metroset{block=transparent}
	\renewcommand{\baselinestretch}{1}
	\begin{block}{Wie funktioniert's?}
		\vspace{2pt}
	Die Schleifenvariable wird zunächst gleich dem unteren Wert in \py{range} gesetzt. Dann wird der \pybw{for}-Block wiederholt ausgeführt. Bei jedem Durchgang wird die Schleifenvariable um \pybw{1} vergrößert und zwar so lange, wie der Wert der Schleifenvariable kleiner als der obere Wert in \py{range} ist. 	
	\end{block}
\end{frame}

\begin{frame}
\begin{block}{Good to know}
	\pause
	\begin{itemize}[<+->]
		\item Achtung: Die Schleifenvariable erreicht nie das obere Ende der \py{range}-Funktion, sondern bleibt immer \pybw{1} drunter. 
		\item Die \py{range}-Funktion ist nicht auf 1er-Schrittweite beschränkt. Mit folgendem Ausdruck werden die Zahlen von \py{0} bis \py{9} z.B. in 3er-Schritten durchlaufen: \py{range(0, 10, 3)}. 
		\item \texttt{For}-Schleifen sind flexibel und können alles mögliche durchlaufen, z.B. auch die einzelnen Buchstaben eines Strings (dazu später mehr).
	\end{itemize}
\end{block}
\end{frame}

\begin{fragile}[Übung]

\begin{block}{Einmaleins: Die 7er-Reihe}
	\vspace{2pt}
Schreibe ein kleines Skript, was die 7er-Reihe (bis 70) wie folgt auf der Konsole ausgibt: 

\texttt{1 mal 7 ist 7}\\	
\texttt{2 mal 7 ist 14}\\
\phantom{4 mal} \vdots  
\end{block}

\vspace{12pt}

\begin{solutionblock}{Lösung}
\begin{minted}{python}
for k in range(1, 10 + 1):
  print(f"{k} mal 7 ist {7 * k}")
\end{minted}
\end{solutionblock}

\end{fragile}



\begin{fragile}[Übung]


\begin{block}{7er-Reihe mit beliebigem oberen Ende}
\vspace{2pt}
Lies eine positive ganze Zahl \py{x} ein und gib die 7er-Reihe von \py{7} bis mindestens \py{x} wie oben auf der Konsole aus.
\end{block}

\vspace{12pt}
\begin{solutionblock}{Lösung}
\begin{minted}{python}
limit = input("Bis wie weit soll die 7-er Reihe gezählt werden? ")
limit = int(limit)

limit = limit // 7 + 1
# man addiert 1, da man nach der Division aufrunden möchte.
# Ist limit eine 7er-Zahl zählt man halt ein bisschen zu weit

for k in range(1, limit + 1):
  print(f"{k} mal 7 ist {7 * k}")
\end{minted}
\end{solutionblock}


\end{fragile}

\begin{fragile}[Übung]
\begin{block}{Schleife über einen String}
\vspace{2pt}
Lies Deinen Namen (oder irgendein Wort) auf der Konsole ein und überprüfe, ob er den Buchstaben \emph{a} (groß/klein) enthält. 
\end{block}
\vspace{12pt}
\begin{solutionblock}{Lösung}
\begin{minted}{python}
name = input("Gib ein Wort ein: ")

# Flag (Schalter) initialisieren
name_contains_letter_a = False

for letter in name:
  if letter == "a" or letter == "A":
    name_contains_letter_a = True

if name_contains_letter_a:
  print("Der Name enthält ein 'a'.")
else:
  print("Der Name enthält kein 'a'.")
\end{minted}
\end{solutionblock}


\end{fragile}




\begin{fragile}[Schwierigere Übungen]

\begin{block}{Das Gauss-Problem}
\vspace{2pt}	
Berechne die Summe der Zahlen 1 bis 100. 
\end{block}
\vspace{12pt}
\begin{solutionblock}{Lösung}
\begin{minted}{python}
result = 0
for k in range(1, 100 + 1):
  result = result + k
print(f"Das Ergebnis ist {result}.")
\end{minted}
\end{solutionblock}



\end{fragile}

\begin{fragile}[Harte Übung I]

\begin{block}{Quersumme}
	\vspace{2pt}
	Lies eine ganze Zahl \py{x} ein und bestimme ihre Quersumme. 
	
	\textbf{Tipp 1:} Die Anzahl der Stellen einer Zahl bekommt man mittels \py{len(str(x))} heraus. \\
	\textbf{Tipp 2:} Man benötigt Tipp 1 gar nicht.  
	
\end{block}

\vspace{12pt}

\begin{solutionblock}{Lösung}
\begin{minted}{python}
number = input("Gib eine Zahl ein: ")
result = 0
# Wir lassen die Zahl als String, damit wir eine Schleife über die Ziffern legen können
for digit in number:
  result = result + int(digit)  # Achtung: digit ist ja eigentlich ein String

print(f"Die Quersumme von {number} ist {result}")
\end{minted}
\end{solutionblock}

\end{fragile}

\begin{fragile}[Harte Übung II]
\begin{block}{Zahlenmuster}
	\vspace{2pt}	
	Gib folgendes Muster auf der Konsole aus: 
	
	\texttt{1} \\
	\texttt{1 2} \\
	\texttt{1 2 3} \\
	\texttt{1 2 3 4} \\
	\phantom{1 2 } \vdots \\
	\texttt{1 2 $\cdots$ 20}
\end{block}

\begin{solutionblock}{Lösung}
\begin{minted}{python}
for row in range(1, 20 + 1):
  row_string = ""  # Hier wird das Ergebnis pro Zeile initialisiert
  for column in range(1, row +1):
    row_string = row_string + str(column) + " "
  print(row_string)
\end{minted}
\end{solutionblock}

\end{fragile}

\begin{fragile}[Brutale Übung]
	
	
	\begin{block}{Fibonacci-Zahlen}
		\vspace{2pt}
		Die Zahlenfolge $1,1,2,3,5,8,13\ldots$ nennt man \emph{Fibonacci}-Folge. Dabei ensteht ein Element der Folge, durch die Addition des vorherigen und vorvorherigen Elements. 
		
		\vspace{1pt}
		
		Berechne die 30. Fibonacci-Zahl.  
	\end{block}
	\vspace{12pt}
	\begin{solutionblock}{Lösung}
		\begin{minted}{python}
		last = 1  # letzte Zahl
		current = 1  # aktuelle Zahl
		
		for k in range(2, 30 + 1):
		old_current = current  # Zahl zwischenspeichern
		current = current + last
		last = old_current
		print(f"Die {k}-te Fibonacci-Zahl ist {current}")
		\end{minted}
	\end{solutionblock}
	
	
\end{fragile}


\section{Die While-Schleife \\ \footnotesize Wie die For-Schleife nur abstrakter und open-end}

\begin{frame}
\begin{block}{Problemstellung}
	\vspace{2pt}
	Lies immer wieder eine Zahl von der Konsole ein. Höre auf, wenn diese Zahl 7 ist. 
	
	Wie macht man das? 
\end{block}
\end{frame}

\begin{fragile}
	
\begin{block}{Lösung}
		\vspace{2pt}
		
	\begin{minted}{python}
		x = 0
		
		while x != 7: 
		  x = input("Enter a number")
		  x = int(x)
		  
		print("Yeah, you picked the right number.")
	\end{minted}
	
\end{block}
\end{fragile}


\begin{frame}

\renewcommand{\baselinestretch}{1.5}
\metroset{block=fill}
\begin{block}{Struktur der \texttt{while}-Schleife}
	\vspace{2pt}
	\pause \py{while} \pause \textit{Bedingung}\pause\texttt{:} \pause \\
	\spacechar\spacechar Codezeile 1 \pause \\ 
	\spacechar\spacechar Codezeile 2 \pause \\
	\spacechar\spacechar \phantom{Code} \vdots \pause  \\
	\textit{Code, der nicht mehr Teil der Schleife ist}
\end{block}
\vspace{12pt}
\pause 
\metroset{block=transparent}
\renewcommand{\baselinestretch}{1}
\begin{block}{Wie funktioniert's?}
	\vspace{2pt}
	Die Schleife wird solange ausgeführt, solange die \emph{Bedingung} \py{True} ergibt. Nach jedem Durchgang wird der Ausdruck der \emph{Bedingung} neu ausgewertet. 
	Ist die Bedingung \py{False} wird der Code unterhalb des Schleifenblocks ausgeführt. 
\end{block}

\end{frame}

\begin{frame}
\begin{alertblock}{Achtung Endlosschleife}
	\vspace{2pt}
	Man sollte immer darauf achten, dass die Bedingung in der \pybw{while}-Schleife auch wirklich irgendwannmal \py{False} wird. Ansonsten bleibt das Programm in einer \emph{Endlosschleife} gefangen. 
\end{alertblock}
\end{frame}


\begin{fragile}{Übung}

\begin{block}{Ersetze eine \pybw{for}-Schleife durch eine \pybw{while}-Schleife}
\vspace{2pt}
Schreib ein Programm, dass alle 7er-Zahlen von 7 bis 700 auf der Konsole ausgibt. 

\begin{solutionblock}{Lösung}
\begin{minted}{python}
k = 7
while k <= 70:
  print(k)
  k = k + 7
\end{minted}
\end{solutionblock}
 
\end{block}

\end{fragile}




%\begin{frame}{Übung}
%
%\begin{block}{Notenrechner}
%\vspace{2pt}
%Schreib ein Programm, dass wiederholt nach einer Note von Dir fragt und Dir dann jeweils die aktuelle Durchschnittsnote auf der Konsole ausgibt. 
%Das Programm soll durch die Eingabe vom Buchstaben \textbf{q} beendet werden können. 
%
%Beispielausgabe: 
%
%\texttt{Bitte gib eine Note oder q zum Beenden ein:} \py{1} \\
%\texttt{Deine Durchschnittsnote ist 1.0} \\
%\texttt{Bitte gib eine Note oder q zum Beenden ein:} \py{2} \\
%\texttt{Deine Durschnittsnote ist 1.5} \\
%\phantom{Code} \vdots 
%	
%\end{block}
%
%\end{frame}

\begin{frame}{Übung}

\begin{block}{Ratespiel}
\vspace{2pt}
Definiere eine positive ganze Zahl \pybw{number_to_guess}. Der User kann nun wiederholt eine Zahl eingeben. Das Spiel endet, wenn die eingegebene Zahl mit \pybw{number_to_guess} übereinstimmt. 
Andernfalls wird auf der Konsole beispielsweise ausgegeben: 

\texttt{Sorry, Deine eingegebene Zahl war zu klein, versuche es nochmal: }

\pause
\textbf{Zusatz 1:} \\
Am Ende soll die Anzahl der Versuche angegeben werden.

\pause
\textbf{Zusatz 2:} \\
Das Spiel soll mit der Eingabe von \pybw{q} abgebrochen werden können. 

\pause
\textbf{Zusatz 3:} \\
Google, wie Python die Zahl \pybw{number_to_guess} zufällig erzeugen kann (das verbessert das Gameplay).  

\end{block}
\end{frame}



\begin{frame}<beamer:0>[fragile]
\frametitle{Lösung}
\begin{solutionblock}{Ratespiel ohne Zusätze}
\begin{minted}{python}
number_to_guess = 512
guess = input("Rate meine Zahl: ")
guess = int(guess)

while guess != number_to_guess:
  if guess < number_to_guess:
    print("Deine Zahl war zu klein")
  else:
    print("Deine Zahl war zu groß")
  guess = input("Versuch's nochmal: ")
  guess = int(guess)

print("Du hast gewonnen")
\end{minted}
\end{solutionblock}
\end{frame}


\begin{frame}<beamer:0>[fragile]
\frametitle{Lösung}
\begin{solutionblock}{Ratespiel mit Zusatz 1}
\begin{minted}{python}
number_to_guess = 512
guess = input("Rate meine Zahl: ")
guess = int(guess)
counter = 1

while guess != number_to_guess:
  if guess < number_to_guess:
    print("Deine Zahl war zu klein")
  else:
    print("Deine Zahl war zu groß")
  guess = input("Versuch's nochmal: ")
  guess = int(guess)
  counter = counter + 1
  
print(f"Du hast nach {counter} Versuchen gewonnen")	
\end{minted}
\end{solutionblock}
\end{frame}


\begin{frame}<beamer:0>[fragile]
\frametitle{Lösung}
\begin{solutionblock}{Ratespiel mit Zusatz 2}
\begin{minted}{python}
number_to_guess = 512
counter = 1
guess = input("Rate meine Zahl: ")
guess = int(guess)
is_quit = False

while guess != number_to_guess and not is_quit:
  if guess < number_to_guess:
    print("Deine Zahl war zu klein")
  else:
    print("Deine Zahl war zu groß")
  guess = input("Versuch's nochmal: ")
  if guess == "q":
    is_quit = True
  else: 
    guess = int(guess)
  counter = counter + 1

if number_to_guess == guess: 
  print(f"Du hast nach {counter} Versuchen gewonnen")	
\end{minted}
\end{solutionblock}
\end{frame}




\begin{frame}<beamer:0>[fragile]
\frametitle{Lösung}
\begin{solutionblock}{Ratespiel mit Zusatz 3}
\begin{minted}{python}
import random 

number_to_guess = random.randint(1, 1000)
guess = input("Rate meine Zahl: ")
guess = int(guess)
# ... usw. 
\end{minted}
\end{solutionblock}
\end{frame}







	





%\section{\texttt{break}, \texttt{continue} und \texttt{else} \\ \footnotesize Den Fluss einer Schleife kontrollieren}


\begin{fragile}
	
\metroset{block=fill}

\begin{block}{Das \texttt{break}-Statement}
Taucht innerhalb einer Schleife das Schlüsselwort \py{break} auf, so wird die weitere Abarbeitung der Schleife abgebrochen. Die Ausführung wird mit dem Code \emph{nach} dem Schleifenblock ausgeführt. 		
\end{block}

\metroset{block=transparent}

\vspace{12pt} \pause 


\begin{exampleblock}{Beispiel}
\vspace{2pt}

\begin{overprint}
	\onslide<2|handout:0>
\begin{minted}{python}
for k in range(1,100):
  print(k)
  if k > 3:
    break
\end{minted}
\onslide<3|handout:1>
\begin{minted}{python}
for k in range(1,100):
  print(k)
  if k > 3:
    break
# prints 1 2 3 4 
\end{minted}
\end{overprint}

\end{exampleblock}

	
\end{fragile}



\begin{fragile}
	
\metroset{block=fill}

\begin{block}{Das \texttt{continue}-Statement}
Taucht innerhalb einer Schleife das Schlüsselwort \py{continue} auf, so wird der aktuelle Schleifendurchgang abgebrochen. Die Ausführung wird mit der nächsten Schleifeniteration fortgesetzt. 
\end{block}

\metroset{block=transparent}

\vspace{12pt} \pause 


\begin{exampleblock}{Beispiel}
\vspace{2pt}

\begin{overprint}
\onslide<2|handout:0>
\begin{minted}{python}
for k in range(1,11):
  if k % 2 == 0:
    continue
  print(k)
\end{minted}
\onslide<3|handout:1>
\begin{minted}{python}
for k in range(1,11):
  if k % 2 == 0:
    continue
  print(k)
# prints 1 3 5 7 9 
\end{minted}
\end{overprint}
\end{exampleblock}
	
	
\end{fragile}


\begin{fragile}
	
\metroset{block=fill}

\begin{block}{Der \texttt{else}-Block einer Schleife}
Analog zum \py{if}-Statement, kann auch eine Schleife einen \py{else}-Block haben. Dieser wird ausgeführt, wenn die Schleife \emph{regulär} (also nicht durch die Verwendung von \py{break}) beendet wird.  
\end{block}

\metroset{block=transparent}

\vspace{12pt} \pause 


\begin{exampleblock}{Beispiel}
\begin{minted}{python}
name = input("Your name: ")

for letter in name: 
  if letter == "a":
    print("Your name contains an a")
    break
else: 
  print("Your name contains no a")
\end{minted}
\end{exampleblock}
	
	
\end{fragile}



\begin{fragile}[Übungen]

\begin{block}{Zählen bis zur nächsten 10er-Zahl}
	\vspace{2pt}
Lies eine Zahl \pybw{x} ein und gib auf der Konsole die Zahlen von \pybw{x} bis zur nächsten 10er-Zahl aus. 
\\
Ist die Eingabe \pybw{x = 17}, so soll die Ausgabe wie folgt aussehen: 
\begin{verbatim}
17
18
19
20
\end{verbatim}
\end{block}
	
\vspace{12pt}
\pause 

\begin{block}{Zählen mit Lücken}
	\vspace{2pt}
	Schreibe ein Skript, dass die Zahlen von 1 bis 99 aufzählt, dabei allerdings die 10er-Zahlen weglässt. Versuche dabei, ein \pybw{continue}-Statement zu verwenden.
\end{block}
\end{fragile}

\begin{frame}<beamer:0>[fragile]{Lösungen}

\begin{solutionblock}{Zählen bis zur nächsten 10er-Zahl}
\begin{minted}{python}
x = input("Gib eine Zahl an: ")
x = int(x)

for k in range(x, x + 11):
  print(k)
  if k % 10 == 0:
    break
\end{minted}
\end{solutionblock}

\vspace{12pt}

\begin{solutionblock}{Zählen mit Lücken}
\begin{minted}{python}
for k in range(1, 100):
  if k % 10 == 0:
    continue
  print(k)
\end{minted}
\end{solutionblock}

\end{frame}



\begin{fragile}[Harte Übung]
\begin{block}{Primzahltest}
\vspace{2pt}
Lies eine ganze Zahl \py{x} ein und überprüfe, ob diese Zahl eine Primzahl ist. Die Ausgabe des Programms soll etwa wie folgt aussehen:  

\texttt{Die Zahl 28061983 ist eine Primzahl.}
\end{block}

\vspace{12pt}
\begin{solutionblock}{Lösung}
\begin{minted}{python}
x = input("Gib eine Zahl ein: ")
x = int(x)

for k in range(2, x):
  if x % k == 0:
    print(f"{x} ist keine Primzahl.")
    break
else:
  print(f"{x} ist eine Primzahl.")
\end{minted}
\end{solutionblock}

\end{fragile}





%\begin{frame}{Harte Übung}
%\begin{block}{Finde die nächste Primzahl}
%\vspace{2pt}
%Lies eine ganze Zahl \py{x} ein und finde die nächste Zahl größer \py{x}, die gleichzeitig eine Primzahl ist. 
%\end{block}
%\end{frame}


\section{Listen \\ \footnotesize Viele Variablen gleichzeitig speichern}


\begin{frame}
\begin{block}{Problemstellung}
\vspace{2pt}
Lies mit Hilfe einer Schleife nach und nach Schulnoten von Dir ein. 
Alle Noten sollen gespeichert werden. Danach sollst Du die Wahl haben, die soundsovielte Note anzeigen lassen zu können.   

\vspace{8pt}

Wie macht man das? 
\end{block}
\end{frame}

\begin{fragile}{}
\begin{block}{Lösung \footnotesize(fast)}
\begin{minted}{python}
# ...
# Um das Eingeben der Noten kümmern wir uns noch
grades = [12, 10, 7, 14, 13, 13, 6, 4, 15, 14] # Noten in Notenpunkten

index = input("Die wievielte Note möchtest Du nochmal anschauen?")
index = int(index)

print(f"Deine { index }. Note ist { grades[index] } Punkte")
\end{minted}
\end{block}
\end{fragile}


\begin{fragile}

\metroset{block=fill}
\begin{block}{Struktur einer \emph{Liste}}
\vspace{2pt}
\large
\texttt{my\_list = }\pause {\Large\texttt{[}}\pause 
\texttt{element\_0}\pause,
\pause 
\texttt{element\_1}, \pause 
 \dots   
, \texttt{element\_n}\pause \Large{\texttt{]}}
\end{block}

\pause 

Die Variable \py{my_list} trägt nicht nur einen Wert, sondern $n+1$ Werte. Ansonsten verhält sich \py{my_list} wie eine ganz \enquote{normale} Variable. 
Als Einträge einer Liste sind beliebige Werte mit beliebigen Datentypen zugelassen. 


\vspace{12pt}

\pause

\textbf{Frage:} Welchen Datentyp hat die Liste \py{[2, 2.3, "Hello"]} ? 
	
\end{fragile}

\begin{frame}
	
\begin{block}{Auf Listenelemente zugreifen}
	
\vspace{2pt}

Auf das \pybw{n}-te Element der Liste \py{my_list} kann man mittels \py{my_list[n]} zugreifen. 

\pause 

Mit \py{my_list[-1]}, \py{my_list[-2]}, etc. kann man auf das letzte, vorletzte, etc. Element 
der Liste zugreifen. 

\end{block}

\pause 
\vspace{12pt}

\begin{alertblock}{Achtung}
\vspace{2pt}
Python fängt bei 0 an zu zählen. D.h. das erste Element in der Liste hat den Index 0. \\
Beispiel: \py{my_list[1]} liefert das \textbf{2. Element} der Liste. 
\end{alertblock}

	
\end{frame}	


\begin{frame}
\begin{block}{Schreibzugriff auf Listenelemente}
\vspace{2pt}
Nach dem gleichen Prinzip lassen sich einzelne Listeneinträge verändern. \\
Beispiel: \py{my_list[3] = -23}. 
\end{block}

\pause 
\vspace{12pt}



\begin{alertblock}{Achtung}
\vspace{2pt}
Man kann nur schon existierende Listeneinträge verändern. 
\end{alertblock}

\pause 
\vspace{12pt}



\begin{exampleblock}{Neues Konzept}
\vspace{2pt}
Listen sind der erste Datentyp, den wir kennenlernen, der \emph{mutable} (veränderbar) ist. Die bisherigen Datentypen waren \emph{immutable}, d.h. man konnte sie zwar überschreiben, aber nicht verändern. 
\end{exampleblock}

\end{frame}

\begin{frame}
	
\begin{block}{Listeneinträge hinzufügen}
	\vspace{2pt}
	Mit der \emph{Methode} \pybw{.append()} kann ein Eintrag zur Liste hinzugefügt werden. \\ 
	Bsp: \py{my_list.append(12)} fügt einen weiteren Eintrag mit Wert \pybw{12} hinzu. 
\end{block}	

\pause 
\vspace{12pt}


\begin{block}{Listeneinträge entfernen}
	\pause 
\vspace{2pt}
Mit dem Keyword \pybw{del} kann man Einträge an einer bestimmten Position löschen. Dabei verschieben sich die darauffolgenden Einträge um \pybw{1} nach vorne. \\
Beispiel: \py{del my_list[2]} löscht das dritte Element.  

\pause 

Mit der Methode \pybw{.remove()} kann man Einträge mit einem bestimmten Wert löschen. \\
Beispiel: \py{my_list.remove(-23)} entfernt den ersten Eintrag mit dem Wert \pybw{-23}. Ist der Wert nicht vorhanden gibt es eine Fehlermeldung. 
\end{block}
\end{frame}

\begin{fragile}[Übung]
\begin{block}{Eine Liste erstellen}
\vspace{2pt}
Schreibe ein kleines Programm, dass Dich ca. 4x nach dem Namen einer Freund*in fragt und Dir am Schluss die Liste der eingegebenen Freund*innen ausgibt. 	
\end{block}
\vspace{12pt}
\begin{solutionblock}{Lösung}
\begin{minted}{python}
friends = []
for k in range(1, 5):
  friend = input("Wie heißt ein*e Freund*in von Dir? ")
  friends.append(friend)
 print(friends)
\end{minted}
\end{solutionblock}
\end{fragile}





\begin{fragile}[Übung]
\begin{block}{Das Eingangsproblem}
\vspace{2pt}
Schreibe ein kleines Programm, dass solange Deine Noten einliest, bis Du \textbf{q} drückst. Danach sollst Du die Möglichkeit haben, eine Zahl \pybw{k} einzugeben, so dass Dir die \pybw{k}-te Note angezeigt wird. 
\end{block}	
\end{fragile}

\begin{frame}<beamer:0>[fragile]{Lösung}
\begin{solutionblock}{Das Eingangsproblem}
\begin{minted}{python}
grades = []
while True:
  grade = input("Gib eine Note an: ")
  if grade == "q":
    break
  grades.append(int(grade))

index = input("Die wievielte Note möchtest Du nochmal anschauen?")
index = int(index)
print(f"Deine { index }. Note ist { grades[index-1] } Punkte")
\end{minted}
\end{solutionblock}
\end{frame}

\begin{fragile}
\begin{block}{Schleife über Liste}
\vspace{2pt}
Analog wie über Strings und Ranges kann man Schleifen auch über eine Liste laufen lassen.  
\end{block}
\vspace{12pt}
\pause 

\begin{exampleblock}{Beispiel}
\vspace{2pt}
\begin{overprint}
\onslide<2|handout:0>
\begin{minted}{python}
my_hobbies = ["Segeln", "Tennis", "Schwimmen", "Lesen"]

for hobby in my_hobbies:
  print(hobby)
\end{minted}
\onslide<3|handout:1>
\begin{minted}{python}
my_hobbies = ["Segeln", "Tennis", "Schwimmen", "Lesen"]

for hobby in my_hobbies:
  print(hobby)
  
# prints:
# Segeln
# Tennis
# Schwimmen
# Lesen
\end{minted}
\end{overprint}
\end{exampleblock}
\end{fragile}


\begin{fragile}
\begin{block}{Schleife über Liste mit Indizes}
\vspace{2pt}
Möchte man in einer Schleife nicht nur die Listeneinträge, sondern auch die Indizes verwenden, so muss man die Funktion \py{enumerate()} auf die Liste anwenden. 
\end{block}
\vspace{12pt}
\pause 

\begin{exampleblock}{Beispiel}
\vspace{2pt}
\begin{overprint}
\onslide<2|handout:0>
\begin{minted}{python}
my_hobbies = ["Segeln", "Tennis", "Schwimmen", "Lesen"]

for index, hobby in enumerate(my_hobbies):
  print(f"Mein {index + 1}.Hobby ist {hobby}")
\end{minted}
\onslide<3|handout:1>
\begin{minted}{python}
my_hobbies = ["Segeln", "Tennis", "Schwimmen", "Lesen"]

for index, hobby in enumerate(my_hobbies):
  print(f"Mein {index + 1}.Hobby ist {hobby}")
  
# prints:
# Mein 1. Hobby ist Segeln
# Mein 2. Hobby ist Tennis
# Mein 3. Hobby ist Schwimmen
# Mein 4. Hobby ist Lesen
\end{minted}
\end{overprint}
\end{exampleblock}
\end{fragile}


\begin{frame}{Übung}

\begin{block}{Liste durchsuchen}
	\vspace{2pt}
Lies wieder eine Liste Deiner Noten ein. Prüfe, ob Du mindestens einmal unterpunktet hast (d.h. 0 Punkte hattest). 
Auf der Konsole soll dann entweder 

\pybw{Du hast irgendwo unterpunktet} 

oder 

\pybw{Du hast nirgendwo unterpunktet} 

ausgegeben werden. 
\end{block}
\end{frame}


\begin{frame}<beamer:0>[fragile]{Lösung}
\begin{solutionblock}{Liste durchsuchen}
\begin{minted}{python}
grades = []
while True:
  grade = input("Gib eine Note an: ")
  if grade == "q":
    break
  grades.append(int(grade))

for grade in grades:
  if grade == 0:
    print("Du hast irgendwo unterpunktet.")
    break
else: 
  print("Du hast nirgendwo unterpunktet.")
\end{minted}
\end{solutionblock}
\end{frame}

\begin{fragile}

\begin{block}{Ist ein Element in einer Liste enthalten?}
\vspace{2pt}	
Möchte man prüfen, ob ein Element in einer Liste enthalten ist, so kann man auch das Schlüsselwort \py{in} verwenden. 
\end{block}

\pause 
\vspace{12pt}

\begin{exampleblock}{Beispiel}
\vspace{2pt}
\begin{overprint}
\onslide<2|handout:0>
\begin{minted}{python}
my_hobbies = ["Segeln", "Tennis", "Schwimmen", "Lesen"]

sailing_in_list = "Segeln" in my_hobbies
climbing_in_list = "Klettern" in my_hobbies

print(sailing_in_list)  
print(climbing_in_list)
\end{minted}
\onslide<3|handout:1>
\begin{minted}{python}
my_hobbies = ["Segeln", "Tennis", "Schwimmen", "Lesen"]

sailing_in_list = "Segeln" in my_hobbies
climbing_in_list = "Klettern" in my_hobbies

print(sailing_in_list)   #  True
print(climbing_in_list)  #  False
\end{minted}
\end{overprint}

\end{exampleblock}
\end{fragile}


\begin{fragile}
	
\begin{block}{Eine Liste sortieren}
\vspace{2pt}
Um eine Liste zu sortieren, verwende die Methode \pybw{.sort()}. Dies verändert die Liste dauerhaft. \\
\pause 
Um eine sortierte Kopie einer Liste zu erstellen, verwende die Funktion \pybw{sorted()}.  \\
\pause 
Mit Hilfe des Parameters \pybw{reverse=True} lässt sich eine Liste absteigend ordnen. 
\end{block}	

\pause \vspace{12pt}

\begin{exampleblock}{Beispiel für \texttt{sort}}
\vspace{2pt}
\begin{overprint}
\onslide<4|handout:0>
\begin{minted}{python}
my_list = [1, 5, 2, 7]
my_list.sort()
print(my_list)  
\end{minted}
\onslide<5-|handout:1>
\begin{minted}{python}
my_list = [1, 5, 2, 7]
my_list.sort()
print(my_list)  # [1, 2, 5, 7]
\end{minted}
\end{overprint}

\end{exampleblock}

\vspace{12pt}

\pause \pause 

\begin{exampleblock}{Beispiel für \texttt{sorted}}
\vspace{2pt}
\begin{overprint}
\onslide<6|handout:0>
\begin{minted}{python}
my_list = [1, 5, 2, 7]
sorted_list = sorted(my_list)
print(my_list)  
print(sorted_list)  
\end{minted}
\onslide<7|handout:1>
\begin{minted}{python}
my_list = [1, 5, 2, 7]
sorted_list = sorted(my_list)
print(my_list)  # [1, 5, 2, 7]
print(sorted_list)  # [1, 2, 5, 7]
\end{minted}
\end{overprint}
\end{exampleblock}
\end{fragile}

\begin{fragile}
\begin{exampleblock}{Beispiel für absteigende Sortierung}
\vspace{2pt}
\begin{overprint}
\onslide<1|handout:0>
\begin{minted}{python}
my_list = [1, 5, 2, 7]
my_list.sort(reverse=True)
print(my_list)  

my_list = [7, 12, 5, 18]
sorted_list = sorted(my_list, reverse=True)
print(sorted_list) 
\end{minted}
\onslide<2|handout:1>
\begin{minted}{python}
my_list = [1, 5, 2, 7]
my_list.sort(reverse=True)
print(my_list)  # [7, 5, 2, 1]

my_list = [7, 12, 5, 18]
sorted_list = sorted(my_list, reverse=True)
print(sorted_list) # [18, 12, 7, 5] 
\end{minted}
\end{overprint}
\end{exampleblock}
\end{fragile}


\begin{fragile}[Übung]
	
\begin{block}{Beste/Schlechteste Note}
\vspace{2pt}
Lies wieder ein paar Noten ein. Gib dann auf der Konsole einmal die beste und einmal die schlechteste Note aus. 
\end{block}	

\vspace{12pt}

\begin{solutionblock}{Lösung}
\begin{minted}{python}
#  ...
#  einlesen wie immer
grades.sort()
min_grade = grades[0]
max_grade = grades[-1]
print(f"Schlechteste Note: {min_grade}")
print(f"Beste Note: {max_grade}")
\end{minted}
\end{solutionblock}
\end{fragile}

\begin{fragile}
\begin{block}{Nützliche Funktionen/Methoden}
\vspace{2pt}	
Für Listen stellt Python viele nützliche Methoden bzw. Funktionen bereit. Wenn Du googlest, findest Du für viele \enquote{Alltagsfragen} eine Lösung. 

Zum Beispiel hier: \texttt{https://docs.python.org/3/tutorial/datastructures.html}
\end{block}

\pause
\vspace{12pt}


\begin{exampleblock}{Beispiele}
\begin{minted}{python}
my_list = [2, 4, 8, 1]

len(my_list)   # = 4  (Gibt die Anzahl der Elemente an)
sum(my_list)   # = 15 (Berechnet die Summe der Elemente)
my_list.reverse() # [1, 8, 4, 2] (Dreht die Reihenfolge um)
my_list.insert(2,-1) # [2, 4, -1, 8, 1] (fügt den Wert -1 an Position 2 ein)
my_list.pop() # 1 (Gibt den letzten Eintrag der Liste zurück und entfernt ihn aus der Liste)
\end{minted}
\end{exampleblock}
\end{fragile}


\begin{fragile}[Übung]

\begin{block}{Durchschnittsnote}
\vspace{2pt}
Lies wieder ein paar Noten ein. Gib auf der Konsole die Durchschnittsnote aus. 
\end{block}	

\vspace{12pt}

\begin{solutionblock}{Lösung}
	\begin{minted}{python}
	#  ...
	#  einlesen wie immer
	
	total_sum = sum(grades)
	count = len(grades)
	average = total_sum/count
	print(f"Die Durchschnittsnote ist {average}")
	\end{minted}
\end{solutionblock}
	
\end{fragile}

\begin{frame}
\begin{block}{Slicing}
\vspace{2pt}
Wenn man eine Liste hat, ist es oft nötig, einen Teil der Liste \enquote{auszuschneiden}.\\
\pause
Dafür hat Python die \emph{Slice-Notation} eingeführt. \\
\pause 
Diese funktioniert nach folgendem Schema: 

\pause  \py{my_list[start:stop:step]}. 

\pause 
Die Einträge (start, stop, step) sind dabei jeweils optional. Wie immer wird der obere Wert (\pybw{stop}) gerade nicht erreicht.  
\pause 


Slicing lässt sich übrigens auch nach dem gleichen Schema auch auf Strings anwenden. 
\end{block}	

\pause
\vspace{12pt}

\begin{alertblock}{Wichtig}
\vspace{2pt}
Wenn man Slicing anwendet, erhält man eine Kopie der ausgewählten Elemente zurück. Die ursprüngliche Liste wird \emph{nicht} verändert. 
\end{alertblock}
\end{frame}


\begin{fragile}
\begin{exampleblock}{Beispiele}
	\vspace{2pt}
\begin{overprint}
\onslide<1|handout:0>
\begin{minted}{python}
my_list = [2, 4, 6, 8, 10]

my_list[1:3]    
my_list[0:4]     
my_list[1:1]     
my_list[0:4:2]   
my_list[:3]      
my_list[2:]      
my_list[:]      
my_list[1:-2]    
my_list[-3:-1]  
my_list[::-1]    
\end{minted}

\onslide<2|handout:0>
\begin{minted}{python}
my_list = [2, 4, 6, 8, 10]

my_list[1:3]     # [4, 6]
my_list[0:4]     
my_list[1:1]     
my_list[0:4:2]   
my_list[:3]      
my_list[2:]     
my_list[:]       
my_list[1:-2]   
my_list[-3:-1]  
my_list[::-1]    
\end{minted}

\onslide<3|handout:0>
\begin{minted}{python}
my_list = [2, 4, 6, 8, 10]

my_list[1:3]     # [4, 6]
my_list[0:4]     # [2, 4, 6, 8]
my_list[1:1]     
my_list[0:4:2]   
my_list[:3]      
my_list[2:]      
my_list[:]       
my_list[1:-2]    
my_list[-3:-1]  
my_list[::-1]    
\end{minted}

\onslide<4|handout:0>
\begin{minted}{python}
my_list = [2, 4, 6, 8, 10]

my_list[1:3]     # [4, 6]
my_list[0:4]     # [2, 4, 6, 8]
my_list[1:1]     # []
my_list[0:4:2]   
my_list[:3]      
my_list[2:]     
my_list[:]       
my_list[1:-2]    
my_list[-3:-1]   
my_list[::-1]    
\end{minted}


\onslide<5|handout:0>
\begin{minted}{python}
my_list = [2, 4, 6, 8, 10]

my_list[1:3]     # [4, 6]
my_list[0:4]     # [2, 4, 6, 8]
my_list[1:1]     # []
my_list[0:4:2]   
my_list[:3]      
my_list[2:]      
my_list[:]       
my_list[1:-2]    
my_list[-3:-1]   
my_list[::-1]    
\end{minted}

\onslide<6|handout:0>
\begin{minted}{python}
my_list = [2, 4, 6, 8, 10]

my_list[1:3]     # [4, 6]
my_list[0:4]     # [2, 4, 6, 8]
my_list[1:1]     # []
my_list[0:4:2]   # [2, 6]
my_list[:3]      
my_list[2:]      
my_list[:]       
my_list[1:-2]    
my_list[-3:-1]   
my_list[::-1]    
\end{minted}

\onslide<7|handout:0>
\begin{minted}{python}
my_list = [2, 4, 6, 8, 10]

my_list[1:3]     # [4, 6]
my_list[0:4]     # [2, 4, 6, 8]
my_list[1:1]     # []
my_list[0:4:2]   # [2, 6]
my_list[:3]      # [2, 4, 6]
my_list[2:]      
my_list[:]       
my_list[1:-2]   
my_list[-3:-1]   
my_list[::-1]    
\end{minted}

\onslide<8|handout:0>
\begin{minted}{python}
my_list = [2, 4, 6, 8, 10]

my_list[1:3]     # [4, 6]
my_list[0:4]     # [2, 4, 6, 8]
my_list[1:1]     # []
my_list[0:4:2]   # [2, 6]
my_list[:3]      # [2, 4, 6]
my_list[2:]      # [6, 8, 10]
my_list[:]       
my_list[1:-2]    
my_list[-3:-1]  
my_list[::-1]    
\end{minted}

\onslide<9|handout:0>
\begin{minted}{python}
my_list = [2, 4, 6, 8, 10]

my_list[1:3]     # [4, 6]
my_list[0:4]     # [2, 4, 6, 8]
my_list[1:1]     # []
my_list[0:4:2]   # [2, 6]
my_list[:3]      # [2, 4, 6]
my_list[2:]      # [6, 8, 10]
my_list[:]       # [2, 4, 6, 8, 10]
my_list[1:-2]    
my_list[-3:-1]   
my_list[::-1]    
\end{minted}

\onslide<10|handout:0>
\begin{minted}{python}
my_list = [2, 4, 6, 8, 10]

my_list[1:3]     # [4, 6]
my_list[0:4]     # [2, 4, 6, 8]
my_list[1:1]     # []
my_list[0:4:2]   # [2, 6]
my_list[:3]      # [2, 4, 6]
my_list[2:]      # [6, 8, 10]
my_list[:]       # [2, 4, 6, 8, 10]
my_list[1:-2]    # [6]
my_list[-3:-1]    
my_list[::-1]     
\end{minted}

\onslide<11|handout:0>
\begin{minted}{python}
my_list = [2, 4, 6, 8, 10]

my_list[1:3]     # [4, 6]
my_list[0:4]     # [2, 4, 6, 8]
my_list[1:1]     # []
my_list[0:4:2]   # [2, 6]
my_list[:3]      # [2, 4, 6]
my_list[2:]      # [6, 8, 10]
my_list[:]       # [2, 4, 6, 8, 10]
my_list[1:-2]    # [6]
my_list[-3:-1]   # [6, 8] 
my_list[::-1]     
\end{minted}

\onslide<12|handout:1>
\begin{minted}{python}
my_list = [2, 4, 6, 8, 10]

my_list[1:3]     # [4, 6]
my_list[0:4]     # [2, 4, 6, 8]
my_list[1:1]     # []
my_list[0:4:2]   # [2, 6]
my_list[:3]      # [2, 4, 6]
my_list[2:]      # [6, 8, 10]
my_list[:]       # [2, 4, 6, 8, 10]
my_list[1:-2]    # [6]
my_list[-3:-1]   # [6, 8] 
my_list[::-1]    # [10, 8, 6, 4, 2]  
\end{minted}

\end{overprint}
\end{exampleblock}
\end{fragile}








\section{Dictionaries}

\begin{frame}
\begin{block}{Problemstellung}
\vspace{2pt}
In einem Notenrechner wie eben, sollen nicht nur die Noten gespeichert werden, sondern auch das Fach, in dem die Note erreicht wurde. 

\vspace{8pt}

Wie macht man das? 
\end{block}
\end{frame}

\begin{fragile}{}
\begin{block}{Lösung}
	\begin{minted}{python}
	# ...
	grades = { "Deutsch": 14, "Mathematik": 8, "Biologie": 11, "Sport": 13}
	
	key = input("Welche Note möchtest Du wissen?")
	
	print(f"Deine Note in { key } ist { grades[key] } Punkte")
	\end{minted}
\end{block}
\end{fragile}

\begin{fragile}
	
	\metroset{block=fill}
	\begin{block}{Struktur eines \emph{Dictionaries}}
		\vspace{2pt}
		\large
		\texttt{my\_dict = }\pause {\Large\texttt{\{}}\pause 
		\texttt{key\_1}\pause\texttt{:}\pause\texttt{value\_1}\pause,
		\pause 
		\texttt{key\_2:value\_2}, \pause 
		\dots   
		, \texttt{key\_n:value\_n}\pause \Large{\texttt{\}}}
	\end{block}
	\pause 
	
	Das \emph{Dictionary} \py{my_dict} enthält Schlüssel-Wert-Paare (\emph{key-value-pairs}). Die Schlüssel müssen eindeutig und unveränderlich sein (z.B. vom Typ \pybw{string} oder \pybw{int}). Die Werte dürfen beliebige Datentypen sein. 
	
\end{fragile}

\begin{fragile}

\begin{block}{Good to know}
	\pause
\begin{itemize}[<+->]
\item Zur besseren Übersichtlichkeit werden Dictionaries oftmals wie folgt formatiert: 
\begin{minted}{python}
grades = { 
  "Deutsch": 14, 
  "Mathematik": 8, 
  "Biologie": 11, 
  "Sport": 13
}
\end{minted}
\item Dictionaries sind mutable, können also verändert werden. 
\item Dictionaries besitzen keine vernünftige Anordnung und können nicht geordnet werden. 
\item Ein Dictionary kann leer sein. 
\end{itemize}
\end{block}
\end{fragile}

\begin{fragile}
\begin{itemize}
\item Oftmals bietet es sich an, statt einem Dictionary eine Liste von Dictionaries zu verwenden:
\begin{minted}{python}
grades = [
  { 
    "subject": "Deutsch", 
    "grade": 14,
    "is_major": True
  },
  # ... 
  {
    "subject": "Sport",
    "grade": 11, 
    "is_major": False
  }
]
\end{minted}
\end{itemize}
\end{fragile}



\begin{frame}


\begin{block}{Auf Dictionary-Elemente zugreifen}
	
	\vspace{2pt}
	
	Sei \py{my_dict = {"a": 5, "b": 8}}.
	
	\pause
	
	Mit der Syntax \py{my_dict["a"]} kann man den Wert an der Stelle \py{"a"} auslesen. 
	
	\pause 
	
	Mit der Syntax \py{my_dict["a"] = 12} kann  man einzelne Werte des Dictionaries verändern. 
	
	\pause 
	
	Auf diese Weise können auch ganz neue Paare hinzugefügt werden. Zum Beispiel: \py{my_dict["c"] = -2}. 
\end{block}
\end{frame}


\begin{frame}{Übung}

\begin{block}{Dictionary manipulieren}
	\vspace{2pt}
	Gegeben sei das folgende Dictionary: 
	
	\py{grades = { "Mathe": 8, "Bio": 11, "Sport": 13}} 
	
	Bestimme die Durchschnittsnote dieser drei Fächer. Verbessere danach Deine Mathenote um einen Punkt und füge noch eine weitere Note für Englisch hinzu (Abfrage über Konsole). Gib danach erneut den Durchschnitt an.  
\end{block}

\end{frame}


\begin{frame}<beamer:0>[fragile]{Lösung}

\begin{solutionblock}{Dictionary manipulieren}
\begin{minted}{python}
grades = { "Mathe": 8, "Bio": 11, "Sport": 13}

grades_sum = grades["Mathe"] + grades["Bio"] + grades["Sport"]
average = grades_sum/len(grades)
print(f"Der Durchschnitt ist {average} Punkte")

grades["Mathe"] += 1

eng_grade = input("Welche Note hast Du in Englisch? ")
eng_grade = int(eng_grade)
grades["Englisch"] = eng_grade

grades_sum += grades["Englisch"]
average = grades_sum/len(grades)
print(f"Der Durchschnitt ist {average} Punkte")
\end{minted}
\end{solutionblock}
\end{frame}



\begin{frame}
\begin{block}{Einen Eintrag aus einem Dictionary entfernen}
\vspace{2pt}
Wie bei Listen, kann man mittels \py{del}-Statement einen Eintrag aus einem Dictionary entfernen: 

\py{del my_dict["a"]}	

\end{block}	



\end{frame}	

\begin{fragile}
\begin{block}{Was wird hier passieren?}
\vspace{2pt}
\begin{minted}{python}
grades = { "Mathe": 8, "Bio": 11, "Sport": 13}
new_grades = grades

new_grades["Mathe"] = 15

print(grades)
print(new_grades)
\end{minted}
\end{block}

\pause 
\vspace{12pt}

\begin{block}{Erklärung}
	\vspace{2pt}
Intern befindet sich in der Variable \pybw{grades} nur ein Verweis auf das Dictionary im Speicher. Die Variable \pybw{new_grades} enthält den gleichen Speicherverweis. Das heißt, wenn man das Dictionary \pybw{new_grades} verändert, verändert sich auch das Dictionary \pybw{grades}, weil es sich um ein-und dasselbe Dictionary handelt. 
\end{block}


\end{fragile}

\begin{frame}
	\begin{block}{Eine Kopie von einem Dictionary erstellen}
		\vspace{2pt}
		Mit der Funktion \py{dict()} kann man eine Kopie von einem Dictionary erstellen. 
		
		Beispiel: \py{dict(my_dict)} erstellt eine Kopie von \py{my_dict}.
	\end{block}
\end{frame}

\begin{fragile}
\begin{block}{Schleife über Dictionary I}
\vspace{2pt}
Ähnlich wie bei Listen kann man Schleifen auch über ein Dictionary laufen lassen.  
\end{block}
\vspace{12pt}
\pause 

\begin{exampleblock}{Beispiel}
\vspace{2pt}
\begin{overprint}
\onslide<2|handout:0>
\begin{minted}{python}
grades = { "Mathe": 8, "Bio": 11, "Sport": 13}

for item in grades:
  print(item)
\end{minted}
\onslide<3|handout:1>
\begin{minted}{python}
grades = { "Mathe": 8, "Bio": 11, "Sport": 13}

for item in grades:
  print(item)

# Mathe
# Bio
# Sport
\end{minted}
\end{overprint}
\end{exampleblock}
\end{fragile}

\begin{fragile}
\begin{block}{Schleife über Dictionary II}
\vspace{2pt}
Möchte man in der Schleife nicht nur die Schlüssel, sondern auch die Werte des Dictionaries zur Verfügung haben, so muss man die Methode \py{.items()} auf das Dictionary anwenden.   
\end{block}
\vspace{12pt}
\pause 


\begin{exampleblock}{Beispiel}
\vspace{2pt}
\begin{overprint}
\onslide<2|handout:0>
\begin{minted}{python}
grades = { "Mathe": 8, "Bio": 11, "Sport": 13}

for key, value in grades.items():
  print(f"{key}:{value}")
\end{minted}
\onslide<3|handout:1>
\begin{minted}{python}
grades = { "Mathe": 8, "Bio": 11, "Sport": 13}

for key, value in grades.items():
  print(f"{key}:{value}")

# Mathe:12
# Bio:11
# Sport:13
\end{minted}
\end{overprint}
\end{exampleblock}
\end{fragile}

\begin{frame}{Übungen}

\begin{block}{Zwei Dictionaries kombinieren}
	\vspace{2pt}
Gegeben seien zwei Dictionaries, z.B.  

\py{majors = {"Deutsch": 11, "Mathe": 7}}

und 

\py{minors = {"Sport": 14, "Bio": 8 }}

Füge die Einträge des zweiten Dictionaries zum ersten Dictionary hinzu. 
\end{block}

\pause 

\vspace{12pt}

\begin{block}{Ein Dictionary \enquote{filtern}}
\vspace{2pt}
Sei ein beliebiges Dictionary mit Noten gegeben. Entferne alle Einträge, deren Note schlechter als 5 Punkte ist. 
\end{block}
\end{frame}


\begin{frame}<beamer:0>[fragile]{Lösung}

\begin{solutionblock}{Zwei Dictionaries kombinieren}
\begin{minted}{python}
majors = {"Deutsch": 11, "Mathe": 7}
minors = {"Sport": 14, "Bio": 8}

for key,value in minors.items():
  majors[key] = value
print(majors)
\end{minted}
\end{solutionblock}

\vspace{12pt}

\begin{solutionblock}{Ein Dictionary \enquote{filtern}}
\begin{minted}{python}
grades = {"Deutsch": 11, "Mathe": 3, "Sport": 14, "Geschichte": 1}
#  Man darf die Länge eines Dictionaries in einer Schleife nicht verändern, deshalb machen wir eine Kopie
result = dict(grades)
for key, value in grades.items():
  if value < 5:
    del result[key]
print(result)
\end{minted}
\end{solutionblock}

\end{frame}


\begin{fragile}

\begin{block}{Ein Dictionary zerlegen}
\vspace{2pt}
Mit der Methode \py{.keys()} erhält man eine Liste aller Schlüssel eines Dictionaries. \pause

Mit der Methode \py{.values()} erhält man eine Liste aller Werte eines Dictionaries. \pause 

In beiden Fällen, muss das Ergebnis mittels der Funktion \py{list()} in eine Liste umgewandelt werden. 
\end{block}


\vspace{12pt}
\pause 

\begin{exampleblock}{Beispiel}
\vspace{2pt}
\begin{overprint}
\onslide<4|handout:0>
\begin{minted}{python}
grades = { "Mathe": 8, "Bio": 11, "Sport": 13}

grade_subjects = grades.keys()
grade_subjects = list(grade_subjects)

grade_numbers = grades.values()
grade_numbers = list(grade_numbers)

print(grade_subjects)
print(grade_numbers)
\end{minted}
\onslide<5|handout:1>
\begin{minted}{python}
grades = { "Mathe": 8, "Bio": 11, "Sport": 13}

grade_subjects = grades.keys()
grade_subjects = list(grade_subjects)

grade_numbers = grades.values()
grade_numbers = list(grade_numbers)

print(grade_subjects)  # ["Mathe", "Bio", "Sport"]
print(grade_numbers)   # [8, 11, 13]
\end{minted}
\end{overprint}
\end{exampleblock}
\end{fragile}

\section{Funktionen \\ \footnotesize Wie man Code wiederverwerten kann}

\begin{frame}
\begin{block}{Problemstellung}
	\vspace{2pt}
	Es sei eine Liste mit Noten gegeben: 
	
	\py{grades = [7, 12, 8, 10, 2, 0, 3, 5, 6]}
	
	\pause
	Es sollen zunächst folgende Durchschnittsnoten berechnet werden: 
	\begin{itemize}
		\item Durchschnitt aller Noten
		\item Der Durchschnitt der ersten drei Noten
		\item Der Durchschnitt jeder zweiten Note
	\end{itemize}
\pause 
	Statt durch eine Zahl, soll das Ergebnis jedoch mit den Worten  
	\begin{itemize}
		\item \py{"Passt"} für Durchschnitte $\geq$ 5 Punkte
		\item \py{"Durchgefallen"} für Durschnitte $<$ 5 Punkte 
	\end{itemize}

	abgespeichert werden. 
	
	
	\pause 
	
	\vspace{8pt}
	
	Wie macht man das \emph{elegant}? 
\end{block}
\end{frame}

\begin{fragile}{}
\begin{block}{Lösung \onslide<8->{\footnotesize (Hauptsache es funktioniert)}}
\vspace{2pt}
\begin{overprint}
\onslide<2|handout:0>
\begin{minted}{python}
average = sum(grades) / len(grades)
   
   














#   
\end{minted}
\onslide<3|handout:0>
\begin{minted}{python}
average = sum(grades) / len(grades)
if average >= 5: 
  average = "Passt"
else: 
  average = "Durchgefallen"
  
  
  
  
  
  
  
  
  
  
  
  
#  
\end{minted}
\onslide<4|handout:0>
\begin{minted}{python}
average = sum(grades) / len(grades)
if average >= 5: 
  average = "Passt"
else: 
  average = "Durchgefallen"

average_2 = sum(grades[:3]) / len(grades[:3])










#
\end{minted}
\onslide<5|handout:0>
\begin{minted}{python}
average = sum(grades) / len(grades)
if average >= 5: 
  average = "Passt"
else: 
  average = "Durchgefallen"

average_2 = sum(grades[:3]) / len(grades[:3])
if average_2 >= 5:
  average_2 = "Passt"
else: 
  average_3 = "Durchgefallen"






#
\end{minted}
\onslide<6|handout:0>
\begin{minted}{python}
average = sum(grades) / len(grades)
if average >= 5: 
  average = "Passt"
else: 
  average = "Durchgefallen"

average_2 = sum(grades[:3]) / len(grades[:3])
if average_2 >= 5:
  average_2 = "Passt"
else: 
  average_3 = "Durchgefallen"

average_3 = sum(grades[::2]) / len(grades[::2])




#
\end{minted}
\onslide<7-|handout:1>
\begin{minted}{python}
average = sum(grades) / len(grades)
if average >= 5: 
  average = "Passt"
else: 
  average = "Durchgefallen"

average_2 = sum(grades[:3]) / len(grades[:3])
if average_2 >= 5:
  average_2 = "Passt"
else: 
  average_3 = "Durchgefallen"

average_3 = sum(grades[::2]) / len(grades[::2])
if average_3 < 5:
  average_3 = "Durchgefallen"
else: 
  average_3 = "Passt"
#
\end{minted}
\end{overprint}
\end{block}
\end{fragile}

\begin{frame}
\begin{block}{Nachteile dieser Lösung}
	\pause 
	\begin{itemize}[<+->]
	\item Viel Schreibarbeit, viel Wiederholung
	\item Der Code ist schwierig zu lesen. Man sieht vor lauter Wiederholungen nicht, was passiert. 
	\item Jedes Mal, wenn man diese \enquote{Berechnungslogik} verwendet, könnte man einen (Tipp-)Fehler machen.  
	\item Wenn man das Anforderungsprofil minimal ändert, muss diese \enquote{Logik} bei \emph{jedem} Auftreten im Code geändert werden
		(z.B. statt \py{"Passt"} soll das Ergebnis \py{"Bestanden"} heißen). In echten Projekten, kann das schnell ein paar Hundert Male sein. 	
	\end{itemize}
\end{block}
\end{frame}

\begin{fragile}
\begin{block}{Bessere Lösung}
\begin{minted}{python}
def compute_average(grade_list):
  result = sum(grade_list) / len(grade_list)
  if result >= 5:
    result = "Passt"
  else: 
    result = "Durchgefallen"
  return result

average = compute_average(grades)
average_2 = compute_average(grades[:3])
average_3 = compute_average(grades[::2])
\end{minted}
\end{block}

\end{fragile}

\begin{frame}
\metroset{block=fill}
\begin{block}{Definition: Funktion}
\vspace{2pt}
Eine Funktion ist ein Codeblock, der nur ausgeführt wird, wenn die Funktion \emph{aufgerufen} wird. 
Man kann der Funktion Werte als \emph{Parameter} übergeben. 
Sie kann auch einen Wert als Ergebnis \emph{zurückgeben}. 
\end{block}

\pause 

\vspace{12pt}
Man kann sich eine Funktion wie eine Maschine vorstellen, wo man oben Dinge (=Parameter) hineinfüllt und unten ein Ergebnis (=Rückgabewert) herausbekommt. 
Unabhängig von dem Eingabe-Ausgabe-Prinzip, kann solch eine Maschine auch Nebeneffekte (z.B. Krach) produzieren. 

\pause 

\vspace{12pt}
Man unterscheidet zwischen \emph{Definition} und \emph{Ausführung} einer Funktion. 
\end{frame}



\begin{frame}
\metroset{block=fill}


\renewcommand{\baselinestretch}{1.5}
\begin{block}{Struktur der Funktions-Definition}	
\vspace{2pt}

\pause 

\texttt{def} \pause \textit{Funktionsname}\pause\texttt{(}\pause\textit{Parameter\_0}\pause, \textit{Parameter\_1}, \dots, \textit{Parameter\_n}\pause\texttt{)}\pause\texttt{:}\\
\pause \spacechar \spacechar \textit{Codezeile1} \\
\pause \spacechar \spacechar \textit{Codezeile2} \\
\pause \phantom{Code} \vdots \\
\pause \spacechar \spacechar \texttt{return} \textit{Ergebnis}
\end{block}
\renewcommand{\baselinestretch}{1}
\vspace{12pt}

\pause 

\begin{block}{Struktur eines Funktionsaufrufs}	
\vspace{2pt}
result = \textit{Funktionsname}\texttt{(}\textit{Argument\_0}, \textit{Argument\_1}, \dots, \textit{Argument\_n}\texttt{)}
\end{block}

\end{frame}


\begin{frame}

\begin{block}{Good to know}
\pause 
\begin{itemize}[<+->]
	\item Eine Funktion muss schon \emph{vor} dem ersten Aufruf definiert worden sein (das ist nicht in allen Sprachen so). 
	\item Die Eingabwerte nennt man in der Funktionsdefintion \emph{Parameter}, beim Aufruf der Funktion nennt man sie jedoch \emph{Argumente}.
	\item Nicht jede Funktion braucht Eingangsdaten. Die Liste von Parametern einer Funktion kann daher leer sein.   
	\item Beim Aufruf spielt die Reihenfolge der angegebenen Argumente eine entscheidene Rolle. Sie werden entsprechend der Reihenfolge den Parametern in der Definition zugeordnet. 
%	\item Den Rückgabewert der Funktion erhält man durch den Zuweisungsoperator (\py{=}).
	\item Eine Funktion muss nicht unbedingt etwas zurückgeben, d.h. das \py{return}-Statement ist optional.
	\item Das \py{return}-Statement muss nicht unbedingt am Schluss der Funktion stehen. Jedoch wird Code, der nach dem \py{return}-Statement kommt, nicht mehr ausgeführt. 
\end{itemize}
\end{block}
\end{frame}

\begin{frame}{Übungen}

\begin{block}{Funktion ohne Parameter}
	\vspace{2pt}
Schreibe eine Funktion, die Deinen Namen auf der Konsole ausgibt. 
\end{block}
\vspace{12pt}
\begin{block}{Funktion mit einem Parameter}
\vspace{2pt}
Schreibe eine Funktion, die die übergebene Zahl verdoppelt. 
\end{block}
\vspace{12pt}
\begin{block}{Funktion mit zwei Parametern}
\vspace{2pt}
Schreibe eine Funktion, die die beiden übergebenen Zahlen multipliziert. 
\end{block}
\vspace{12pt}
\begin{block}{Funktion ohne Rückgabewert}
\vspace{2pt}
Was gibt eine Funktion zurück, die kein \py{return}-Statement enthält?
\end{block}
\end{frame}

\begin{frame}<beamer:0>[fragile]{Lösungen}

\begin{solutionblock}{Funktion ohne Parameter}
\begin{minted}{python}
def my_name(): 
  print("Aaron Kunert")
\end{minted}
\end{solutionblock}

\vspace{12pt}

\begin{solutionblock}{Funktion mit einem Parameter}
\begin{minted}{python}
def double(number): 
  return number * 2
\end{minted}
\end{solutionblock}

\vspace{12pt}

\begin{solutionblock}{Funktion mit zwei Parametern}
\begin{minted}{python}
def multiply(number1,number2): 
  return number1 * number2
\end{minted}
\end{solutionblock}


\end{frame}

\begin{frame}{Übung}
\begin{block}{Aggregatzustand von Wasser}
\vspace{2pt}
Schreibe eine Funktion, die entsprechend der übergebenen Temperatur den Aggregatzustand von Wasser (\py{"fest"},\py{"flüssig"},\py{"gasförmig"}) als String zurückgibt. 

Schaffst Du es ohne die Schlüsselwörter \py{elif} und \py{else}? 
\end{block}
\end{frame}

\begin{frame}<beamer:0>[fragile]{Lösung}

\begin{solutionblock}{Aggregatzustand von Wasser}
\begin{minted}{python}
def get_state(temp):
  if temp < 0:
    return "fest"
  if temp > 100:
    return "gasförmig"
  return "flüssig"
\end{minted}
\end{solutionblock}

\end{frame}


\begin{fragile}[Komplexere Übung]

\begin{block}{Gewichtete Durschnittsnote}
	\vspace{2pt}
Schreibe eine Funktion, die eine Liste der folgenden Struktur erwartet: 

\begin{minted}{python}
grades = [ 
  { 
    "subject": "Deutsch", 
    "grade": 14,
    "is_major": True
  },
  # ... 
  {
    "subject": "Sport",
    "grade": 11, 
    "is_major": False
  }
]
\end{minted}

Berechne die Durchschnittsnote, wobei Hauptfächer doppelt gewichtet werden sollen. 
\end{block}

\end{fragile}

\begin{frame}<beamer:0>[fragile]{Lösung}

\begin{solutionblock}{Gewichtete Durchschnittsnote}
\begin{minted}{python}
def weighted_average(grades): 
  weighted_sum = 0
  weighted_length = 0
  for grade in grades: 
    if grade["is_major"]: 
      weighted_sum += 2 * grade["grade"]
      weighted_length += 2
    else:
      weighted_sum += grade["grade"]
      weighted_length += 1
  result = weighted_sum/weighted_length
  return result 
\end{minted}
\end{solutionblock}

\end{frame}


\begin{fragile}

\begin{block}{Optionale Parameter}
	
	\pause 
	
\vspace{2pt}
Manchmal wirst Du bei Funktionen bemerken, dass einige der Parameter fast immer den gleichen Wert haben. In diesem Fall, möchtest Du diese Parameter nicht bei jedem Aufruf immer hinschreiben, sondern nur dort, wo er vom Standardfall abweicht. Dies ist möglich, wenn man den Standardwert (\emph{default value}) bei der Definition mit angibt. 

\pause 

\textbf{Wichtig:} Bei der Definition müssen die optionalen Parameter immer hinter den Pflichtparametern stehen.
\end{block}

\vspace{12pt}
\pause 

\begin{exampleblock}{Beispiel}
\begin{minted}{python}
def double(number, factor=2): 
  return number * factor
\end{minted} 


\pause 

Diese Funktion ist sehr vielseitig: Im einfachen Fall verdoppelt sie die eingegebene Zahl. Optional lässt sich der Faktor aber beliebig verändern. 
\end{exampleblock}

\end{fragile}


\begin{frame}
\begin{block}{Typischer Einsatzbereich}
\vspace{2pt}
Oftmals merkt man im Verlauf eines Projektes, dass eine gegebene Funktion nicht flexibel genug ist, dann kann man sie um einen optionalen Parameter erweitern, ohne den bisherigen Code verändern zu müssen. 
\end{block}

\vspace{12pt}

\pause 

\begin{exampleblock}{Fiktives Beispiel}
\vspace{2pt}
Stell Dir vor, Du baust einen Rechner für Deine Endnote. Hauptfachnoten werden immer doppelt gewichtet, daher verwendest Du die Funktion \py{weighted_average}, wie in der Übung. Plötzlich kommt raus, dass in der Abschlussprüfung, Hauptfächer vierfach gewichtet werden. Also erweiterst Du die Funktion, so dass der Gewichtungsfaktor anpassbar ist.

\pause

Jedoch möchtest Du den bisherigen Code nicht verändern. Daher definierst Du den Gewichtungsfaktor als optionalen Parameter, so dass die Funktion \enquote{abwärtskompatibel} zu ihrer bisherigen Verwendung ist.

\pause 
Die Definition startet dann mit \py{def weighted_average(grades, weight=2):} 
 
\end{exampleblock}

\end{frame}

\begin{frame}{Übung}

\begin{block}{Flexibler Durchschnittsrechner}
\vspace{2pt}
Erweitere die Funktion zur Berechnung von gewichteten Durchschnittsnoten so, dass optional der Gewichtungsfaktor angegeben werden kann. 	
\end{block}

\end{frame}

\begin{frame}<beamer:0>[fragile]{Lösung}

\begin{solutionblock}{Flexibler Durchschnittsrechner}
\begin{minted}{python}
def weighted_average(grades, weight=2): 
  weighted_sum = 0
  weighted_length = 0
  for grade in grades: 
    if grade["is_major"]: 
      weighted_sum += weight * grade["grade"]
      weighted_length += weight
    else:
      weighted_sum += grade["grade"]
      weighted_length += 1
  result = weighted_sum/weighted_length
  return result 
\end{minted}
\end{solutionblock}

\end{frame}

\begin{fragile}
	
\begin{block}{Named Parameters}
\vspace{2pt}
Hat eine Funktion viele Parameter, von denen etliche optional sind, so kann man einen Parameter statt über die Reihenfolge auch über den Namen übergeben. 
\end{block}

\pause 
\vspace{12pt}

\begin{exampleblock}{Beispiel}
\vspace{2pt}

\begin{minted}{python}
def my_function(parameter1, parameter2=0, parameter3="x", parameter4=-17):
  # ... 
\end{minted}
Möchte man jetzt die Funktion mit einem eigenen Wert \pybw{parameter1} und \pybw{parameter4} aufrufen aber alles andere auf Standard lassen, so geht das wie folgt: 

\py{my_function(15, parameter4=-20)}
\end{exampleblock}

	
\end{fragile}





%\section{Input/Output II \\ \footnotesize Dateien lesen/schreiben}




%\section{Arbeit mit einer IDE}

\begin{frame}
\begin{block}{Integrierte Entwicklungsumgebungen (IDE)}
	\vspace{2pt}
	\uncover<+->{
		Die Arbeit mit gewöhnlichen Texteditoren ist auf Dauer sehr mühsam. Daher empfiehlt es sich eine IDE zu verwenden. 
		Das bringt zum u.a. folgende Vorteile: 
		\begin{itemize}[<+->]
			\item Syntax-Highlighting
			\item Code-Inspection
			\item Autocomplete
			\item Geile Shortcuts
			\item Code direkt ausführen
			\item Hilfe bei der Fehlersuche (\emph{Debugging})
		\end{itemize}	
	}
\end{block}
\end{frame}

\begin{frame}
\uncover<+->{
\begin{block}{Installation von PyCharm}
	\vspace{2pt}
	\begin{enumerate}
		\item Gehe auf 
		\texttt{https://www.jetbrains.com/pycharm/download} \\
		\item Lade die kostenlose \textit{Community Edition} herunter
		\item Führe den Installer aus
		\item Öffne PyCharm
	\end{enumerate}
\end{block}
}

\uncover<+->{
\begin{block}{Wenn alles passt, sollte es etwa so aussehen:}
	\vspace{2pt}
	\begin{center}
		\includegraphics[width=0.65\textwidth]{pycharme.jpg}
	\end{center}
\end{block}
}	

\end{frame}








\begin{frame}
\begin{block}{Installation PyCharm}
\vspace{2pt}
\begin{enumerate}
\item Gehe auf \texttt{Customize > All Settings...}
\item Einstellungen synchronisieren
\begin{enumerate}
	\item \texttt{Tools > Settings Repository}
	\item Unter \textit{Read-only Sources} auf \texttt{+}
	\item \texttt{https://github.com/a-kunert/ide-settings.git}	eingeben
	%\item  File > Manage IDE Settings > Sync with Settings Repository > Merge ausführen 
\end{enumerate}
\item Verknüpfe den Interpreter
\begin{enumerate}
	\item In den Settings auf \texttt{Python Interpreter}
	\item Falls möglich unter \texttt{Python Interpreter} einen Interpreter wählen. Ansonsten wie folgt: 
	\item \texttt{Zahnrad > Add}
	\item \texttt{System Interpreter}
	\item Dort den Pfad zu Python angeben
\end{enumerate}
\item Mit dem Button \textit{Apply} alles bestätigen
\end{enumerate}
\end{block}
\end{frame}

\begin{frame}
\begin{block}{Konfiguration abschließen}
\begin{enumerate}
\item Lege eine Ordner für den Kurs an
\item Öffne diesen Ordner mit \texttt{Projects > Open}
\item \texttt{File > Manage IDE Settings > Sync with Settings Repository > Merge} ausführen
\item Bei \texttt{File > Settings} unter \texttt{Keymap} die Keymap \textit{Salem-Win/Mac} auswählen. 
\item Code in die Datei \texttt{main.py} schreiben
\item Mittels grünem Pfeil (oben rechts) Code ausführen
\end{enumerate}	
\end{block}
\end{frame}









\end{document}






