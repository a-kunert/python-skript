\section{Scope \\ \footnotesize Wo Variablen gültig sind}


\begin{frame}
\begin{block}{Problemstellung}
\vspace{2pt}
Sei \py{my_variable} eine Variable mit Wert 1. 
Schreibe eine Funktion, die bei Aufruf die Variable \py{my_variable} um 1 erhöht. 

\vspace{8pt}

Wie macht man das? 
\end{block}
\end{frame}

\begin{fragile}
	
\begin{block}{Das Problem}
\vspace{2pt}

\begin{minted}{python}
my_variable = 1

def increment(): 
  my_variable = my_variable + 1

increment()
print(my_variable)
\end{minted}

\pause 
Die offensichtliche Lösung 
funktioniert nicht. Warum nicht? 
\end{block}
\end{fragile}


\begin{fragile}
	
\begin{block}{Experiment I}
\vspace{2pt}

\begin{minted}{python}
global_variable = 1

def my_function(): 
  local_variable = 5

my_function()
print(global_variable)
print(local_variable)
\end{minted}


\vspace{12pt}

\end{block}

\begin{exampleblock}{Beobachtung}

\pause 

Eine Variable, die innerhalb einer Funktion definiert wurde, ist auch nur innerhalb der Funktion sichtbar. 
\end{exampleblock}


\end{fragile}

\begin{fragile}
	
\begin{block}{Experiment II}
\vspace{2pt}

\begin{minted}{python}
global_variable = 1

def my_function(): 
  print(global_variable)
 

my_function()
print(global_variable)
\end{minted}


\vspace{12pt}

\end{block}

\begin{exampleblock}{Beobachtung}

\pause 

Eine \emph{globale} Variable ist auch innerhalb einer Funktion definiert.  
\end{exampleblock}

	
\end{fragile}

\begin{fragile}

\begin{block}{Experiment III}
\vspace{2pt}

\begin{minted}{python}
global_variable = 1

def my_function(): 
  global_variable = 5
  print(global_variable)


my_function()
print(global_variable)
\end{minted}
\vspace{12pt}

\end{block}

\begin{exampleblock}{Beobachtung}

\pause 

Eine Variable innerhalb einer Funktion kann den gleichen Namen wie eine Variable außerhalb haben, allerdings ist die innere Variable nur innerhalb der Funktion sichtbar. 
\end{exampleblock}


\end{fragile}


\begin{fragile}
	
\begin{block}{Experiment IV}
\vspace{2pt}

\begin{minted}{python}
global_variable = 1

def my_function(): 
  print(global_variable)
  global_variable = 5


my_function()
print(global_variable)
\end{minted}
\vspace{12pt}

\end{block}

\begin{exampleblock}{Beobachtung/Erklärung}

\pause 

Python entscheidet anhand des Kontexts ob \py{global_variable} eine globale Variable ist, oder eine lokale Variable, die zufällig den gleichen Namen wie eine globale Variable trägt. 

\pause 

Falls Python denkt, dass es sich um eine globale Variable handelt, so kann diese nur gelesen, nicht aber geschrieben (d.h. neu definiert) werden. 

\end{exampleblock}

	
\end{fragile}


\begin{fragile}
	
\begin{block}{Das Eingangsbeispiel}
\vspace{2pt}

\begin{minted}{python}
my_variable = 1

def increment(): 
  my_variable = my_variable + 1

increment()
print(my_variable)
\end{minted}

\vspace{12pt}

\end{block}

\begin{exampleblock}{Erklärung}

\pause 

Da \py{my_variable} rechts vom Gleichheitszeichen steht, denkt Python, dass es sich um die globale Variable \py{my_variable} handelt. Da \py{my_variable} aber auch links vom Gleichheitszeichen steht, wird auch schreibend auf die Variable zugegriffen. Das ist nicht erlaubt. 

\end{exampleblock}
	
\end{fragile}


\begin{fragile}

\begin{block}{Mögliche Lösung}
	\vspace{2pt}
\begin{minted}{python}
my_variable = 1

def increment(var): 
  return var + 1

my_variable = increment(my_variable)
print(my_variable)
\end{minted}

\vspace{12pt}

\end{block}
	
\end{fragile}

\begin{frame}
\metroset{block=fill}
	
\begin{block}{Definition}
\vspace{2pt}
Der Gültigkeitsbereich einer Variable wird \emph{Scope} genannt. 
\end{block}
\vspace{12pt}
\pause 

\metroset{block=transparent}
\begin{block}{Scope in Python}
\vspace{2pt}
In Python unterscheidet man zwischen \emph{global Scope} und \emph{local Scope}. Im local Scope hat man nur Lesezugriff auf den global Scope. 	
\end{block}
	
\end{frame}

\begin{fragile}
\begin{alertblock}{Achtung Ausnahme}
\begin{minted}{python}
my_list = [1, 2, 3]

def append(item):
  my_list.append(item)

append(4)
print(my_list)
\end{minted}
\end{alertblock}

\vspace{12pt}

\begin{exampleblock}{Erklärung}
	
\pause 

Da die Variable \py{my_list} nicht überschrieben wird, sondern nur das referenzierte Objekt verändert wird, erkennt Python dies nicht als Schreibzugriff und erlaubt dieses Vorgehen. 
\end{exampleblock}
\end{fragile}


\begin{frame}
	
\begin{block}{Warum ist der Zugriff auf den Global Scope eingeschränkt?}
	\pause 
	\begin{itemize}[<+->]
		\item Funktionen sollen möglichst wenige Nebeneffekte haben. Wenn eine Funktion den global Scope verändern kann, ist dies ein großer Nebeneffekt. 
		\item Wenn man eine Funktion schreibt, muss man sich keine Gedanken machen, ob ein Variablenname schon vergeben ist. 
		\item Wenn man sich innerhalb einer Funktion den Kontakt zum global Scope reduziert, so ist die Funktion besser zu verstehen, zu warten und zu testen. 
		\item \dots
	\end{itemize}
\end{block}
	
	
\end{frame}


\section{Input/Output II \\ \footnotesize Lesen und Schreiben von Dateien}

\begin{frame}
	
\begin{block}{Grundprinzip}
\vspace{2pt}
Um mit Dateien zu arbeiten, geht man immer in 3 Schritten vor:
\pause 
\begin{enumerate}
	\item<2-> Datei öffnen 
	\item<3-> Datei bearbeiten (d.h. z.B. lesen, überschreiben, etwas anhängen)
	\item<4-> Datei schließen
\end{enumerate}
\pause \pause \pause
Das Schließen von Dateien ist relativ wichtig, kann aber schnell mal vergessen werden. Daher bietet Python eine spezielle Syntax mithilfe des Keywords \py{with} an. 
\end{block}
\end{frame}


\begin{fragile}
\begin{block}{Gesamten Text einer Datei einlesen}
\vspace{2pt}
\pause 
\begin{minted}{python}
with open("some_file.txt") as my_file:
  my_text = my_file.read()
  print(my_text)
\end{minted}

\pause
\vspace{12pt}

\begin{exampleblock}{Erklärung}
\vspace{2pt}
\begin{itemize}[<+->]
	\item Die Funktion \py{open} öffnet die angegebene Datei (Python geht per se davon aus, dass die Datei im gleichen Ordner wie das ausgeführte Skript liegt).
	\item Ein \emph{Dateiobjekt} wird in der Variable \py{my_file} gespeichert (der Variablenname ist beliebig)
	\item Die Methode \py{.read()} liest den Text-Inhalt der Datei, so dass er in einer Variable gespeichert werden kann 
	\item Sobald der eingerückte Block verlassen wird, wird die Datei automatisch geschlossen
\end{itemize}
\end{exampleblock}

\end{block}
\end{fragile}

\begin{fragile}
\begin{block}{Den Text einer Datei zeilenweise einlesen}
\pause 
\vspace{2pt}

\begin{minted}{python}
with open("some_file.txt") as my_file:
  my_lines = my_file.readlines()
  for line in my_lines:
    print(f"The line reads: {line}")
\end{minted}

\pause
\vspace{12pt}

\begin{exampleblock}{Erklärung}
\vspace{2pt}
\begin{itemize}[<+->]
\item Die Methode \py{.readlines()} gibt eine \emph{Liste} der Zeilen des Inhalts der Datei \py{"some_file.txt"} zurück. 
\item Durch diese Liste kann man mittels einer \pybw{for}-Schleife durchiterieren. 
\end{itemize}
\end{exampleblock}
\end{block}


\end{fragile}


\begin{fragile}
\begin{block}{Text in eine Datei schreiben}
\pause 
\vspace{2pt}

\begin{minted}{python}
with open("some_file.txt", "w") as my_file:
  my_file.write("Hello everybody")
\end{minted}

\pause
\vspace{12pt}

\begin{exampleblock}{Erklärung}
\vspace{2pt}
\begin{itemize}[<+->]
\item Ruft man \py{open} mit dem zweiten Parameter \py{"w"} auf, so wird die Datei im Schreibmodus geöffnet. 
\item Existierte die Datei zuvor noch nicht, so wird sie erzeugt. 
\item Mit der Methode \py{.write("Inhalt")} lässt sich Text in eine Datei schreiben. 
\item Achtung: Öffnet man eine Datei im Schreibmodus, so wird der bisherige Inhalt überschrieben. 
\end{itemize}
\end{exampleblock}
\end{block}
\end{fragile}

\begin{fragile}
\begin{block}{Text an eine Datei anhängen}
\pause 
\vspace{2pt}

\begin{minted}{python}
with open("some_file.txt", "a") as my_file:
  my_file.write("Some text to append")
\end{minted}

\pause
\vspace{12pt}

\begin{exampleblock}{Erklärung}
\vspace{2pt}
\begin{itemize}[<+->]
\item Ruft man \py{open} mit dem zweiten Parameter \py{"a"} auf, so wird die Datei im \emph{Append}-Modus geöffnet. 
\item Existierte die Datei zuvor noch nicht, so wird sie erzeugt. 
\item Mit der Methode \py{.write("Inhalt")} lässt sich Text an die Datei anhängen. 
\item Der bis dahin in der Datei vorhandene Inhalt wird nicht verändert oder gelöscht. 
\item Der einzige Unterschied zum letzten Punkt ist der Modus (\py{"a"} statt \py{"w"}).   
\end{itemize}
\end{exampleblock}
\end{block}

\end{fragile}

\begin{fragile}
	
\begin{alertblock}{Achtung Umlaute}
\vspace{2pt}
Will man Dateien, die Umlauten und andere Sonderzeichen enthalten, bearbeiten, so muss man beim Öffnen der Datei noch den Parameter \pybw{encoding="utf-8"} übergeben. 
\end{alertblock}	

 \vspace{12pt}
 
\begin{exampleblock}{Beispiel}
	\vspace{2pt}
\begin{minted}{python}
with open("some_file.txt", "a", encoding="utf-8") as my_file:
  my_file.write("Hier ein Text mit Umlauten: äöüß")
\end{minted}
\end{exampleblock}

	
\end{fragile}


\section{JSON \\ \footnotesize Ein universelles Datenformat}

\begin{frame}
\metroset{block=fill}

\begin{block}{Definition: JSON}
\vspace{2pt}	
JSON (Java Script Object Notation) ist ein Daten-Format, um Verschachtelungen von Listen und Dictionaries darzustellen, zu speichern und auszutauschen. Die Syntax entspricht (fast) der üblichen Python-Syntax und wird von den meisten Programmiersprachen \enquote{verstanden}.  
\end{block}
\end{frame}

\begin{fragile}
	
\begin{exampleblock}{Beispiel: Eine Liste von Ländern}
\vspace{2pt}
\begin{minted}{python}
[
  {
    "name": "Germany",
    "capital": "Berlin",
    "population": 83190556,
    "cities": ["Berlin", "Hamburg","München","Köln"] 
  },
  {
    "name": "France", 
    "capital": "Paris",
    "population": 67422000
    "cities": ["Paris","Marseilles", "Lyon", "Toulouse"]
  },
  ...
]
\end{minted}
\end{exampleblock}
\end{fragile}

\begin{frame}
\begin{block}{Eigenschaften}
\vspace{2pt}
\pause 
\begin{itemize}[<+->]
	\item Dictionaries und Listen dürfen beliebig verschachtelt werden. 
	\item Die äußerste Ebene kann ein Dictionary oder eine Liste sein. 
	\item Es müssen doppelte Anführungsstriche verwendet werden. 
	\item Neben Dictionaries und Listen können folgende Datentypen verwendet werden: 
	\begin{itemize}
		\item Integer
		\item String
		\item Float
		\item Boolean (\pybw{true} bzw. \pybw{false})
		\item \pybw{null} (entspricht \pybw{None})
	\end{itemize}
\end{itemize}
\end{block}
\end{frame}


\begin{fragile}
	
\begin{block}{Python's JSON-Modul}
\vspace{2pt}
Um in Python Daten im JSON-Format einzulesen und zu speichern, benötigt man das mitgelieferte JSON-\emph{Modul}. 
Dazu einfach die folgende Zeile am Beginn des Python-Skripts anfügen: 

\pause 

\begin{minted}{python}
import json 

...


\end{minted}
\end{block}	

\end{fragile}


\begin{fragile}
	\begin{block}{Daten als JSON-Datei abspeichern}
		\vspace{2pt}
		
		\begin{minted}{python}
		import json 
		
		my_data = {"a": 1, "b": 2}  # some dummy data
		
		with open("my_data.json","w") as my_file:
		  json.dump(my_data, my_file)
		\end{minted}
		
		\pause
		
		\vspace{12pt}
		
		\begin{exampleblock}{Erklärung}
			\vspace{2pt}
			\begin{itemize}[<+->]
				\item Zunächst wird die Datei \pybw{my_data.json} im Schreibmodus geöffnet.  
				\item Die Funktion \pybw{json.dump} erwartet die Daten und eine Datei. Die Daten werden im JSON-Format in der Datei abgespeichert. 
				\item Achtung: Der bisherige Inhalt von \pybw{my_data.json} wird überschrieben.  
			\end{itemize}
		\end{exampleblock}
	\end{block}
\end{fragile}


\begin{fragile}
\begin{block}{Daten aus einer JSON-Datei importieren}
\vspace{2pt}

\begin{minted}{python}
import json 

with open("my_data.json") as my_file:
  data = json.load(my_file)
  
print(data)
\end{minted}

\pause

\vspace{12pt}

\begin{exampleblock}{Erklärung}
\vspace{2pt}
\begin{itemize}[<+->]
\item Zunächst wird die Datei \pybw{my_data.json} im Lesemodus geöffnet.  
\item Die Funktion \pybw{json.load} erwartet eine JSON-Datei und gibt die eingelesenen Daten als Liste bzw. Dictionary zurück.  
\end{itemize}
\end{exampleblock}
\end{block}
\end{fragile}






